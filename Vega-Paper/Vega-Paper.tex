% David Vega and Stan Warford
% Pepperdine University
% File: Vega-Paper

\documentclass[fleqn, leqno]{article}

\usepackage{times}
\usepackage{amsmath, amsthm, amssymb,latexsym}
\usepackage{paralist}
\usepackage{ellipsis}
\usepackage{array}
\usepackage[student]{optional}
%\usepackage[doc]{optional}
%\usepackage[pub]{optional}

\newcommand{\lgap}{2pt}                             % Line gap
\newcommand{\llgap}{6pt}                            % Larger line gap
\newcommand{\lllgap}{32pt}                          % Largest line gap for students to write in
\newcommand{\mymathindent}{24pt}                    % Indentation for math tabbing
\newcommand{\equivs}{\ensuremath{\;\equiv\;}}       % Equivales with space
\newcommand{\equivss}{\ensuremath{\;\;\equiv\;\;}}  % Equivales with double space
\newcommand{\nequiv}{\ensuremath{\not\equiv}}       % Inequivalent
\newcommand{\impl}{\ensuremath{\Rightarrow}}        % Implies
\newcommand{\nimpl}{\ensuremath{\not\Rightarrow}}   % Does not imply
\newcommand{\foll}{\ensuremath{\Leftarrow}}         % Follows from
\newcommand{\nfoll}{\ensuremath{\not\Leftarrow}}    % Does not follow from

% Macros for Temporal Operators
\newcommand{\Until}{\;\mathcal{U}\;}
\newcommand{\Wait}{\;\mathcal{W}\;}
\newcommand{\Next}{\;\,\text{\raisebox{3.5pt}{\circle{6}}}}
\newcommand{\Event}{\Diamond\,}
\newcommand{\Always}{\Box\,}

\newcommand{\myqed}{\hfill\rule[-.23ex]{1.2ex}{2.0ex}}
\newcommand{\spacer}{\vspace{-30pt}}
\newcommand{\firstspacer}{\vspace{-26pt}}

% Thanks to David Gries for sharing the following macros
% Macros for quantifications.
\newcommand{\thedr}{\rule[-.25ex]{.32mm}{1.75ex}}   % Symbol that separates dummy from range in quantification
\newcommand{\dr}{\;\,\thedr\,\;}                    % Symbol that separates dummy from range, with spacing
\newcommand{\rb}{:}                                 % Symbol that separates range from body in quantification
\newcommand{\drrb}{\;\thedr\,{:}\;}                 % Symbol that separates dummy from body when range is missing
\newcommand{\all}{\forall}                          % Universal quantification
\newcommand{\ext}{\exists}                          % Existential quantification

% Macros for proof hints
\newcommand{\Gll} {\langle}                         % Open hint
\newcommand{\Ggg} {\rangle}                         % Close hint
\newlength{\Glllength}                              % Length of open hint symbol
\settowidth{\Glllength}{$.\Gll$}
\newcommand{\Hint}[1]     {\ \ \ $\Gll              \mbox{#1} \Ggg$ }   % Single line hint
\newcommand{\Hintfirst}[1]{\ \ \ $\Gll              \mbox{#1}$ }        % First line of multiline hint
\newcommand{\Hintmid}[1]  {\ \ $\hspace{\Glllength} \mbox{#1}$ }        % Middle line of multiline hint
\newcommand{\Hintlast}[1] {\ \ $\hspace{\Glllength} \mbox{#1} \Ggg$ }   % Last line of multiline hint

% Single and double quotes
\newcommand{\Lq}{\mbox{`}}
\newcommand{\Rq}{\mbox{'}}
\newcommand{\Lqq}{\mbox{``}}
\newcommand{\Rqq}{\mbox{''}}

\oddsidemargin  0.0in
\evensidemargin 0.0in
\textwidth      6.0in
\headheight     0.0in
\topmargin      0.0in
\textheight=8.5in
\parindent=0in
%\pagestyle{plain}

\pagestyle{myheadings} 
\markboth{\textbf{Draft}} {\textbf{Draft}}

\opt{student}{
\title{An Equational Deductive System\\for Linear Temporal Logic (Student)}
}

\opt{doc}{
\title{An Equational Deductive System\\for Linear Temporal Logic (Complete)}
}

\opt{pub}{
\title{An Equational Deductive System\\for Linear Temporal Logic}
}

\author{David Vega\thanks{Research supported by Tooma Undergraduate Research Fellowship Program, Summer 2009
        and academic year 2009-10.}\\
   Computer Science Department\\
   Pepperdine University\\
   Malibu, CA 90265
   \and
   J. Stanley Warford\\
   Computer Science Department\\
   Pepperdine University\\
   Malibu, CA 90265}
\date{} % Required for no date to appear in heading

\begin{document}
\maketitle
\begin{abstract}
This paper presents an equational deductive system for linear temporal logic.
It differs from previous developments of temporal logic in several respects.
First, the proofs are given in the equational logic developed by Dijkstra and Scholten as opposed to the older Hilbert-style logics.
Second, it presents several new and interesting linear temporal theorems, including the associative
properties of the \textit{until} operator.
Third, although space limitations preclude giving a proof of every theorem in this paper,
every theorem has been proved with equational logic.
The summary of equational logic is minimal, and the paper can serve as an introduction to linear temporal logic for those 
already familiar with equational logic.
\end{abstract}

\thispagestyle{plain}

\section{Introduction}

Propositional calculus is a formal system of logic based on the unary operator negation $\neg$,
the binary operators conjunction $\land$, disjunction $\lor$, implies $\impl$ (also written $\rightarrow$),
and equivalence $\equiv$ (also written $\leftrightarrow$),
variables (lowercase letters $p$, $q$, \dots), and the constants $true$ and $false$.
Hilbert-style logic systems, $\mathcal{H}$, are the deductive logic systems traditionally used in mathematics.
In the late 1980's, Dijkstra and Scholten \cite{DandS}, and Feijen \cite{Feij} developed a method of proving
program correctness with a new logic based on an equational style.
This equational deductive system, $\mathcal{E}$, has been the basis of textbooks by Kaldewaij \cite{Kald},
Cohen \cite{Cohen}, and Gries and Schneider \cite{LADM}.\\

The equational deductive system of proofs is slow to be adopted by the computer science community.
The problem is two-fold.
First, a fair amount of technical detail must be mastered,
and many computer science educators and practitioners do not have the requisite
knowledge of formal logic systems, much less the equational deductive system.
Scores of textbooks for discrete mathematics for computer science could be cited that give only a cursory introduction to
formal logic. Most of these texts, such as the classic one by Rosen \cite{Rosen} are beginning to move to a more
formal treatment of logic appropriate for computer science.\\

Second, even when formal logic is taught at a depth necessary to apply it to program proofs, the older Hilbert style
still dominates.
The previously-cited texts \cite{Cohen, LADM, Kald} are among the few that rely on the equational deductive system. 
More typical is Ben-Ari's book \cite{Ben}, which is based enitrely on natural deduction systems.\\

Linear temporal logic describes how the truth values of propositions change over time.
It extends the propositional operators with the unary operators next $\Next$, eventually $\Event$, and always $\Always$,
and the binary operators until $\Until$ and wait $\Wait$.
Propositional calculus applies to program correctness with the formulation of the Hoare triple to establish invariants
in programs that terminate.
Temporal logic applies to program correctness with concurrent processing to establish safety and liveness properties
in programs that possibly do not terminate.\\

As is the case for propositional and predicate calculus, all treatments of linear temporal logic use $\mathcal{H}$
instead of $\mathcal{E}$. Typical are Ben-Ari \cite{Ben2}, Emerson \cite{Emer}, Kr\"{o}ger \cite{Kroger},
Manna and Pnueli \cite{Manna}, and Schneider \cite{Schn}.
The only appearance of an equational proof of a temporal logic theorem appears to be a single example in \cite{Schn},
which otherwise uses a Hilbert-style system for temporal logic.
The development of linear temporal logic in these works is motivated by its use to prove correctness of concurrent programs.
The presentation typically consists of lists of valid temporal formulas, with little emphasis on which formulas are required
as axioms and which are theorems that can be proved using a deductive system.\\

This paper presents an equational deductive system for linear temporal logic.
It differs from previous developments of such systems in several respects.
First, the proofs are given in the equational logic $\mathcal{E}$ instead of $\mathcal{H}$.
Second, it presents several new and interesting linear temporal theorems, including the associative
properties of the \textit{until} operator.
Third, although space limitations preclude giving a proof of every theorem in this paper,
every theorem has been proved with $\mathcal{E}$.\\

Section 2 describes the deductive axioms and the proof rules for $\mathcal{E}$.
It also defines the syntax and semantics of linear temporal logic.
Section 3 presents the equational deductive system for linear temporal logic.\\

\section{Background}

The first section below summarizes the equational system $\mathcal{E}$ from Gries and Schneider \cite{LADM}.
The summary is minimal, and the remainder of the paper assumes familiarity with $\mathcal{E}$.
The second section introduces temporal logic and assumes no prior familiarity with it.
The paper can serve as an introduction to temporal logic for those familiar with $\mathcal{E}$.

\subsection{Equational Deductive Systems}

The definition of an expression has four parts:
\begin{itemize}[$\bullet$]
\item A constant or variable is an expression.
\item If $E$ is an expression, then $(E)$ is an expression.
\item If $\triangleright$ is a unary prefix operator and $E$ is an expression, then $\triangleright E$ is an expression with operand $E$.
\item If $\star$ is a binary infix operator and $D$ and $E$ are expressions, then $D \star E$ is an expression with operands $D$
and $E$.
\end{itemize}

By convention, upper-case letters ({\itshape e.g.\/} $X$, $Y$, \dots) represent expressions,
and lower-case letters ({\itshape e.g.\/} $x$, $y$, \dots) represent variables.
In the propositional calculus, the constants are {\itshape true\/} and {\itshape false\/}.\\

Here is the table of precedences.\\

\setlength\extrarowheight{2pt}
\begin{tabular}{lr}
\hline
$[x := e]$ (textual substitution) & Highest precedence\\
$\neg$\quad $\Next$\quad $\Event$\quad $\Always$ &\\
$\Until$\quad $\Wait$ &\\
$=$\quad (conjunctional) &\\
$\lor$\quad $\land$ &\\
$\impl$\quad $\foll$ &\\
$\equiv$ \quad (associative) & Lowest precedence\\
\hline
\end{tabular}\\[\llgap]

Textual substitution has the highest precedence.
All the unary operators have the next highest precedence.
They are necessarily right associative.
For example, $\neg \Next \neg p$ means $\neg (\Next (\neg p))$.
In this system, two binary operators that have the same precedence require parentheses to disambiguate.
As in \cite{LADM}, conjunction $\land$ and disjunction $\lor$ have the same precedence so that $p\land q\lor r$
must be disambiguated as either $(p\land q)\lor r$ or $p\land (q\lor r)$.
This contrasts with many systems in which conjunction has higher precedence than disjunction.\\

Also consistent with the equational system of \cite{LADM} but different from most other deductive logic systems
is the difference between operators equals $=$ and equivales $\equiv$.
Equals applies to any mathematical type including, {\itshape e.g.\/}, boolean, natural number, and set.
Equivales applies only to boolean, and is commonly denoted $\leftrightarrow$ in other systems.
Another difference is that equals is conjunctive, while equivales is associative.
For example, the expression $p = q = r$ means $(p = q) \land (q = r)$, while the expression $p \equiv q \equiv r$
can be taken as either $(p \equiv q) \equiv r$ or $p \equiv (q \equiv r)$.
This property of equivales is the first axiom in the equational deductive system of \cite{LADM}.
\\

The equational deductive system relies on the three deductive axioms for equality
\[
\textbf{Reflexivity:}\quad x=x
\]
\[
\textbf{Symmetry:}\quad (x=y) = (y=x)
\]
\[
\textbf{Transitivity:}\quad \frac{X=Y, \quad Y=Z}{X=Z}
\]
and the proof rule
\[
\textbf{Leibniz:}\quad \frac{X=Y}{E[z:=X]=E[z:=Y]}
\]

where the square bracket indicates textual substitution of expression $X$ for variable $z$ and substitution
of expression $Y$ for variable $z$.
Roughly speaking, Leibniz allows for the substitution of equals for equals in a proof step.
The general form of a proof step is

\begin{tabbing}
\hspace{\mymathindent} \= $= \;$ \=  \kill
  \> \>   $E[z:=X]$\\[\lgap]
  \> $=$  \>  \Hint{$X=Y$} \\[\lgap]
  \> \>   $E[z:=Y]$
\end{tabbing}

where the expression enclosed in angle brackets $\Gll\;\Ggg$ called the ``hint'' is the justification for the step.
An example of a proof step from the proof of theorem (\ref{E:distNextAnd}) from Section 3.1 is

\begin{tabbing}
\hspace{\mymathindent} \= $= \;$ \= \kill
  \> \>   $\neg\Next (\neg p \lor \neg q)$\\[\lgap]
  \> $=$  \>  \Hint{(\ref{E:distNextOr}) with $p,q := \neg p, \neg q$}\\[\lgap]
  \> \>   $\neg (\Next\neg p \lor \Next \neg q)$
\end{tabbing}

This proof step uses the previously proved theorem (\ref{E:distNextOr}), distributivity of $\Next$ over $\lor$,
which is $\Next (p \lor q) \equiv \Next p \lor \Next q$.
The expressions in Leibniz for the step are

\begin{tabbing}
\hspace{\mymathindent} \= $= \;$ \= \kill
  \> $E:\quad \neg z$\\[\lgap]
  \> $X:\quad \Next (\neg p \lor \neg q)$\\[\lgap]
  \> $Y:\quad \Next \neg p \lor \Next \neg q$
\end{tabbing}

The textual substitutions are

\begin{tabbing}
\hspace{\mymathindent} \= $= \;$ \= \kill
  \> $E[z:=X]:\quad \neg\Next (\neg p \lor \neg q)$\\[\lgap]
  \> $E[z:=Y]:\quad \neg\Next (\neg p \lor \neg q)$
\end{tabbing}

And the justification in the hint $X=Y$ comes from the textual substitution of $\neg p$ for $p$
and $\neg q$ for $q$ in (\ref{E:distNextOr}) as follows

\begin{tabbing}
\hspace{\mymathindent} \= $= \;$ \= \kill
  \> $(\Next (p \lor q) \equiv \Next p \lor \Next q)[p,q:=\neg p, \neg q]:\quad
      \Next (\neg p \lor \neg q) \equiv \Next \neg p \lor \Next \neg q$
\end{tabbing}

Gries and Schneider \cite{LADM} extend the proof format to incorporate implication using its transitive properties
with itself and with equivales.
An example is a proof of (\ref{E:impEvent}) from Section 3.3, $p \impl \Event p$.

\begin{tabbing}
\hspace{\mymathindent} \= $= \;$ \= \kill
  \> \>   $\Event p$\\[\lgap]
  \> $=$  \>  \Hint{(\ref{E:expansionEvent}) Expansion of $\Event$}\\[\lgap]
  \> \>   $p \lor \Next\Event p$\\[\lgap]
  \> $\Leftarrow$  \>  \Hint{(3.76a) Weakening $p\impl p\lor q$ with $q:=\Next\Event p$}\\[\lgap]
  \> \>   $p$
\end{tabbing}

Because $\Event p$ equivales $p \lor \Next\Event p$, and $p \lor \Next\Event p$ follows from $p$, it follows by
transitivity that $\Event p$ follows from $p$. 

\subsection{Temporal Logic}

The operators of propositional calculus, $\neg$, $=$, $\land$, $\lor$, $\impl$, $\foll$, and $\equiv$ are static.
That is, they apply at a single point in time.
Each operator has a truth table that dictates how to evaluate the truth value of an expression.
A state is an assignment of a truth value to each variable in the expression.
A given boolean expression may be false in all states, true in some states and false in others, or true in all states, in which case the expression is known as a theorem or validity or tautology.\\

The operators of temporal logic, $\Next$, $\Event$, $\Always$, $\Until$, and $\Wait$ are dynamic.
That is, they do not apply at a single point in time, but apply over an infinite sequence of states.
Each state corresponds to a discrete point in time that represents one point in the execution of a program,
possibly having several threads running concurrently but whose instruction executions have been serialized.
As one instruction in the program executes, the state changes, and hence the truth value of an expression may change as well.\\

A model $\sigma$ is an infinite sequence of the form

\begin{tabbing}
\hspace{\mymathindent} \= $= \;$ \= \kill
  \> $\sigma: s_0, s_1, s_2, \dots$
\end{tabbing}

where $s_0$ is the initial state and each state $s_i, 0 \le i$ is the state at time $i$.
For example, suppose $x$ is an integer variable whose value varies at each step of the computation.
Then $x$ and the expression $x\ge 10$, known as a state expression, might evolve as follows.\\

\begin{tabular}{c|ccccccc}
  $\sigma$      & $s_0$ & $s_1$ & $s_2$ & $s_3$ & $s_4$ & \dots \\
  \hline
  $x$           & 8     & 9     & 10    & 11    & 12    & \dots\\
  $x\ge 10$     & F     & F     & T     & T     & T     & \dots
\end{tabular}\\

The bottom row shows the evaluation of the state expression for each state in the sequence.
Temporal logic extends propositional logic by considering the evolution of expression evaluations in time.
For example, if you assume that $x$ in the above sequence keeps increasing by one you can assert
informally in English, ``For the sequence $\sigma$, eventually $x\ge 10$ will always be true.''\\

The notation

\begin{tabbing}
\hspace{\mymathindent} \= $= \;$ \= \kill
  \> $(\sigma, j) \models p$
\end{tabbing}

means that the expression $p$ holds at position $j$ in a sequence $\sigma$.
In the above example,

\begin{tabbing}
\hspace{\mymathindent} \= $= \;$ \= \kill
  \> $(\sigma, 3) \models x\ge 10$
\end{tabbing}

the symbol $\models$ means ``satisfies'', so the above expression is read as
``State 3 of sequence $\sigma$ satisfies $x\ge 10$''.
Or, using ``holds'', the same expression is read as, ``$x\ge 10$ holds in state 3 of sequence $\sigma$''.
The following sections use $\models$ to formalize the interpretation of each temporal operator.

\subsubsection*{The \textit{next} operator $\Next$}

The semantics of the unary prefix operator $\Next$ is

\begin{tabbing}
\hspace{\mymathindent} \= $= \;$ \= \kill
  \> $(\sigma, j) \models \Next p$ \quad iff \quad $(\sigma, j+1) \models p$
\end{tabbing}

That is, $\Next p$ holds at position $j$ iff $p$ holds at position $j+1$.\\

For example, in the above sequence $\Next x \ge 10$ holds at state $s_1$ because $x \ge 10$
holds at state $s_2$.
In other words,

\begin{tabbing}
\hspace{\mymathindent} \= $= \;$ \= \kill
  \> $(\sigma, 1) \models \Next x \ge 10$ \quad because \quad $(\sigma, 2) \models  x \ge 10$
\end{tabbing}

\subsubsection*{The \textit{until} operator $\Until$}

The semantics of the binary infix operator $\Until$ is

\begin{tabbing}
\hspace{\mymathindent} \= $= \;$ \= \kill
  \> $(\sigma, j) \models p \Until q$ \quad iff \quad $(\ext k \dr k \ge j \rb (\sigma,k) \models q \land
      (\all i \dr j\le i < k \rb (\sigma,i) \models p))$
\end{tabbing}

If $p \Until q$ holds at state $s_j$, then $p$ holds at state $s_j$ and continues to hold at every state
after $s_j$ until $q$ holds at some future state.
$p \Until q$ guarantees that $q$ will eventually hold at some future state, and that $p$ will continue to
hold until then.
After the state in which $q$ holds for the first time, there are no restrictions on either $p$ or $q$.\\

For example, suppose $x$ and $y$ evolve in the computation as follows.\\

\begin{tabular}{c|ccccccccccc}
  $\sigma$                  & $s_0$ & $s_1$ & $s_2$ & $s_3$ & $s_4$ & $s_5$ & $s_6$ & $s_7$ & $s_8$ & $s_8$ & \dots \\
  \hline
  $x$                       & $-1$  & 0     & 1     & 2     & 3     & 4     & 5     &  6    &  7    &  8    &  \dots\\
  $y$                       & 9     & 8     & 7     & 6     & 5     & 4     & 3     &  2    &  1    &  0    &  \dots\\
  $0<x<y$                   & F     & F     & T     & T     & T     & F     & F     &  F    &  F    &  F    &  \dots\\
  $2\le y<5$                & F     & F     & F     & F     & F     & T     & T     &  T    &  F    &  F    &  \dots\\
  $(0<x<y)\Until(2\le y<5)$ & F     & F     & T     & T     & T     & T     & T     &  T    &  F    &  F    &  \dots\\
\end{tabular}\\

The bottom row shows the evaluation of the expression $p\Until q$ where $p\equiv 0<x<y$ and $q\equiv 2\le y<5$.
In states $s_0$ and $s_1$, $p\Until q$ is false because both $p$ and $q$ are false.
Starting at state $s_2$, $p\Until q$ is true because in that state $p$ is true and will remain true until $q$
eventually becomes true in state $s_5$.\\

From the semantics of $p\Until q$, if $q$ is true in any state, then $p\Until q$ is true in that state regardless of $p$.
For example, not only is $p\Until q$ true in state $s_5$, before which $p$ was true in several preceding states,
it is also true in states $s_6$ and $s_7$, because in those states $q$ is true.
This behavior of $p\Until q$ comes from the empty range and one-point rules \cite{LADM} of the predicate calculus in the case that
$q$ holds in state $s_j$ and $k=j$.

\begin{tabbing}
\hspace{\mymathindent} \= $= \;$ \= \kill
	\> \>   $(\ext k \dr k \ge j \rb (\sigma,k) \models q \land (\all i \dr j\le i < k \rb (\sigma,i) \models p))$\\[\lgap]
	\> $=$  \>  \Hint{Case $k=j$}\\[\lgap]
	\> \>   $(\ext k \dr k=j \rb (\sigma,k) \models q \land (\all i \dr j\le i < j \rb (\sigma,i) \models p))$\\[\lgap]
	\> $=$  \>  \Hint{$j\le i < j \equiv false$}\\[\lgap]
	\> \>   $(\ext k \dr k=j \rb (\sigma,k) \models q \land (\all i \dr false \rb (\sigma,i) \models p))$\\[\lgap]
	\> $=$  \>  \Hint{Empty range rule $(\star x \dr false \rb P) =u$ with $true$ the identity of $\land$}\\[\lgap]
	\> \>   $(\ext k \dr k=j \rb (\sigma,k) \models q \land true$)\\[\lgap]
	\> $=$  \>  \Hint{Identity of $\land$ and one-point rule $(\star x\dr x=E\rb P) = P[x:=E]$}\\[\lgap]
	\> \>   $((\sigma,k) \models q)[k := j]$\\[\lgap]
	\> $=$  \>  \Hint{Textual substitution}\\[\lgap]
	\> \>   $(\sigma,j) \models q$\\[\lgap]
	\> $=$  \>  \Hint{Case $q$ holds in state $s_j$}\\[\lgap]
	\> \>   $true$
\end{tabbing}

This result is theorem (\ref{E:zeroUntil}) $p \Until true \equiv true$ proved in the next section.
$true$ is the right zero of the until operator.\\

\opt{doc}{
The until operator $\Until$ is not associative as shown by the following sequence.\\

\begin{tabular}{c|cccccccc}
  $\sigma$                  & $s_0$ & $s_1$ & $s_2$ & $s_3$ & $s_4$ & \dots \\
  \hline
  $p$                       & F     & T     & F     & T     & F     &  \dots\\
  $q$                       & F     & F     & T     & F     & T     &  \dots\\
  $r$                       & F     & F     & F     & F     & T     &  \dots\\
  $q\Until r$               & F     & F     & F     & F     & T     &  \dots\\
  $p\Until q$               & F     & T     & T     & T     & T     &  \dots\\
  $p\Until (q\Until r)$     & F     & F     & F     & T     & T     &  \dots\\
  $(p\Until q)\Until r$     & F     & T     & T     & T     & T     &  \dots\\
\end{tabular}\\[\llgap]

However $p\Until (q\Until r) \impl (p\Until q)\Until r$ as follows.
Expanding $p\Until (q\Until r)$,

\begin{tabbing}
\hspace{\mymathindent} \= $= \;$ \= \kill
	\> \>   $(\sigma,j) \models p\Until (q\Until r)$\\[\lgap]
	\> $=$  \>  \Hint{Definition of $\Until$}\\[\lgap]
	\> \>   $(\ext k \dr k\ge j \rb (\sigma,k) \models (q \Until r) \land (\all i \dr j\le i < k \rb (\sigma, i) \models p))$\\[\lgap]
	\> $=$  \>  \Hint{Definition of $\Until$}\\[\lgap]
	\> \>   $(\ext k \dr k\ge j \rb (\ext n \dr n\ge k \rb (\sigma,n)\models r \land (\all m \dr k\le m < n \rb (\sigma,m)\models q))$\\[\lgap]
	\> \>   $\quad \quad \land \; (\all i \dr j\le i < k \rb (\sigma, i) \models p)\;)$
\end{tabbing}

The following figure illustrates the temporal segments for which $p$, $q$, and $r$ hold for $p\Until(q\Until r)$.

\begin{picture}(360,120)
\thicklines
\put(320,110){$(\sigma,n)\models r$}
\put(0,100)  {\line(1,0){360}}
\put(20,100) {\circle*{8}} \put(180,100) {\circle*{8}} \put(340,100) {\circle*{8}}
\put(18,86)  {$j$} \put(178,86)  {$k$} \put(338,86)  {$n$} 
\put(168,74) {$k\ge j$} \put(328,74)  {$n\ge k$}
\put(180,52) {\line(1,0){156}}
\put(180,52) {\circle*{8}} \put(340,52) {\circle{8}}
\put(240,58) {$(\sigma,m)\models q$}
\put(240,42) {$k\le m < n$}
\put(20,20)  {\line(1,0){156}}
\put(20,20)  {\circle*{8}} \put(180,20) {\circle{8}}
\put(80,28) {$(\sigma,i)\models p$}
\put(80,10) {$j\le i < k$}
\end{picture}

To prove that $p\Until (q\Until r) \impl (p\Until q)\Until r$ it suffices to show that $(p\Until q)$ holds between $j$ and $n$,
given that $p$, $q$, and $r$ hold as above.\\

Case 1. $\quad (p\Until q)$ holds at $(\sigma,i)$ for $j\le i \le k$.\\
$(p\Until q)$ holds in this case, because $p$ holds at $(\sigma,i)$ for $j\le i < k$, and $q$ holds at $(\sigma,k)$.\\

Case 2. $\quad (p\Until q)$ holds at $(\sigma,m)$ for $k<m<n$.\\
$(p\Until q)$ holds in this case, because $q$ holds at $(\sigma,m)$ for $k<m<n$, and $true$ is the right zero of $\Until$.\\

From Case 1 and Case 2, $(p\Until q)$ holds at $(\sigma,i)$ for $j\le i<n$.
Because $r$ holds at $(\sigma,n)$, $(p\Until q)\Until r$ holds between $j$ and $n$ inclusively, which completes the proof.
This theorem does not seem to appear in the linear temporal logic literature.
It is the basis of axiom (\ref{E:untilAssocImp}) $p \Until (q \Until r) \impl (p \Until q) \Until r$ assumed in the next section.
}

\opt{student}{
\textbf{Exercise 1.}\\

 Fill in the blank entries in the table below.

\begin{tabular}{c|cccccccc}
  $\sigma$                  & $s_0$ & $s_1$ & $s_2$ & $s_3$ & $s_4$ & \dots \\
  \hline
  $p$                       & F     & T     & F     & T     & F     &  \dots\\
  $q$                       & F     & F     & T     & F     & T     &  \dots\\
  $r$                       & F     & F     & F     & F     & T     &  \dots\\
  $q\Until r$               &       &       &       &       &       &  \dots\\
  $p\Until q$               &       &       &       &       &       &  \dots\\
  $p\Until (q\Until r)$     &       &       &       &       &       &  \dots\\
  $(p\Until q)\Until r$     &       &       &       &       &       &  \dots\\
\end{tabular}\\

From the table, do you believe the until operator $\Until$ is associative?

}
\subsubsection*{The \textit{eventually} operator $\Event$}

The semantics of the unary prefix operator $\Event$ is

\begin{tabbing}
\hspace{\mymathindent} \= $= \;$ \= \kill
  \> $(\sigma, j) \models \Event p$ \quad iff \quad $(\ext k \dr k \ge j \rb (\sigma,k) \models p)$
\end{tabbing}

So, $\Event p$ holds in state $s_j$ if $p$ holds in state $s_j$ or in any other state $s_k$ where $k\ge j$,
that is, if $p$ holds in the current state or in any other future state.\\

For example, suppose $x$ evolves in the computation as follows.\\

\begin{tabular}{c|cccccccc}
  $\sigma$                  & $s_0$ & $s_1$ & $s_2$ & $s_3$ & $s_4$ & $s_5$ & $s_6$ & \dots \\
  \hline
  $x$                       & 1     & 2     & 3     & 4     & 5     & 6     & 7     &  \dots\\
  $3\le x<6$                & F     & F     & T     & T     & T     & F     & F     &  \dots\\
  $\Event(3\le x<6)$        & T     & T     & T     & T     & T     & F     & F     &  \dots\\
\end{tabular}\\

The bottom row shows the evaluation of the expression $\Event p$ where $p\equiv 3\le x<6$.
In states $s_0$ and $s_1$, $\Event p$ is true because there is a state, either now or in the future, in which $p$ will hold.\\

If $\Event p$ is ever false in any state $s_i$ in a sequence $\sigma$, it must be false in all subsequent states $s_j$, $j\ge i$.
If $\Event p$ is ever true in any state $s_i$ in a sequence $\sigma$, it must be true in all preceding states $s_j$, $j\le i$.
For example, suppose $p$ and $q$ evolve in the computation as follows.\\

\begin{tabular}{c|ccccccccccc}
  $\sigma$       & $s_0$ & $s_1$ & $s_2$ & $s_3$ & $s_4$ & $s_5$ & $s_6$ & $s_7$ & $s_8$& $s_9$  & \dots \\
  \hline
  $p$            & F     & F     & T     & F     & F     & T     & F     & F     & F     & F     &  \dots\\
  $q$            & F     & F     & T     & T     & F     & F     & T     & T     & F     & F     &  \dots\\
  $\Event p$     & T     & T     & T     & T     & T     & T     & F     & F     & F     & F     &  \dots\\
  $\Event q$     & T     & T     & T     & T     & T     & T     & T     & T     & T     & T     &  \dots\\
\end{tabular}\\

The bottom two rows show the evaluation of the expressions $\Event p$ and $\Event q$
assuming that $p$ remains false indefinitely and $q$ continues to switch between true and false indefinitely.\\

The eventually operator is a special case of the until operator.
Namely, $true \Until q$ is equivalent to $\Event q$ as follows.

\begin{tabbing}
\hspace{\mymathindent} \= $= \;$ \= \kill
	\> \>   $(\sigma, j) \models true\Until q$\\[\lgap]
	\> $=$  \>  \Hint{Semantics of $p\Until q$ with $p:=true$}\\[\lgap]
	\> \>   $(\ext k \dr k \ge j \rb (\sigma,k) \models q \land (\all i \dr j\le i < k \rb (\sigma,i) \models true))$\\[\lgap]
	\> $=$  \>  \Hint{$true$ holds in all states}\\[\lgap]
	\> \>   $(\ext k \dr k \ge j \rb (\sigma,k) \models q \land (\all i \dr j\le i < k \rb true))$\\[\lgap]
	\> $=$  \>  \Hint{Theorem (9.8) from \cite{LADM}, $(\all x\dr R\rb true)\equiv true$}\\[\lgap]
	\> \>   $(\ext k \dr k \ge j \rb (\sigma,k) \models q \land true)$\\[\lgap]
	\> $=$  \>  \Hint{Identity of $\land$}\\[\lgap]
	\> \>   $(\ext k \dr k \ge j \rb (\sigma,k) \models q)$\\[\lgap]
	\> $=$  \>  \Hint{Semantics of $\Event q$}\\[\lgap]
	\> \>   $(\sigma, j) \models \Event q$
\end{tabbing}

This relationship is the basis of the definition of $\Event p$ in equation (\ref{E:defEvent}) $\Event p \equiv true \Until p$
assumed in the next section.

\subsubsection*{The \textit{always} operator $\Always$}

The semantics of the unary prefix operator $\Always$ is

\begin{tabbing}
\hspace{\mymathindent} \= $= \;$ \= \kill
  \> $(\sigma, j) \models \Always p$ \quad iff \quad $(\all k \dr k \ge j \rb (\sigma,k) \models p)$
\end{tabbing}

So, $\Always p$ holds in state $s_j$ if $p$ holds in state $s_j$ and in all other states $s_k$ where $k\ge j$,
that is, if $p$ holds in the current state and in all other future states.
For example, suppose $x$ evolves in the computation as follows.\\

\begin{tabular}{c|ccccccccccc}
  $\sigma$                      & $s_0$ & $s_1$ & $s_2$ & $s_3$ & $s_4$ & $s_5$ & $s_6$ & $s_7$ & \dots \\
  \hline
  $x$                           & 1     & 2     & 3     & 4     & 5     & 6     & 7     & 8     &  \dots\\
  $x < 4 \lor x \ge 6$          & T     & T     & T     & F     & F     & T     & T     & T     &  \dots\\
  $\Always(x < 4 \lor x \ge 6)$ & F     & F     & F     & F     & F     & T     & T     & T     &  \dots\\
\end{tabular}\\

The bottom row shows the evaluation of the expression $\Always p$ where $p\equiv x < 4 \lor x \ge 6$.
In states $s_3$ and $s_4$, $\Always p$ is false because $p$ does not hold in those states.
In states $s_0$, $s_1$, and $s_2$, $p$ is true. However, $\Always p$ is false in those states because $p$ does no hold
in all future states.
In states $s_5$, $s_6$, $s_7$, and subsequent states, $\Always p$ is true because $p$ holds in in those states
and in all future states as well.\\

If $\Always p$ is ever true in any state $s_i$ in a sequence $\sigma$, it must be true in all subsequent states $s_j$, $j\ge i$.
If $\Always p$ is ever false in any state $s_i$ in a sequence $\sigma$, it must be false in all preceding states $s_j$, $j\le i$.
For example, suppose $p$ and $q$ evolve in the computation as follows.\\

\begin{tabular}{c|ccccccccccc}
  $\sigma$       & $s_0$ & $s_1$ & $s_2$ & $s_3$ & $s_4$ & $s_5$ & $s_6$ & $s_7$ & $s_8$& $s_9$  & \dots \\
  \hline
  $p$            & T     & T     & F     & T     & T     & F     & T     & T     & T     & T     &  \dots\\
  $q$            & T     & T     & F     & F     & T     & T     & F     & F     & T     & T     &  \dots\\
  $\Always p$    & F     & F     & F     & F     & F     & F     & T     & T     & T     & T     &  \dots\\
  $\Always q$    & F     & F     & F     & F     & F     & F     & F     & F     & F     & F     &  \dots\\
\end{tabular}\\

The bottom two rows show the evaluation of the expressions $\Always p$ and $\Always q$
assuming that $p$ remains true indefinitely and $q$ continues to switch between true and false indefinitely.\\

$\Event p$ is an existential operator, while $\Always p$ is a universal operator.
They are related through the generalized De Morgan theorem \cite{LADM} $\neg (\ext x\dr R\rb \neg P)\equiv (\all x\dr R\rb P)$
as follows.

\begin{tabbing}
\hspace{\mymathindent} \= $= \;$ \= \kill
	\> \>   $(\sigma, j) \models \Always p$\\[\lgap]
	\> $=$  \>  \Hint{Semantics of $\Always p$}\\[\lgap]
	\> \>   $(\all k \dr k \ge j \rb (\sigma,k) \models p)$\\[\lgap]
	\> $=$  \>  \Hint{Generalized De Morgan $\neg (\ext x\dr R\rb \neg P)\equiv (\all x\dr R\rb P)$}\\[\lgap]
	\> \>   $\neg (\ext k \dr k \ge j \rb \neg((\sigma,k) \models p))$\\[\lgap]
	\> $=$  \>  \Hint{$p$ does not hold in a state iff $\neg p$ holds in that state}\\[\lgap]
	\> \>   $\neg (\ext k \dr k \ge j \rb (\sigma,k) \models \neg p)$\\[\lgap]
	\> $=$  \>  \Hint{Semantics of $\Event q$}\\[\lgap]
	\> \>   $\neg ((\sigma, j) \models \Event \neg q)$\\[\lgap]
	\> $=$  \>  \Hint{$p$ does not hold in a state iff $\neg p$ holds in that state}\\[\lgap]
	\> \>   $(\sigma, j) \models \neg \Event \neg q$
\end{tabbing}

This relationship is the basis of the definition of $\Always p$ in equation (\ref{E:defAlways})
$\Always p \equiv \neg\Event\neg p$ assumed in the next section.\\

The above demonstration that $(\sigma, j) \models \Always p \equivs (\sigma, j) \models \neg \Event \neg q$ depends on the
rule, ``$p$ does not hold in a state iff $\neg p$ holds in that state'', written formally as

\begin{tabbing}
\hspace{\mymathindent} \= $= \;$ \= \kill
  \> $\neg ((\sigma, j) \models p)$ \quad iff \quad $(\sigma, j) \models \neg p$
\end{tabbing}

The corresponding rules for the binary operators are

\begin{tabbing}
\hspace{\mymathindent} \= $= \;$ \= \kill
  \> $((\sigma, j) \models p) \;\land\; ((\sigma, j) \models q)$ \quad iff \quad $(\sigma, j) \models p\land q$\\
  \> $((\sigma, j) \models p) \;\lor\; ((\sigma, j) \models q)$ \quad iff \quad $(\sigma, j) \models p\lor q$\\
  \> $((\sigma, j) \models p) \;\impl\; ((\sigma, j) \models q)$ \quad iff \quad $(\sigma, j) \models p \impl q$\\
  \> $((\sigma, j) \models p) \;\equiv\; ((\sigma, j) \models q)$ \quad iff \quad $(\sigma, j) \models p \equiv q$
\end{tabbing}

\subsubsection*{The \textit{wait} operator $\Wait$}

The semantics of the binary infix operator $\Wait$ in terms of $\Until$ and $\Always$ is

\begin{tabbing}
\hspace{\mymathindent} \= $= \;$ \= \kill
  \> $(\sigma, j) \models p \Wait q$ \quad iff \quad $(\sigma, j) \models p \Until q \; \lor \; (\sigma, j) \models \Always p$
\end{tabbing}

The wait operator $\Wait$ is weaker than the until operator $\Until$, because $p\Wait q$ does not require $q$ to ever be true,
while $p\Until q$ does.
For example, suppose $p$ and $q$ evolve in the computation as follows.\\

\begin{tabular}{c|cccccccccccc}
  $\sigma$       & $s_0$ & $s_1$ & $s_2$ & $s_3$ & $s_4$ & $s_5$ & $s_6$ & $s_7$ & $s_8$& $s_9$  & $s_{10}$&  \dots \\
  \hline
  $p$            & F     & F     & T     & T     & F     & F     & F     & F     & T     & T     & T     &  \dots\\
  $q$            & F     & F     & F     & F     & T     & T     & F     & F     & F     & F     & F     &  \dots\\
  $\Always p$    & F     & F     & F     & F     & F     & F     & F     & F     & T     & T     & T     &  \dots\\
  $p\Until q$    & F     & F     & T     & T     & T     & T     & F     & F     & F     & F     & F     &  \dots\\
  $p\Wait q$     & F     & F     & T     & T     & T     & T     & F     & F     & T     & T     & T     &  \dots\\
\end{tabular}\\

The bottom two rows show the evaluation of the expressions $p\Until q$ and $p\Wait q$
assuming that $p$ remains true indefinitely and $q$ remains false indefinitely.
From $s_0$ to $s_7$, $p\Until q$ and $p\Wait q$ hold in the same states.
From $s_8$ on, however, $p\Until q$ does not hold because $q$ never holds thereafter,
while $p\Wait q$ does hold because $p$ always holds thereafter.

\section{The Equational Temporal System}

This section presents an axiomatic deductive system of temporal logic and proves its theorems with the equational
logic $\mathcal{E}$ of \cite{LADM}.
Theorems cited in a proof hint take two forms.
A numbered reference enclosed in parentheses \textit{without} a decimal point is a reference to an axiom or a previously-proved
theorem in this paper.
A numbered reference enclosed in parentheses \textit{with} a decimal point is a reference to an axiom or a
theorem from the propositional calculus in \cite{LADM}.
The terms ``definition'' and ``axiom'' are synonymous.\\

\subsection{Next}

The following two axioms define the next operator $\Next$.
\begin{equation}\label{E:selfDual}
\textbf{Axiom, Self-dual:}\quad \Next\neg p \equiv \neg\Next p
\end{equation}
%\spacer
\begin{equation}\label{E:distNextImp}
\textbf{Axiom, Distributivity of $\Next$ over $\impl$:}\quad \Next (p \impl q) \equiv \Next p \impl \Next q
\end{equation}

Self duality states that $p$ not holding in the next state is equivalent to next $p$ not holding in the current state.
Neither of the other unary temporal operators, eventually $\Event$ nor always $\Always$, are equivalent to their duals.
The next operator describes the state fixed at one step into the future.
In contrast, $\Event$ and $\Always$ describe sequences of states arbitrarily far into the future.\\

Distributivity states that $p$ implies $q$ in the next state is equivalent to next $p$ implies next $q$ in the current state.
From this axiom, subsequent theorems prove that the next operator distributes over all the propositional binary operators.\\

Linearity follows from self-dual and distributivity of $\Next$ over $\impl$.
\begin{equation}\label{E:linearity}
\textbf{Linearity:}\quad \Next p \equiv \neg\Next\neg p
\end{equation}

\emph{Proof:}
\begin{tabbing}
\hspace{\mymathindent} \= $= \;$ \= \kill
  \> \>   $\Next p \equiv \neg\Next\neg p$\\[\lgap]
  \> $=$  \>  \Hint{(3.11) $\neg p \equiv q \equiv p \equiv \neg q$ with $p,q := \Next\neg p, \Next p$} \\[\lgap]
  \> \>   $\neg\Next p \equiv \Next\neg p$
\end{tabbing}
which is (\ref{E:selfDual}), Self-dual. \myqed\\[\lgap]

Here are proofs that $\Next$ distributes over $\lor$, $\land$, and $\equiv$.
\begin{equation}\label{E:distNextOr}
\textbf{Distributivity of $\Next$ over $\lor$:}\quad \Next (p \lor q) \equiv \Next p \lor \Next q
\end{equation}

\emph{Proof:}
\begin{tabbing}
\hspace{\mymathindent} \= $= \;$ \= \kill
	\> \>   $\Next(p \lor q)$\\[\lgap]
	\> $=$  \>  \Hint{(3.59) Implication $p\impl q \equivs \neg p \lor q$}\\[\lgap]
	\> \>   $\Next(\neg p \impl q)$\\[\lgap]
	\> $=$  \>  \Hint{(\ref{E:distNextImp}) Distributivity of $\Next$ over $\impl$}\\[\lgap]
	\> \>   $\Next\neg p \impl \Next q$\\[\lgap]
	\> $=$  \>  \Hint{(3.59) Implication $p\impl q \equivs \neg p \lor q$ with $p,q := \Next\neg p, \Next p$}\\[\lgap]
	\> \>   $\neg\Next\neg p \lor \Next q$\\[\lgap]
	\> $=$  \>  \Hint{(\ref{E:linearity}) Linearity and (3.12) Double negation $\neg\neg p\equiv p$}\\[\lgap]
	\> \>   $\Next p \lor \Next q$
\end{tabbing}
\myqed\\[\lgap]

\begin{equation}\label{E:distNextAnd}
\textbf{Distributivity of $\Next$ over $\land$:}\quad \Next (p \land q) \equiv \Next p \land \Next q
\end{equation}

\emph{Proof:}
\begin{tabbing}
\hspace{\mymathindent} \= $= \;$ \= \kill
  \> \>   $\Next (p \land q)$\\[\lgap]
  \> $=$  \>  \Hint{(3.12) Double Negation $\neg\neg p\equiv p$, twice}\\[\lgap]
  \> \>   $\Next (\neg\neg p \land \neg\neg q)$\\[\lgap]
  \> $=$  \>  \Hint{(3.47b) De Morgan $\neg (p \lor q) \equiv \neg p \land \neg q$}\\[\lgap]
  \> \>   $\Next\neg(\neg p \lor \neg q)$\\[\lgap]
  \> $=$  \>  \Hint{(\ref{E:selfDual}) with $p:= (\neg p \lor \neg q$)}\\[\lgap]
  \> \>   $\neg\Next (\neg p \lor \neg q)$\\[\lgap]
  \> $=$  \>  \Hint{(\ref{E:distNextOr}) with $p,q := \neg p, \neg q$}\\[\lgap]
  \> \>   $\neg (\Next\neg p \lor \Next \neg q)$\\[\lgap]
  \> $=$  \>  \Hint{(\ref{E:selfDual}) twice}\\[\lgap]
  \> \>   $\neg(\neg\Next p \lor \neg\Next q)$\\[\lgap]
  \> $=$  \>  \Hint{(3.47a) De Morgan $\neg (p \land q) \equiv \neg p \lor \neg q$}\\[\lgap]
  \> \>   $\neg\neg(\Next p \land \Next q)$\\[\lgap]
  \> $=$  \>  \Hint{(3.12) Double Negation $\neg\neg p\equiv p$}\\[\lgap]
  \> \>   $\Next p \land \Next q$
\end{tabbing}
\myqed\\[\lgap]

\begin{equation}\label{E:distNextEquiv}
\textbf{Distributivity of $\Next$ over $\equiv$:}\quad \Next (p \equiv q) \equiv \Next p \equiv \Next q
\end{equation}

\emph{Proof:}
\opt{doc}{
\begin{tabbing}
\hspace{\mymathindent} \= $= \;$ \= \kill
  \> \>   $\Next (p \equiv q)$\\[\lgap]
  \> $=$  \>  \Hint{(3.80) Mutual Implication $(p\impl q) \land (q\impl p) \equivs (p\equiv q)$}\\[\lgap]
  \> \>   $\Next ((p \impl q) \land (q \impl p))$\\[\lgap]
  \> $=$  \>  \Hint{(\ref{E:distNextAnd}) Distributivity of $\Next$ over $\land$}\\[\lgap]
  \> \>   $\Next (p \impl q) \land \Next (p \impl q)$\\[\lgap]
  \> $=$  \>  \Hint{(\ref{E:distNextImp}) Distributivity of $\Next$ over $\impl$}\\[\lgap]
  \> \>   $(\Next p \impl \Next q) \land (\Next q \impl \Next p)$\\[\lgap]
  \> $=$  \>  \Hint{(3.80) Mutual Implication $(p\impl q) \land (q\impl p) \equivs (p\equiv q)$}\\[\lgap]
  \> \>   $\Next p \equiv \Next q$
\end{tabbing}
\myqed\\[\lgap]
}

\opt{student}{
Exercise for the student. Hint: Start with mutual implication.\\
}

Now, $true$ holds in the next state, and $false$ does not hold in the next state.
Theorems (\ref{E:nextTruth}) and (\ref{E:nextFalse}) are unique to this system.
In equational logic, $true$ is theorem (3.4) and is equivalent to all other theorems.
Theorem (\ref{E:nextTruth}) shows that all propositional logic theorems hold at the next state
and, by induction, hold at all states.

\begin{equation}\label{E:nextTruth}
\textbf{Truth:}\quad \Next true \equiv true
\end{equation}

\emph{Proof:}
\begin{tabbing}
\hspace{\mymathindent} \= $= \;$ \= \kill
	\> \>   $\Next true$\\[\lgap]
	\> $=$  \>  \Hint{(3.28) Excluded middle $p\lor\neg p$}\\[\lgap]
	\> \>   $\Next(p \lor \neg p)$\\[\lgap]
	\> $=$  \>  \Hint{(\ref{E:distNextOr}) Distributivty of $\Next$ over $\lor$}\\[\lgap]
	\> \>   $\Next p \lor \Next\neg p$\\[\lgap]
	\> $=$  \>  \Hint{(\ref{E:selfDual}) Self-dual}\\[\lgap]
	\> \>   $\Next p \lor \neg\Next p$\\[\lgap]
	\> $=$  \>  \Hint{(3.28) Excluded middle $p\lor\neg p$}\\[\lgap]
	\> \>   $true$
\end{tabbing}
\myqed\\[\lgap]

\begin{equation}\label{E:nextFalse}
\textbf{Falsehood:}\quad \Next false \equiv false
\end{equation}

\emph{Proof:}
\opt{doc}{
\begin{tabbing}
\hspace{\mymathindent} \= $= \;$ \= \kill
  \> \>   $\Next false \equiv false$\\[\lgap]
  \> $=$  \>  \Hint{(3.8) Definition of $false$, $false\equiv \neg true$} \\[\lgap]
  \> \>   $\Next\neg true \equiv \neg true$\\[\lgap]
  \> $=$  \>  \Hint{(3.11) $\neg p \equiv q \equiv p \equiv \neg q$ with $p,q := true, \Next\neg true$}\\[\lgap]
  \> \>   $\neg\Next\neg true \equiv true$\\[\lgap]
  \> $=$  \>  \Hint{(\ref{E:linearity}) Linearity}\\[\lgap]
  \> \>   $\Next true \equiv true$
\end{tabbing}
which is (\ref{E:nextTruth}). \myqed\\[\lgap]
}

\opt{student}{
Exercise for the student.\\
}

\subsection{Until}

This system defines the until operator $\Until$ with the following ten axioms.
The distributivity of next over until (\ref{E:distNextUntil}) implies the distributivity of next over wait as Section (3.5) shows.
Thus next distributes over all binary operators, both propositional and temporal.\\

The until operator is unlike any propositional binary operator.
Its right operand has an existential characteristic and its left operand has a universal characteristic.
Expansion (\ref{E:expansionUntil}) states that
$p\Until q$ is true iff $q$ is true in the current state, or $p$ is true in the current state and $p\Until q$ is
true in the next state.
Thus, $q$ relates to the definition through disjunction, which is existential,
while $p$ relates through conjunction, which is universal.
Consequently, the until operator is neither symmetric (\textit{i.e.} commutative) nor associative.
Axiom (\ref{E:untilAssocImp}) shows that the associativity holds in only one direction.
\begin{equation}\label{E:distNextUntil}
\textbf{Axiom, Distributivity of $\Next$ over $\Until$:}\quad \Next (p \Until q) \equiv \Next p \Until \Next q
\end{equation}

\firstspacer

\begin{equation}\label{E:expansionUntil}
\textbf{Axiom, Expansion of $\Until$:}\quad p \Until q \equiv q \lor (p \land \Next (p \Until q))
\end{equation}

\firstspacer

\begin{equation}\label{E:untilFalse}
\textbf{Axiom, Right zero of $\Until$:}\quad p \Until false \equiv false
\end{equation}

\firstspacer

\begin{equation}\label{E:untilAssocImp}
\textbf{Axiom:}\quad p \Until (q \Until r) \impl (p \Until q) \Until r
\end{equation}

The following four axioms describe how the $\Until$ operator distributes over conjunction and disjunction.
Because $\Until$ is not symmetric, this system requires separate axioms for left and right distributivity.
\begin{equation}\label{E:untilOrImp}
\textbf{Axiom:}\quad (p \Until r) \lor (q \Until r) \impl (p \lor q) \Until r
\end{equation}

\firstspacer

\begin{equation}\label{E:untilAndImp}
\textbf{Axiom:}\quad p \Until (q \land r) \impl (p \Until q) \land (p \Until r)
\end{equation}

\firstspacer

\begin{equation}\label{E:untilAndEquiv}
\textbf{Axiom:}\quad (p \land q) \Until r \equiv (p \Until r) \land (q \Until r)
\end{equation}

\firstspacer

\begin{equation}\label{E:untilOrEquiv}
\textbf{Axiom:}\quad p \Until (q \lor r) \equiv (p \Until q) \lor (p \Until r)
\end{equation}

The following two axioms are necessary to prove both directions of the absorption laws for the $\Until$ operator.
\begin{equation}\label{E:untilIdemAx}
\textbf{Axiom:}\quad p \Until q \impl p \Until (p \Until q)
\end{equation}

\firstspacer

\begin{equation}\label{E:untilIdemRAX}
\textbf{Axiom:}\quad (p \Until q) \Until q \impl p \Until q
\end{equation}

Theorem (\ref{E:idemUntil}) shows that the $\Until$ operator is idempotent.
Theorem (\ref{E:zeroUntil}) shows that $true$ is a right zero of $\Until$, which is unusual because
axiom (\ref{E:untilFalse}) shows that $false$ is also a right zero of $\Until$.
Theorem (\ref{E:leftIdUntil}) shows that $false$ is the left identity of $\Until$.
Theorems (\ref{E:untilFalse}), (\ref{E:zeroUntil}), and (\ref{E:leftIdUntil}) cover three of the
possibilities of constants $true$ and $false$ on either side of $\Until$.
None of these three theorems seem to appear in the temporal logic literature.
The fourth possibility with $true$ as the left argument is the basis of the definition of the eventually
operator $\Event$ in Section 3.3.
\begin{equation}\label{E:idemUntil}
\textbf{Idempotency of $\Until$:}\quad p \Until p \equiv p
\end{equation}

\emph{Proof:}
\begin{tabbing}
\hspace{\mymathindent} \= $= \;$ \= \kill
  \> \>   $p \Until p$\\[\lgap]
  \> $=$  \>  \Hint{(\ref{E:expansionUntil}) Expansion of $\Until$}\\[\lgap]
  \> \>   $p \lor (p \land \Next(p \Until p))$\\[\lgap]
  \> $=$  \>  \Hint{(3.43b) Absorption, $p \lor (p \land q) \equiv p$}\\[\lgap]
  \> \>   $p$
\end{tabbing}
\myqed\\[\lgap]

\begin{equation}\label{E:zeroUntil}
\textbf{Right zero of $\Until$:}\quad p \Until true \equiv true
\end{equation}

\emph{Proof:}
\opt{doc}{
\begin{tabbing}
\hspace{\mymathindent} \= $= \;$ \= \kill
  \> \>   $p \Until true$\\[\lgap]
  \> $=$  \>  \Hint{(\ref{E:expansionUntil}) Expansion of $\Until$}\\[\lgap]
  \> \>   $true \lor (p \land \Next(p \Until true))$\\[\lgap]
  \> $=$  \>  \Hint{(3.29) Zero of $\lor$, $p\lor true\equiv true$}\\[\lgap]
  \> \>   $true$
\end{tabbing}
\myqed\\[\lgap]
}

\opt{student}{
Exercise for the student.\\
}

\begin{equation}\label{E:leftIdUntil}
\textbf{Left identity of $\Until$:}\quad false \Until p \equiv p
\end{equation}

\emph{Proof:}
\opt{doc}{
\begin{tabbing}
\hspace{\mymathindent} \= $= \;$ \= \kill
  \> \>   $false \Until p$\\[\lgap]
  \> $=$  \>  \Hint{(\ref{E:expansionUntil}) Expansion of $\Until$}\\[\lgap]
  \> \>   $p \lor (false \land \Next(false \Until p))$\\[\lgap]
  \> $=$  \>  \Hint{(3.40) Zero of $\land$, $p\land false\equiv false$}\\[\lgap]
  \> \>   $p \lor false$\\[\lgap]
  \> $=$  \>  \Hint{(3.30) Identity of $\lor$, $p\lor false\equiv p$}\\[\lgap]
  \> \>   $p$\\[\lgap]
\end{tabbing}
\myqed\\[\lgap]
}

\opt{student}{
Exercise for the student.\\
}

Theorem (\ref{E:untilImpOr}) is interesting because it relates the temporal expression on the left hand side
to the propositional expression on the right hand side.
\begin{equation}\label{E:untilImpOr}
p \Until q \impl p \lor q
\end{equation}

\emph{Proof:}
\opt{doc}{
\begin{tabbing}
\hspace{\mymathindent} \= $= \;$ \= \kill
  \> \>   $p \Until q$\\[\lgap]
  \> $=$  \>  \Hint{(\ref{E:expansionUntil}) Expansion of $\Until$}\\[\lgap]
  \> \>   $q \lor (p \land \Next(p \Until q))$\\[\lgap]
  \> $\impl$  \>  \Hint{(3.76d) $p\lor (q\land r) \impl p\lor q$ with $p,q,r := q,p,\Next(p \Until q)$}\\[\lgap]
  \> \>   $p \lor q$
\end{tabbing}
\myqed\\[\lgap]
}

\opt{student}{
Exercise for the student.\\
}

The following two theorems are the absorption laws of the $\Until$ operator.
All systems have these two absorption theorems, because the $\Until$ operator is not associative.
Manna and Pnueli \cite{Manna} refer to these as idempotence properties.
This paper follows Schneider \cite{Schn}, which refers to them as absorption properties.
\begin{equation}\label{E:untilIdem}
\textbf{Left absorption of $\Until$:}\quad p \Until (p \Until q) \equiv p \Until q
\end{equation}

\emph{Proof:} The proof is by mutual implication (3.80), $(p\impl q) \land (q\impl p) \equivs (p\equiv q)$.
The proof in the first direction follows.
\begin{tabbing}
\hspace{\mymathindent} \= $= \;$ \= \kill
  \> \>   $p \Until (p \Until q)$\\[\lgap]
  \> $=$  \>  \Hint{(\ref{E:untilAssocImp})}\\[\lgap]
  \> \>   $(p \Until p) \Until q$\\[\lgap]
  \> $\impl$  \>  \Hint{(\ref{E:idemUntil}) Idempotency of $\Until$}\\[\lgap]
  \> \>   $p \Until q$
\end{tabbing}

Axiom (\ref{E:untilIdemAx}) supplies the proof in the second direction.
\myqed\\[\lgap]

\begin{equation}\label{E:untilIdemR}
\textbf{Right absorption of $\Until$:}\quad (p \Until q) \Until q \equiv p \Until q
\end{equation}

\emph{Proof:} Also by mutual implication.
The proof in the first direction follows.
\begin{tabbing}
\hspace{\mymathindent} \= $= \;$ \= \kill
  \> \>   $p \Until q$\\[\lgap]
  \> $=$  \>  \Hint{(\ref{E:idemUntil}) Idempotency of $\Until$}\\[\lgap]
  \> \>   $p \Until (q \Until q)$\\[\lgap]
  \> $\impl$  \>  \Hint{(\ref{E:untilAssocImp})}\\[\lgap]
  \> \>   $(p \Until q) \Until q$
\end{tabbing}

Axiom (\ref{E:untilIdemRAX}) supplies the proof in the second direction.
\myqed\\[\lgap]

This system has the following five additional absorption properties that do not seem to
appear in the temporal logic literature.
\begin{equation}\label{E:untilOrP}
\textbf{Absorption:}\quad p \lor (p \Until q) \equiv p \lor q
\end{equation}

\emph{Proof:}
\begin{tabbing}
\hspace{\mymathindent} \= $= \;$ \= \kill
  \> \>   $p \lor (p \Until q)$\\[\lgap]
  \> $=$  \>  \Hint{(\ref{E:expansionUntil}) Expansion of $\Until$}\\[\lgap]
  \> \>   $p \lor q \lor (p \land \Next(p \Until q))$\\[\lgap]
  \> $=$  \>  \Hint{(3.43b) Absorption $p \lor (p \land q) \equiv p$}\\[\lgap]
  \> \>   $p \lor q$
\end{tabbing}
\myqed\\[\lgap]

\begin{equation}\label{E:untilOrQ}
\textbf{Absorption:}\quad (p \Until q) \lor q \equiv p \Until q
\end{equation}

\emph{Proof:}
\begin{tabbing}
\hspace{\mymathindent} \= $= \;$ \= \kill
  \> \>   $(p \Until q) \lor q$\\[\lgap]
  \> $=$  \>  \Hint{(\ref{E:expansionUntil}) Expansion of $\Until$}\\[\lgap]
  \> \>   $q \lor (p \land \Next(p \Until q)) \lor q$\\[\lgap]
  \> $=$  \>  \Hint{(3.26) Idempotency of $\lor$, $p \lor p \equiv p$}\\[\lgap]
  \> \>   $q \lor (p \land \Next(p \Until q))$\\[\lgap]
  \> $=$  \>  \Hint{(\ref{E:expansionUntil}) Expansion of $\Until$}\\[\lgap]
  \> \>   $p \Until q$
\end{tabbing}
\myqed\\[\lgap]

\begin{equation}\label{E:untilAndQ}
\textbf{Absorption:}\quad (p \Until q) \land q \equiv q
\end{equation}

\emph{Proof:}
\begin{tabbing}
\hspace{\mymathindent} \= $= \;$ \= \kill
  \> \>   $(p \Until q) \land q$\\[\lgap]
  \> $=$  \>  \Hint{(\ref{E:expansionUntil}) Expansion of $\Until$}\\[\lgap]
  \> \>   $(q \lor (p \land \Next(p \Until q))) \land q$\\[\lgap]
  \> $=$  \>  \Hint{(3.43a) Absorption $p \land (p \lor q) \equiv p$}\\[\lgap]
  \> \>   $q$
\end{tabbing}
\myqed\\[\lgap]

\begin{equation}\label{E:untilAndOr}
\textbf{Absorption:}\quad (p \Until q) \land (p \lor q) \equiv p \Until q
\end{equation}

\emph{Proof:}
\begin{tabbing}
\hspace{\mymathindent} \= $= \;$ \= \kill
  \> \>   $(p \Until q) \land (p \lor q)$\\[\lgap]
  \> $=$  \>  \Hint{(\ref{E:expansionUntil}) Expansion of $\Until$}\\[\lgap]
  \> \>   $(q \lor (p \land \Next(p \Until q))) \land (p \lor q)$\\[\lgap]
  \> $=$  \>  \Hint{(3.45) Distributivity of $\lor$ over $\land$, $p\lor (q\land r)\equiv (p\lor q)\land (p\lor r)$}\\[\lgap]
  \> \>   $q \lor (p \land p \land \Next(p \Until q))$\\[\lgap]
  \> $=$  \>  \Hint{(3.38) Idempotency of $\land$, $p \land p \equiv p$}\\[\lgap]
  \> \>   $q \lor (p \land \Next(p \Until q))$\\[\lgap]
  \> $=$  \>  \Hint{(\ref{E:expansionUntil}) Expansion of $\Until$}\\[\lgap]
  \> \>   $p \Until q$
\end{tabbing}
\myqed\\[\lgap]

\begin{equation}\label{E:untilOrAnd}
\textbf{Absorption:}\quad (p \Until q) \lor (p \land q) \equiv p \Until q
\end{equation}

\emph{Proof:}
\begin{tabbing}
\hspace{\mymathindent} \= $= \;$ \= \kill
  \> \>   $(p \Until q) \lor (p \land q)$\\[\lgap]
  \> $=$  \>  \Hint{(\ref{E:expansionUntil}) Expansion of $\Until$}\\[\lgap]
  \> \>   $q \lor (p \land \Next(p \Until q)) \lor (p \land q)$\\[\lgap]
  \> $=$  \>  \Hint{(3.43b) Absorption $p \lor (p \land q) \equiv p$}\\[\lgap]
  \> \>   $q \lor (p \land \Next(p \Until q))$\\[\lgap]
  \> $=$  \>  \Hint{(\ref{E:expansionUntil}) Expansion of $\Until$}\\[\lgap]
  \> \>   $p \Until q$
\end{tabbing}
\myqed\\[\lgap]

\subsection{Eventually}

Eventually $\Event$ is a special case of $\Until$ when the left hand side is $true$.
Equation (\ref{E:defEvent}) is its only defining axiom.
\begin{equation}\label{E:defEvent}
\textbf{Definition of $\Event$:}\quad \Event p \equiv true \Until p
\end{equation}
$p\Until q$ guarantees that $q$ will eventually be $true$ as follows.
\begin{equation}\label{E:eventuality}
\textbf{Eventuality:}\quad p \Until q \impl \Event q
\end{equation}

\emph{Proof:}
\begin{tabbing}
\hspace{\mymathindent} \= $= \;$ \= \kill
  \> \>   $p \Until q$\\[\lgap]
  \> $\impl$  \>  \Hint{(3.76a) Weakening, $p\impl p\lor q$}\\[\lgap]
  \> \>   $(p \Until q) \lor (true \Until q)$\\[\lgap]
  \> $\impl$  \>  \Hint{(\ref{E:untilOrImp})}\\[\lgap]
  \> \>   $(p \lor true) \Until q$\\[\lgap]
  \> $=$  \>  \Hint{(3.29) Zero of $\lor$, $p\lor true\equiv true$}\\[\lgap]
  \> \>   $true \Until q$\\[\lgap]
  \> $=$  \>  \Hint{(\ref{E:defEvent}) Definition of $\Event$}\\[\lgap]
  \> \>   $\Event q$
\end{tabbing}
\myqed\\[\lgap]

Theorems (\ref{E:eventTrue}) and (\ref{E:eventFalse}), Truth and Falsehood, are unique to this system.
\begin{equation}\label{E:eventTrue}
\textbf{Truth:}\quad \Event true \equiv true
\end{equation}

\emph{Proof:}
\opt{doc}{
\begin{tabbing}
\hspace{\mymathindent} \= $= \;$ \= \kill
  \> \>   $\Event true$\\[\lgap]
  \> $=$  \>  \Hint{(\ref{E:defEvent}) Definition of $\Event$}\\[\lgap]
  \> \>   $true \Until true$\\[\lgap]
  \> $=$  \>  \Hint{(\ref{E:idemUntil}) Idempotency of $\Until$}\\[\lgap]
  \> \>   $true$
\end{tabbing}
\myqed\\[\lgap]
}

\opt{student}{
Exercise for the student.\\
}

\begin{equation}\label{E:eventFalse}
\textbf{Falsehood:}\quad \Event false \equiv false
\end{equation}

\emph{Proof:}
\opt{doc}{
\begin{tabbing}
\hspace{\mymathindent} \= $= \;$ \= \kill
  \> \>   $\Event false$\\[\lgap]
  \> $=$  \>  \Hint{(\ref{E:defEvent}) Definition of $\Event$}\\[\lgap]
  \> \>   $true \Until false$\\[\lgap]
  \> $=$  \>  \Hint{(\ref{E:untilFalse})}\\[\lgap]
  \> \>   $false$
\end{tabbing}
\myqed\\[\lgap]
}

\opt{student}{
Exercise for the student.\\
}

Expansion of $\Event$, like expansion of $\Until$, has two disjuncts.
The first describes the current state and the second contains the operation in the next state.
\begin{equation}\label{E:expansionEvent}
\textbf{Expansion of $\Event$:}\quad \Event p \equiv p \lor \Next\Event p
\end{equation}

\emph{Proof:}
\opt{doc}{
\begin{tabbing}
\hspace{\mymathindent} \= $= \;$ \= \kill
  \> \>   $\Event p$\\[\lgap]
  \> $=$  \>  \Hint{(\ref{E:defEvent}) Definition of $\Event$}\\[\lgap]
  \> \>   $true \Until p$\\[\lgap]
  \> $=$  \>  \Hint{(\ref{E:expansionUntil}) Expansion of $\Until$}\\[\lgap]
  \> \>   $p \lor (true \land \Next(true \Until p))$\\[\lgap]
  \> $=$  \>  \Hint{(\ref{E:defEvent}) Definition of $\Event$}\\[\lgap]
  \> \>   $p \lor (true \land \Next\Event p)$\\[\lgap]
  \> $=$  \>  \Hint{(3.39) Identity of $\land$, $p\land true\equiv p$}\\[\lgap]
  \> \>   $p \lor \Next\Event p$
\end{tabbing}
\myqed\\[\lgap]
}

\opt{student}{
Exercise for the student.\\
}

The following six theorems for $\Event$ are common to all temporal logic systems.
\begin{equation}\label{E:impEvent}
\textbf{Weakening of $\Event$:}\quad p \impl \Event p
\end{equation}

\emph{Proof:}
\begin{tabbing}
\hspace{\mymathindent} \= $= \;$ \= \kill
  \> \>   $\Event p$\\[\lgap]
  \> $=$  \>  \Hint{(\ref{E:expansionEvent}) Expansion of $\Event$}\\[\lgap]
  \> \>   $p \lor \Next\Event p$\\[\lgap]
  \> $\Leftarrow$  \>  \Hint{(3.76a) Weakening, $p\impl p\lor q$}\\[\lgap]
  \> \>   $p$
\end{tabbing}
\myqed\\[\lgap]


\begin{equation}\label{E:nextEvent}
\textbf{Weakening of $\Event$:}\quad \Next p \impl \Event p
\end{equation}

\emph{Proof:}
\begin{tabbing}
\hspace{\mymathindent} \= $= \;$ \= \kill
  \> \>   $\Event p$\\[\lgap]
  \> $=$  \>  \Hint{(\ref{E:expansionEvent}) Expansion of $\Event$}\\[\lgap]
  \> \>   $p \lor \Next\Event p$\\[\lgap]
  \> $=$  \>  \Hint{(\ref{E:expansionEvent}) Expansion of $\Event$}\\[\lgap]
  \> \>   $p \lor \Next (p \lor \Next\Event p)$\\[\lgap]
  \> $=$  \>  \Hint{(\ref{E:distNextOr}) Distributivity of $\Next$ over $\lor$}\\[\lgap]
  \> \>   $p \lor \Next p \lor \Next\Next\Event p$\\[\lgap]
  \> $\Leftarrow$  \>  \Hint{(3.76a) Weakening, $p\impl p\lor q$}\\[\lgap]
  \> \>   $\Next p$
\end{tabbing}
\myqed\\[\lgap]


\begin{equation}\label{E:IdemEvent}
\textbf{Absorption of $\Event$:}\quad \Event\Event p \equiv \Event p
\end{equation}

\emph{Proof:}
\opt{doc}{
\begin{tabbing}
\hspace{\mymathindent} \= $= \;$ \= \kill
  \> \>   $\Event\Event p$\\[\lgap]
  \> $=$  \>  \Hint{(\ref{E:defEvent}) Definition of $\Event$, with $p := \Event p$}\\[\lgap]
  \> \>   $true \Until \Event p$\\[\lgap]
  \> $=$  \>  \Hint{(\ref{E:defEvent}) Definition of $\Event$}\\[\lgap]
  \> \>   $true \Until (true \Until p)$\\[\lgap]
  \> $=$  \>  \Hint{(\ref{E:untilIdem}) with $p,q := true,p$}\\[\lgap]
  \> \>   $true \Until p$\\[\lgap]
  \> $=$  \>  \Hint{(\ref{E:defEvent}) Definition of $\Event$}\\[\lgap]
  \> \>   $\Event p$
\end{tabbing}
\myqed\\[\lgap]
}

\opt{student}{
Exercise for the student.\\
}

\begin{equation}\label{E:dNextEvent}
\textbf{Exchange of $\Next$ and $\Event$:}\quad \Next\Event p \equiv \Event\Next p
\end{equation}

\emph{Proof:}
\begin{tabbing}
\hspace{\mymathindent} \= $= \;$ \= \kill
  \> \>   $\Next\Event p$\\[\lgap]
  \> $=$  \>  \Hint{(\ref{E:defEvent}) Definition of $\Event$}\\[\lgap]
  \> \>   $\Next(true \Until p)$\\[\lgap]
  \> $=$  \>  \Hint{(\ref{E:distNextUntil}) Distributivity of $\Next$ over $\Until$}\\[\lgap]
  \> \>   $\Next true \Until \Next p$\\[\lgap]
  \> $=$  \>  \Hint{(\ref{E:nextTruth})}\\[\lgap]
  \> \>   $true \Until \Next p$\\[\lgap]
  \> $=$  \>  \Hint{(\ref{E:defEvent}) Definition of $\Event$}\\[\lgap]
  \> \>   $\Event\Next p$
\end{tabbing}
\myqed\\[\lgap]


\begin{equation}\label{E:distEventOr}
\textbf{Distributivity of $\Event$ over $\lor$:}\quad \Event(p \lor q) \equiv \Event p \lor \Event q
\end{equation}

\emph{Proof:}
\opt{doc}{
\begin{tabbing}
\hspace{\mymathindent} \= $= \;$ \= \kill
  \> \>   $\Event(p \lor q)$\\[\lgap]
  \> $=$  \>  \Hint{(\ref{E:defEvent}) Definition of $\Event$}\\[\lgap]
  \> \>   $true \Until (p \lor q)$\\[\lgap]
  \> $=$  \>  \Hint{(\ref{E:untilOrEquiv})}\\[\lgap]
  \> \>   $(true \Until p) \lor (true \Until q)$\\[\lgap]
  \> $=$  \>  \Hint{(\ref{E:defEvent}) Definition of $\Event$ twice}\\[\lgap]
  \> \>   $\Event p \lor \Event q$
\end{tabbing}
\myqed\\[\lgap]
}

\opt{student}{
Exercise for the student.\\
}

\begin{equation}\label{E:distEventAnd}
\textbf{Distributivity of $\Event$ over $\land$:}\quad \Event(p \land q) \impl \Event p \land \Event q
\end{equation}

\emph{Proof:}
\begin{tabbing}
\hspace{\mymathindent} \= $= \;$ \= \kill
  \> \>   $\Event(p \land q)$\\[\lgap]
  \> $=$  \>  \Hint{(\ref{E:defEvent}) Definition of $\Event$}\\[\lgap]
  \> \>   $true \Until (p \land q)$\\[\lgap]
  \> $\impl$  \>  \Hint{(\ref{E:untilAndImp})}\\[\lgap]
  \> \>   $(true \Until p) \land (true \Until q)$\\[\lgap]
  \> $=$  \>  \Hint{(\ref{E:defEvent}) Definition of $\Event$ twice}\\[\lgap]
  \> \>   $\Event p \land \Event q$
\end{tabbing}
\myqed\\[\lgap]

\subsection{Always}

This system defines the \textit{always} operator $\Always$ in terms of the \textit{eventually} operator $\Event$.
$\Always p$ is true when $p$ is true in the current state and in all future states.
The defining equation (\ref{E:defAlways}) states that $p$ is always true iff it is not the case that $\neg p$ is eventually true.
\begin{equation}\label{E:defAlways}
\textbf{Definition of $\Always$:}\quad \Always p \equiv \neg\Event\neg p
\end{equation}

The following induction axiom is included in other temporal logic systems but is not used in any subsequent proofs in this system.
\begin{equation}\label{E:induction}
\textbf{Axiom, Induction:}\quad \Always (p \impl \Next p) \impl (p \impl \Always p)
\end{equation}

The following two axioms are required to prove the theorems where $\Always\Event$ and $\Event\Always$ distribute over $\equiv$.
\begin{equation}\label{E:distAlwaysEventOrAx}
\textbf{Axiom, Distributivity of $\Always\Event$ over $\lor$:}\quad \Always\Event(p \lor q) \impl \Always\Event p \lor \Always\Event q
\end{equation}

\begin{equation}\label{E:distEventAlwaysAndAx}
\textbf{Axiom, Distributivity of $\Event\Always$ over $\land$:}\quad \Event\Always p \land \Event\Always q \impl \Event\Always(p \land q)
\end{equation}

The following theorem expresses $\Event p$ in terms of $\Always p$ and mirrors the defining equation (\ref{E:defAlways}).
\begin{equation}\label{E:eventAsAlways}
\Event p \equiv \neg\Always\neg p
\end{equation}

\emph{Proof:}
\opt{doc}{
\begin{tabbing}
\hspace{\mymathindent} \= $= \;$ \= \kill
  \> \>   $\neg\Always\neg p$\\[\lgap]
  \> $=$  \>  \Hint{(\ref{E:defAlways}) Definition of $\Always$}\\[\lgap]
  \> \>   $\neg\neg\Event\neg\neg p$\\[\lgap]
  \> $=$  \>  \Hint{(3.12) Double Negation $\neg\neg p\equiv p$, twice}\\[\lgap]
  \> \>   $\Event p$
\end{tabbing}
\myqed\\[\lgap]
}

\opt{student}{
Exercise for the student.\\
}

Whereas the \textit{next} operator $\Next$ is its own dual, the \textit{eventually} operator $\Event$
and the \textit{always} operator $\Always$
are mutually dual.
\begin{equation}\label{E:dualAlways}
\textbf{Dual of $\Always$:}\quad \neg\Always p \equiv \Event\neg p
\end{equation}

\emph{Proof:}
\begin{tabbing}
\hspace{\mymathindent} \= $= \;$ \= \kill
  \> \>   $\neg\Always p \equiv \Event\neg p$\\[\lgap]
  \> $=$  \>  \Hint{(3.11) $\neg p \equiv q \equiv p \equiv \neg q$ with $p,q := \Always p, \Event\neg p$}\\[\lgap]
  \> \>   $\Always p \equiv \neg\Event\neg p$
\end{tabbing}
which is (\ref{E:defAlways}). \myqed\\[\lgap]

\begin{equation}\label{E:dualEvent}
\textbf{Dual of $\Event$:}\quad \neg\Event p \equiv \Always\neg p
\end{equation}

\emph{Proof:}
\begin{tabbing}
\hspace{\mymathindent} \= $= \;$ \= \kill
  \> \>   $\Always\neg p$\\[\lgap]
  \> $=$  \>  \Hint{(\ref{E:defAlways}) Definition of $\Always$}\\[\lgap]
  \> \>   $\neg\Event\neg\neg p$\\[\lgap]
  \> $=$  \>  \Hint{(3.12) Double Negation $\neg\neg p\equiv p$}\\[\lgap]
  \> \>   $\neg\Event p$
\end{tabbing}
\myqed\\[\lgap]

Theorems (\ref{E:alwaysTrue}) and (\ref{E:alwaysFalse}), Truth and Falsehood, are unique to this system.
\begin{equation}\label{E:alwaysTrue}
\textbf{Truth:}\quad \Always true \equiv true
\end{equation}

\emph{Proof:}
\begin{tabbing}
\hspace{\mymathindent} \= $= \;$ \= \kill
  \> \>   $\Always true$\\[\lgap]
  \> $=$  \>  \Hint{(\ref{E:defAlways}) Definition of $\Always$}\\[\lgap]
  \> \>   $\neg\Event\neg true$\\[\lgap]
  \> $=$  \>  \Hint{(3.8) Definition of $false$, $false\equiv \neg true$}\\[\lgap]
  \> \>   $\neg\Event false$\\[\lgap]
  \> $=$  \>  \Hint{(\ref{E:eventFalse}) Falsehood}\\[\lgap]
  \> \>   $\neg false$\\[\lgap]
  \> $=$  \>  \Hint{(3.13) Negation of $false$, $\neg false\equiv true$}\\[\lgap]
  \> \>   $true$
\end{tabbing}
\myqed\\[\lgap]

\begin{equation}\label{E:alwaysFalse}
\textbf{Falsehood:}\quad \Always false \equiv false
\end{equation}

\emph{Proof:}
\opt{doc}{
\begin{tabbing}
\hspace{\mymathindent} \= $= \;$ \= \kill
  \> \>   $\Always false \equiv false$\\[\lgap]
  \> $=$  \>  \Hint{(3.8) Definition of $false$, $false\equiv \neg true$, twice}\\[\lgap]
  \> \>   $\Always\neg true \equiv \neg true$\\[\lgap]
  \> $=$  \>  \Hint{(3.11) $\neg p \equiv q \equiv p \equiv \neg q$}\\[\lgap]
  \> \>   $\neg\Always\neg\ true \equiv true$\\[\lgap]
  \> $=$  \>  \Hint{(\ref{E:eventAsAlways})}\\[\lgap]
  \> \>   $\Event true \equiv true$
\end{tabbing}
which is (\ref{E:eventTrue}) Truth. \myqed\\[\lgap]
}

\opt{student}{
Exercise for the student.\\
}

While the expansions of $\Until$ and $\Event$ have two disjuncts,
the expansion of $\Always$ has two conjuncts.
As usual, the first describes the current state and the second contains the operation in the next state.
\begin{equation}\label{E:expansionAlways}
\textbf{Expansion of $\Always$:}\quad \Always p \equiv p \land \Next\Always p
\end{equation}

\emph{Proof:}
\begin{tabbing}
\hspace{\mymathindent} \= $= \;$ \= \kill
  \> \>   $\Always p$\\[\lgap]
  \> $=$  \>  \Hint{(\ref{E:defAlways}) Definition of $\Always$}\\[\lgap]
  \> \>   $\neg\Event\neg p$\\[\lgap]
  \> $=$  \>  \Hint{(\ref{E:defEvent}) Definition of $\Event$ with $p := \neg p$}\\[\lgap]
  \> \>   $\neg(true \Until \neg p)$\\[\lgap]
  \> $=$  \>  \Hint{(\ref{E:expansionUntil}) Expansion of $\Until$}\\[\lgap]
  \> \>   $\neg(\neg p \lor (true \land \Next(true \Until \neg p)))$\\[\lgap]
  \> $=$  \>  \Hint{(3.39) Identity of $\land$, $p\land true\equiv p$}\\[\lgap]
  \> \>   $\neg(\neg p \lor \Next(true \Until \neg p))$\\[\lgap]
  \> $=$  \>  \Hint{(3.47b) De Morgan $\neg (p \lor q) \equiv \neg p \land \neg q$}\\[\lgap]
  \> \>   $\neg\neg p \land \neg\Next(true \Until \neg p)$\\[\lgap]
  \> $=$  \>  \Hint{(3.12) Double Negation $\neg\neg p\equiv p$}\\[\lgap]
  \> \>   $p \land \neg\Next(true \Until \neg p)$\\[\lgap]
  \> $=$  \>  \Hint{(\ref{E:defEvent}) Defintion of $\Event$}\\[\lgap]
  \> \>   $p \land \neg\Next\Event\neg p$\\[\lgap]
  \> $=$  \>  \Hint{(\ref{E:dualAlways}) Dual of $\Always$}\\[\lgap]
  \> \>   $p \land \neg\Next\neg\Always p$\\[\lgap]
  \> $=$  \>  \Hint{(\ref{E:linearity}) Linearity}\\[\lgap]
  \> \>   $p \land \Next\Always p$
\end{tabbing}
\myqed\\[\lgap]

The absorption of $\Always$ mirrors (\ref{E:IdemEvent}) the absorption of $\Event$,
and the exchange of $\Next$ and $\Always$ mirrors (\ref{E:dNextEvent}) the exchange of $\Next$ and $\Event$.
\begin{equation}\label{E:IdemAlways}
\textbf{Absorption of $\Always$:}\quad \Always\Always p \equiv \Always p
\end{equation}

\emph{Proof:}
\opt{doc}{
\begin{tabbing}
\hspace{\mymathindent} \= $= \;$ \= \kill
  \> \>   $\Always\Always p \equiv \Always p$\\[\lgap]
  \> $=$  \>  \Hint{(\ref{E:defAlways}) Definition of $\Always$ with $p := \Always p$}\\[\lgap]
  \> \>   $\neg\Event\neg\Always p \equiv \Always p$\\[\lgap]
  \> $=$  \>  \Hint{(3.11) $\neg p \equiv q \equiv p \equiv \neg q$ with $p,q := \Event\neg\Always p, \Always p$}\\[\lgap]
  \> \>   $\Event\neg\Always p \equiv \neg\Always p$\\[\lgap]
  \> $=$  \>  \Hint{(\ref{E:dualAlways}) Dual of $\Always$, twice}\\[\lgap]
  \> \>   $\Event\Event\neg p \equiv \Event\neg p$\\[\lgap]
  \> $=$  \>  \Hint{(\ref{E:IdemEvent}) Absorption of $\Event$}\\[\lgap]
  \> \>   $\Event\neg p \equiv \Event\neg p$
\end{tabbing}
which is (3.5) Reflexivity of $\equiv$, $p\equiv p$ with $p := \Event\neg p$. \myqed\\[\lgap]
}

\opt{student}{
Exercise for the student.\\
}

\begin{equation}\label{E:dNextAlways}
\textbf{Exchange of $\Next$ and $\Always$:}\quad \Next\Always p \equiv \Always\Next p
\end{equation}

\emph{Proof:}
\opt{doc}{
\begin{tabbing}
\hspace{\mymathindent} \= $= \;$ \= \kill
  \> \>   $\Next\Always p$\\[\lgap]
  \> $=$  \>  \Hint{(\ref{E:defAlways}) Definition of $\Always$}\\[\lgap]
  \> \>   $\Next\neg\Event\neg p$\\[\lgap]
  \> $=$  \>  \Hint{(\ref{E:selfDual}) Self-dual}\\[\lgap]
  \> \>   $\neg\Next\Event\neg p$\\[\lgap]
  \> $=$  \>  \Hint{(\ref{E:dNextEvent}) with $p := \neg p$}\\[\lgap]
  \> \>   $\neg\Event\Next\neg p$\\[\lgap]
  \> $=$  \>  \Hint{(\ref{E:selfDual}) Self-dual}\\[\lgap]
  \> \>   $\neg\Event\neg\Next p$\\[\lgap]
  \> $=$  \>  \Hint{(\ref{E:defAlways}) Definition of $\Always$}\\[\lgap]
  \> \>   $\Always\Next p$
\end{tabbing}
\myqed\\[\lgap]
}

\opt{student}{
Exercise for the student.\\
}

Theorem (\ref{E:impNext}) is unique to this system.
\begin{equation}\label{E:impNext}
p \impl \Always p \equiv p \impl \Next\Always p
\end{equation}

\emph{Proof:}
\opt{doc}{
\begin{tabbing}
\hspace{\mymathindent} \= $= \;$ \= \kill
  \> \>   $p\impl \Always p$\\[\lgap]
  \> $=$  \>  \Hint{(\ref{E:expansionAlways}) Expansion of $\Always$}\\[\lgap]
  \> \>   $p\impl p\land \Next\Always p$\\[\lgap]
  \> $=$  \>  \Hint{(3.59) Implication $p\impl q \equivs \neg p \lor q$}\\[\lgap]
  \> \>   $\neg p\lor (p\land \Next\Always p)$\\[\lgap]
  \> $=$  \>  \Hint{(3.12) Double negation $\neg\neg p\equiv p$}\\[\lgap]
  \> \>   $\neg p\lor (\neg\neg p\land \Next\Always p)$\\[\lgap]
  \> $=$  \>  \Hint{(3.44b) Absorption, $p \lor (\neg p \land q) \equiv p \lor q$ with $p,q := \neg p,\Next\Always p$}\\[\lgap]
  \> \>   $\neg p\lor \Next\Always p$\\[\lgap]
  \> $=$  \>  \Hint{(3.59) Implication $p\impl q \equivs \neg p \lor q$}\\[\lgap]
  \> \>   $p \impl \Next\Always p$
\end{tabbing}
\myqed\\[\lgap]
}

\opt{student}{
Exercise for the student.\\
}

The following four strengthening theorems for $\Always$ contrast with the weakening theorems
(\ref{E:impEvent}) and (\ref{E:nextEvent})
for $\Event$.
\begin{equation}\label{E:impAlways}
\textbf{Strengthening of $\Always$:}\quad \Always p \impl p
\end{equation}

\emph{Proof:}
\begin{tabbing}
\hspace{\mymathindent} \= $= \;$ \= \kill
  \> \>   $\Always p$\\[\lgap]
  \> $=$  \>  \Hint{(\ref{E:defAlways}) Definition of $\Always$}\\[\lgap]
  \> \>   $\neg\Event\neg p$\\[\lgap]
  \> $=$  \>  \Hint{(\ref{E:expansionEvent}) Expansion of $\Event$}\\[\lgap]
  \> \>   $\neg(\neg p \lor \Next\Event\neg p)$\\[\lgap]
  \> $=$  \>  \Hint{(3.47b) De Morgan $\neg (p \lor q) \equiv \neg p \land \neg q$}\\[\lgap]
  \> \>   $\neg\neg p \land \neg\Next\Event\neg p$\\[\lgap]
  \> $=$  \>  \Hint{(3.12) Double Negation $\neg\neg p\equiv p$}\\[\lgap]
  \> \>   $p \land \neg\Next\Event\neg p$\\[\lgap]
  \> $\impl$  \>  \Hint{(3.76b) Strengthening, $p\land q \impl p$}\\[\lgap]
  \> \>   $p$
\end{tabbing}
\myqed\\[\lgap]

\begin{equation}\label{E:impAlwaysE}
\textbf{Strengthening of $\Always$:}\quad \Always p \impl \Event p
\end{equation}

\emph{Proof:}
\opt{doc}{
\begin{tabbing}
\hspace{\mymathindent} \= $= \;$ \= \kill
  \> \>   $\Always p$\\[\lgap]
  \> $\impl$  \>  \Hint{(\ref{E:impAlways}) Strengthening of $\Always$}\\[\lgap]
  \> \>   $p$\\[\lgap]
  \> $\impl$  \>  \Hint{(\ref{E:impEvent}) Weakening of $\Event$}\\[\lgap]
  \> \>   $\Event p$
\end{tabbing}
\myqed\\[\lgap]
}

\opt{student}{
Exercise for the student.\\
}

\begin{equation}\label{E:impAlwaysN}
\textbf{Strengthening of $\Always$:}\quad \Always p \impl \Next p
\end{equation}

\emph{Proof:}
\opt{doc}{
\begin{tabbing}
\hspace{\mymathindent} \= $= \;$ \= \kill
  \> \>   $\Always p$\\[\lgap]
  \> $=$  \>  \Hint{(\ref{E:expansionAlways}) Expansion of $\Always$}\\[\lgap]
  \> \>   $p \land \Next\Always p$\\[\lgap]
  \> $=$  \>  \Hint{(\ref{E:dNextAlways})}\\[\lgap]
  \> \>   $p \land \Always\Next p$\\[\lgap]
  \> $=$  \>  \Hint{(\ref{E:expansionAlways}) Expansion of $\Always$ with $p := \Next p$}\\[\lgap]
  \> \>   $p \land \Next p \land \Next\Always\Next p$\\[\lgap]
  \> $\impl$  \>  \Hint{(3.76b) Strengthening, $p\land q \impl p$}\\[\lgap]
  \> \>   $\Next p$
\end{tabbing}
\myqed\\[\lgap]
}

\opt{student}{
Exercise for the student.\\
}

\begin{equation}\label{E:impAlwaysNA}
\textbf{Strengthening of $\Always$:}\quad \Always p \impl \Next\Always p
\end{equation}

\emph{Proof:}
\opt{doc}{
\begin{tabbing}
\hspace{\mymathindent} \= $= \;$ \= \kill
  \> \>   $\Always p$\\[\lgap]
  \> $=$  \>  \Hint{(\ref{E:expansionAlways}) Expansion of $\Always$}\\[\lgap]
  \> \>   $p \land \Next\Always p$\\[\lgap]
  \> $\impl$  \>  \Hint{(3.76b) Strengthening, $p\land q \impl p$}\\[\lgap]
  \> \>   $\Next\Always p$
\end{tabbing}
\myqed\\[\lgap]
}

\opt{student}{
Exercise for the student.\\
}

Theorem (\ref{E:exAlwaysNot}) states that if it is always the case that $p$ is false then it is not the case that $p$ is always true, but not vice versa.
Suppose, for example, that $p$ continually oscillates between true and false over time.
Then the consequent of (\ref{E:exAlwaysNot}) is true, but the antecedent is false.
\begin{equation}\label{E:exAlwaysNot}
\Always\neg p \impl \neg\Always p
\end{equation}

\emph{Proof:}
\opt{doc}{
\begin{tabbing}
\hspace{\mymathindent} \= $= \;$ \= \kill
  \> \>   $\Always\neg p$\\[\lgap]
  \> $\impl$  \>  \Hint{(\ref{E:impAlwaysE}) Strengthening of $\Always$}\\[\lgap]
  \> \>   $\Event\neg p$\\[\lgap]
  \> $=$  \>  \Hint{(\ref{E:dualAlways}) Dual of $\Always$}\\[\lgap]
  \> \>   $\neg\Always p$
\end{tabbing}
\myqed\\[\lgap]
}

\opt{student}{
Exercise for the student.\\
}

Theorem (\ref{E:excludedMid}) is the linear temporal version of the excluded middle axiom of propositional logic.
\begin{equation}\label{E:excludedMid}
\textbf{Excluded Middle:}\quad \Event p \lor \Always\neg p
\end{equation}

\emph{Proof:}
\begin{tabbing}
\hspace{\mymathindent} \= $= \;$ \= \kill
  \> \>   $\Event p \lor \Always\neg p$\\[\lgap]
  \> $=$  \>  \Hint{(\ref{E:dualEvent}) Dual of $\Event$}\\[\lgap]
  \> \>   $\Event p \lor \neg\Event p$
\end{tabbing}
which is (3.28) Excluded middle, $p\lor\neg p$ with $p := \Event p$. \myqed\\[\lgap]

Theorem (\ref{E:distAlwaysAnd}) shows that $\Always$, a universal operator, distributes over conjunction.
Because disjunction is existential, Theorem (\ref{E:distAlwaysOr}) shows that $\Always$ distributes over it in only one
direction.
Theorems (\ref{E:distAlwaysEquiv}) and (\ref{E:distAlwaysImp}) show how $\Always$ distributes over the other two
propositional binary operators.
\begin{equation}\label{E:distAlwaysAnd}
\textbf{Distributivity of $\Always$ over $\land$:}\quad \Always (p \land q) \equiv \Always p \land \Always q
\end{equation}

\emph{Proof:}
\opt{doc}{
\begin{tabbing}
\hspace{\mymathindent} \= $= \;$ \= \kill
  \> \>   $\Always (p \land q)$\\[\lgap]
  \> $=$  \>  \Hint{(\ref{E:defAlways}) Definition of $\Always$}\\[\lgap]
  \> \>   $\neg\Event\neg (p \land q)$\\[\lgap]
  \> $=$  \>  \Hint{(3.47a) De Morgan $\neg (p \land q) \equiv \neg p \lor \neg q$}\\[\lgap]
  \> \>   $\neg\Event (\neg p \lor \neg q)$\\[\lgap]
  \> $=$  \>  \Hint{(\ref{E:distEventOr}) Distributivity of $\Event$ over $\lor$}\\[\lgap]
  \> \>   $\neg (\Event\neg p \lor \Event\neg q)$\\[\lgap]
  \> $=$  \>  \Hint{(3.47b) De Morgan $\neg (p \lor q) \equiv \neg p \land \neg q$}\\[\lgap]
  \> \>   $\neg\Event\neg p \land \neg\Event\neg q$\\[\lgap]
  \> $=$  \>  \Hint{(\ref{E:defAlways}) Definition of $\Always$, twice}\\[\lgap]
  \> \>   $\Always p \land \Always q$
\end{tabbing}
\myqed\\[\lgap]
}

\opt{student}{
Exercise for the student.\\
}

\begin{equation}\label{E:distAlwaysOr}
\textbf{Distributivity of $\Always$ over $\lor$:}\quad (\Always p \lor \Always q) \impl \Always (p \lor q)
\end{equation}

\emph{Proof:}
\begin{tabbing}
\hspace{\mymathindent} \= $= \;$ \= \kill
  \> \>   $\Always p \lor \Always q \impl \Always(p \lor q)$\\[\lgap]
  \> $=$  \>  \Hint{(3.60) Implication, $p\impl q \equivs p\land q \equivs p$}\\[\lgap]
  \> \>   $(\Always p \lor \Always q) \land \Always(p \lor q) \equiv \Always p \lor \Always q$\\[\lgap]
  \> $=$  \>  \Hint{(3.46) Distributivity of $\land$ over $\lor$, $p\land (q\lor r)\equiv (p\land q)\lor (p\land r)$}\\[\lgap]
  \> \>   $(\Always (p \lor q) \land \Always p) \lor (\Always (p \lor q) \land \Always q) \equiv \Always p \lor \Always q$\\[\lgap]
  \> $=$  \>  \Hint{(\ref{E:distAlwaysAnd}) Distributivity of $\Always$ over $\land$}\\[\lgap]
  \> \>   $\Always(p \land (p \lor q)) \lor \Always(q \land (p \lor q)) \equiv \Always p \lor \Always q$\\[\lgap]
  \> $=$  \>  \Hint{(3.43a) Absorption, $p \land (p \lor q) \equiv p$ twice}\\[\lgap]
  \> \>   $\Always p \lor \Always q \equiv \Always p \lor \Always q$
\end{tabbing}
which is (3.5) Reflexivity of $\equiv$, $p\equiv p$. \myqed\\[\lgap]

\begin{equation}\label{E:distAlwaysEquiv}
\textbf{Distributivity of $\Always$ over $\equiv$:}\quad \Always (p \equiv q) \impl (\Always p \equiv \Always q)
\end{equation}

\emph{Proof:}
\begin{tabbing}
\hspace{\mymathindent} \= $= \;$ \= \kill
  \> \>   $\Always (p \equiv q) \impl (\Always p \equiv \Always q)$\\[\lgap]
  \> $=$  \>  \Hint{(3.62) $p\impl (q\equiv r) \equivs p\land q\equivs p\land r$}\\[\lgap]
  \> \>   $\Always (p \equiv q) \land \Always p \equiv \Always (p \equiv q) \land \Always q$\\[\lgap]
  \> $=$  \>  \Hint{(\ref{E:distAlwaysAnd}) Distributivity of $\Always$ over $\land$, twice}\\[\lgap]
  \> \>   $\Always((p \equiv q) \land p) \equiv \Always((p \equiv q) \land q)$\\[\lgap]
  \> $=$  \>  \Hint{(3.50) $p\land (q\equiv p)\equivs p\land q$ twice}\\[\lgap]
  \> \>   $\Always(p \land q) \equiv \Always (p \land q)$
\end{tabbing}
which is (3.5) Reflexivity of $\equiv$, $p\equiv p$. \myqed\\[\lgap]


\begin{equation}\label{E:distAlwaysImp}
\textbf{Distributivity of $\Always$ over $\impl$:}\quad \Always (p \impl q) \impl (\Always p \impl \Always q)
\end{equation}

\emph{Proof:}
\opt{doc}{
\begin{tabbing}
\hspace{\mymathindent} \= $= \;$ \= \kill
  \> \>   $\Always (p \impl q)$\\[\lgap]
  \> $=$  \>  \Hint{(3.60) Implication, $p\impl q \equivs p\land q \equivs p$}\\[\lgap]
  \> \>   $\Always (p \land q \equiv p)$\\[\lgap]
  \> $\impl$  \>  \Hint{(\ref{E:distAlwaysEquiv}) Distributivity of $\Always$ over $\equiv$}\\[\lgap]
  \> \>   $\Always(p \land q) \equiv \Always p$\\[\lgap]
  \> $=$  \>  \Hint{(\ref{E:distAlwaysAnd}) Distributivity of $\Always$ over $\land$}\\[\lgap]
  \> \>   $\Always p \land \Always q \equiv \Always p$\\[\lgap]
  \> $=$  \>  \Hint{(3.60) Implication, $p\impl q \equivs p\land q \equivs p$}\\[\lgap]
  \> \>   $\Always p \impl \Always q$
\end{tabbing}
\myqed\\[\lgap]
}

\opt{student}{
Exercise for the student.\\
}

The next group of four distributivity theorems show how $\Always\Event$ and $\Event\Always$
distribute over conjunction and disjunction.
Theorem (\ref{E:distAlwaysEventAnd}) shows that $\Always\Event$ distributes over conjunction only in one direction.
Similarly, Theorem (\ref{E:distEventAlwaysOr}) shows that $\Event\Always$ distributes over disjunction only in one direction.
However, Theorems (\ref{E:distAlwaysEventOr}) and (\ref{E:distEventAlwaysAnd}) show that $\Always\Event$ distributes over
disjunction and $\Event\Always$ distributes over conjunction in both directions.
\begin{equation}\label{E:distAlwaysEventAnd}
\textbf{Distributivity of $\Always\Event$ over $\land$:}\quad \Always\Event(p \land q) \impl \Always\Event p \land \Always\Event q
\end{equation}

\emph{Proof:}
\begin{tabbing}
\hspace{\mymathindent} \= $= \;$ \= \kill
  \> \>   $\Always\Event(p \land q) \impl \Always\Event p \land \Always\Event q$\\[\lgap]
  \> $=$  \>  \Hint{(3.60) Implication, $p\impl q \equivs p\land q \equivs p$}\\[\lgap]
  \> \>   $\Always\Event(p \land q) \land \Always\Event p \land \Always\Event q \equiv \Always\Event(p \land q)$\\[\lgap]
  \> $=$  \>  \Hint{(\ref{E:distAlwaysAnd}) Distributivity of $\Always$ over $\land$}\\[\lgap]
  \> \>   $\Always(\Event(p \land q) \land \Event p \land \Event q) \equiv \Always\Event(p \land q)$\\[\lgap]
  \> $=$  \>  \Hint{Lemma: $\Event(p \land q) \land \Event p \land \Event q \equiv \Event(p \land q)$}\\[\lgap]
  \> \>   $\Always\Event(p \land q) \equiv \Always\Event(p \land q)$
\end{tabbing}
which is (3.5) Reflexivity of $\equiv$, $p\equiv p$.\\[\lgap]

\emph{Proof of Lemma:}
\begin{tabbing}
\hspace{\mymathindent} \= $= \;$ \= \kill
  \> \>   $\Event(p \land q) \land \Event p \land \Event q \equiv \Event(p \land q)$\\[\lgap]
  \> $=$  \>  \Hint{(3.60) Implication, $p\impl q \equivs p\land q \equivs p$}\\[\lgap]
  \> \>   $\Event(p \land q) \impl \Event p \land \Event q$
\end{tabbing}
which is (\ref{E:distEventAnd}) Distributivity of $\Event$ over $\land$. \myqed\\[\lgap]

\begin{equation}\label{E:distEventAlwaysOr}
\textbf{Distributivity of $\Event\Always$ over $\lor$:}\quad \Event\Always p \lor \Event\Always q \impl \Event\Always (p \lor q)
\end{equation}

\emph{Proof:}
\begin{tabbing}
\hspace{\mymathindent} \= $= \;$ \= \kill
  \> \>   $\Event\Always p \lor \Event\Always q \impl \Event\Always(p \lor q)$\\[\lgap]
  \> $=$  \>  \Hint{(3.57) Definition of Implication, $p\impl q \equivs p\lor q \equivs q$}\\[\lgap]
  \> \>   $\Event\Always p \lor \Event\Always q \lor \Event\Always(p \lor q) \equiv \Event\Always(p \lor q)$\\[\lgap]
  \> $=$  \>  \Hint{(\ref{E:distEventOr}) Distributivity of $\Event$ over $\lor$}\\[\lgap]
  \> \>   $\Event(\Always p \lor \Always q \lor \Always(p \lor q)) \equiv \Event\Always(p \lor q)$\\[\lgap]
  \> $=$  \>  \Hint{Lemma: $\Always p \lor \Always q \lor \Always(p \lor q) \equiv \Always(p \lor q)$}\\[\lgap]
  \> \>   $\Event\Always(p \lor q) \equiv \Event\Always(p \lor q)$
\end{tabbing}
which is (3.5) Reflexivity of $\equiv$, $p\equiv p$.\\[\lgap]

\emph{Proof of Lemma:}
\begin{tabbing}
\hspace{\mymathindent} \= $= \;$ \= \kill
  \> \>   $\Always p \lor \Always q \lor \Always(p \lor q) \equiv \Always(p \lor q)$\\[\lgap]
  \> $=$  \>  \Hint{(3.57) Definition of Implication, $p\impl q \equivs p\lor q \equivs q$}\\[\lgap]
  \> \>   $\Always p \lor \Always q \impl \Always(p \lor q)$
\end{tabbing}
which is (\ref{E:distAlwaysOr}) Distributivity of $\Always$ over $\lor$. \myqed\\[\lgap]

\begin{equation}\label{E:distAlwaysEventOr}
\textbf{Distributivity of $\Always\Event$ over $\lor$:}\quad \Always\Event(p \lor q) \equiv \Always\Event p \lor \Always\Event q
\end{equation}

\emph{Proof:} The proof is by mutual implication (3.80), $(p\impl q) \land (q\impl p) \equivs (p\equiv q)$.
The proof in the first direction follows.
\begin{tabbing}
\hspace{\mymathindent} \= $= \;$ \= \kill
  \> \>   $\Always\Event p \lor \Always\Event q$\\[\lgap]
  \> $\impl$  \>  \Hint{(\ref{E:distAlwaysOr}) Distributivity of $\Always$ over $\lor$}\\[\lgap]
  \> \>   $\Always(\Event p \lor \Event q)$\\[\lgap]
  \> $=$  \>  \Hint{(\ref{E:distEventOr}) Distributivity of $\Event$ over $\lor$}\\[\lgap]
  \> \>   $\Always\Event(p \lor q)$\\[\lgap]
\end{tabbing}

Axiom (\ref{E:distAlwaysEventOrAx}) supplies the proof in the second direction.
\myqed\\[\lgap]

\begin{equation}\label{E:distEventAlwaysAnd}
\textbf{Distributivity of $\Event\Always$ over $\land$:}\quad \Event\Always(p \land q) \equiv \Event\Always p \land \Event\Always q
\end{equation}

\emph{Proof:} Also by mutual implication.
The proof in the first direction follows.
\begin{tabbing}
\hspace{\mymathindent} \= $= \;$ \= \kill
  \> \>   $\Event\Always (p \land q)$\\[\lgap]
  \> $=$  \>  \Hint{(\ref{E:distAlwaysAnd}) Distributivity of $\Always$ over $\land$}\\[\lgap]
  \> \>   $\Event(\Always p \land \Always q)$\\[\lgap]
  \> $\impl$  \>  \Hint{(\ref{E:distEventAnd}) Distributivity of $\Event$ over $\land$}\\[\lgap]
  \> \>   $\Event\Always p \land \Event\Always q$\\[\lgap]
\end{tabbing}

Axiom (\ref{E:distEventAlwaysAndAx}) supplies the proof in the second direction.
\myqed\\[\lgap]

Theorem (\ref{E:eventAlwaysImp}) states that $\Event\Always p$ ensures that $p$ will always eventually hold, but not vice versa.
That is, $\Always\Event p$ does not guarantee that eventually $p$ will always hold.
\begin{equation}\label{E:eventAlwaysImp}
\Event\Always p \impl \Always\Event p
\end{equation}

\emph{Proof:}
\begin{tabbing}
\hspace{\mymathindent} \= $= \;$ \= \kill
  \> \>   $\Event\Always p \impl \Always\Event p$\\[\lgap]
  \> $=$  \>  \Hint{(3.59) Implication, $p\impl q \equivs \neg p \lor q$}\\[\lgap]
  \> \>   $\neg\Event\Always p \lor \Always\Event p$\\[\lgap]
  \> $=$  \>  \Hint{(\ref{E:dualEvent}) Dual of $\Event$}\\[\lgap]
  \> \>   $\Always\neg\Always p \lor \Always\Event p$\\[\lgap]  
  \> $=$  \>  \Hint{(\ref{E:dualAlways}) Dual of $\Always$}\\[\lgap]
  \> \>   $\Always\Event\neg p \lor \Always\Event p$\\[\lgap]
  \> $=$  \>  \Hint{(\ref{E:distAlwaysEventOr}) Distributivity of $\Always\Event$ over $\lor$}\\[\lgap]
  \> \>   $\Always\Event(\neg p \lor p)$\\[\lgap]
  \> $=$  \>  \Hint{(3.28) Excluded middle, $p\lor\neg p$}\\[\lgap]
  \> \>   $\Always\Event true$\\[\lgap]
  \> $=$  \>  \Hint{(\ref{E:eventTrue}) Truth}\\[\lgap]
  \> \>   $\Always true$\\[\lgap]
  \> $=$  \>  \Hint{(\ref{E:alwaysTrue}) Truth}\\[\lgap]
  \> \>   $true$
\end{tabbing}
\myqed\\[\lgap]

The absorption theorems, (\ref{E:absEvent}) and (\ref{E:absAlways}), together with absorption theorems
(\ref{E:IdemEvent}) and (\ref{E:IdemAlways}), allow any arbitrary string of $\Event$ and $\Always$ operators
of any arbitrary length to be collapsed into one of four expressions: \;$\Event p$, \;$\Always p$,
\;$\Always\Event p$, or \;$\Event\Always p$.
\begin{equation}\label{E:absEvent}
\textbf{Absorption of $\Event$ into $\Always\Event$:}\quad \Event\Always\Event p \equiv \Always\Event p
\end{equation}

\emph{Proof:} Proof by Mutual Implication (3.80) $(p\impl q) \land (q\impl p) \equivs (p\equiv q)$.
The proof in the first direction follows.
\begin{tabbing}
\hspace{\mymathindent} \= $= \;$ \= \kill
  \> \>   $\Event\Always\Event p$\\[\lgap]
  \> $\impl$  \>  \Hint{(\ref{E:eventAlwaysImp})}\\[\lgap]
  \> \>   $\Always\Event\Event p$\\[\lgap]
  \> $=$  \>  \Hint{(\ref{E:IdemEvent}) Absorption of $\Event$}\\[\lgap]
  \> \>   $\Always\Event p$
\end{tabbing}
The proof in the second direction follows.
\begin{tabbing}
\hspace{\mymathindent} \= $= \;$ \= \kill
  \> \>   $\Always\Event p$\\[\lgap]
  \> $\impl$  \>  \Hint{(\ref{E:impEvent}) Weakening of $\Event$}\\[\lgap]
  \> \>   $\Event\Always\Event p$
\end{tabbing}
\myqed\\[\lgap]

\begin{equation}\label{E:absAlways}
\textbf{Absorption of $\Always$ into $\Event\Always$:}\quad \Always\Event\Always p \equiv \Event\Always p
\end{equation}

\emph{Proof:} Also by mutual implication.
The proof in the first direction follows.
\opt{doc}{
\begin{tabbing}
\hspace{\mymathindent} \= $= \;$ \= \kill
  \> \>   $\Always\Event\Always p$\\[\lgap]
  \> $\impl$  \>  \Hint{(\ref{E:impAlways}) Strengthening of $\Always$}\\[\lgap]
  \> \>   $\Event\Always p$
\end{tabbing}

The proof in the second direction follows.
\begin{tabbing}
\hspace{\mymathindent} \= $= \;$ \= \kill
  \> \>   $\Event\Always p$\\[\lgap]
  \> $=$  \>  \Hint{(\ref{E:IdemAlways}) Absorption of $\Always$}\\[\lgap]
  \> \>   $\Event\Always\Always p$\\[\lgap]
  \> $\impl$  \>  \Hint{(\ref{E:eventAlwaysImp})}\\[\lgap]
  \> \>   $\Always\Event\Always p$
\end{tabbing}
\myqed\\[\lgap]
}

\opt{student}{
Exercise for the student.\\
}

The monotonicity theorems (\ref{E:alwaysImpNexts}) and (\ref{E:alwaysImpEvents}) have $\Always (p \impl q)$
as the antecedent.
Theorem (\ref{E:distAlwaysImp}), distributivity of $\Always$ over $\impl$, has the same antecedent and hence can be considered a
monotonicity theorem as well.
Theorems (\ref{E:distAlwaysImp}), (\ref{E:alwaysImpNexts}), and (\ref{E:alwaysImpEvents}) show that all unary temporal
operators are monotonic.
\begin{equation}\label{E:alwaysImpNexts}
\textbf{Monotonicity of $\Next$:}\quad \Always (p \impl q) \impl (\Next p \impl \Next q)
\end{equation}

\emph{Proof:}
\opt{doc}{
\begin{tabbing}
\hspace{\mymathindent} \= $= \;$ \= \kill
  \> \>   $\Always (p \impl q)$\\[\lgap]
  \> $\impl$  \>  \Hint{(\ref{E:impAlwaysN}) Strengthening}\\[\lgap]
  \> \>   $\Next (p \impl q)$\\[\lgap]
  \> $=$  \>  \Hint{(\ref{E:distNextImp}) Distributivity of $\Next$ over $\impl$}\\[\lgap]
  \> \>   $\Next p \impl \Next q$
\end{tabbing}
\myqed\\[\lgap]
}

\opt{student}{
Exercise for the student.\\
}

\begin{equation}\label{E:alwaysImpEvents}
\textbf{Monotonicity of $\Event$:}\quad \Always (p \impl q) \impl (\Event p \impl \Event q)
\end{equation}

\emph{Proof:}
\opt{doc}{
\begin{tabbing}
\hspace{\mymathindent} \= $= \;$ \= \kill
  \> \>   $\Always (p \impl q) \impl (\Event p \impl \Event q)$\\[\lgap]
  \> $=$  \>  \Hint{(3.59) Implication, $p\impl q \equivs \neg p \lor q$ three times}\\[\lgap]
  \> \>   $\neg\Always (\neg p \lor q) \lor \neg\Event p \lor \Event q$\\[\lgap]
  \> $=$  \>  \Hint{(\ref{E:dualAlways}) Dual of $\Always$}\\[\lgap]
  \> \>   $\Event\neg (\neg p \lor q) \lor \neg\Event p \lor \Event q$\\[\lgap]  
  \> $=$  \>  \Hint{(3.47b) De Morgan $\neg (p \lor q) \equiv \neg p \land \neg q$, and (3.12) Double negation, $\neg\neg p\equiv p$}\\[\lgap]
  \> \>   $\Event(p \land \neg q) \lor \neg\Event p \lor \Event q$\\[\lgap]
  \> $=$  \>  \Hint{(\ref{E:distEventOr}) Distributivity of $\Event$ over $\lor$}\\[\lgap]
  \> \>   $\Event((p \land \neg q) \lor q) \lor \neg\Event p$\\[\lgap]
  \> $=$  \>  \Hint{(3.44b) Absorption, $p \lor (\neg p \land q) \equiv p \lor q$}\\[\lgap]
  \> \>   $\Event(p \lor q) \lor \neg\Event p$\\[\lgap]
  \> $=$  \>  \Hint{(\ref{E:distEventOr}) Distributivity of $\Event$ over $\lor$}\\[\lgap]
  \> \>   $\Event p \lor \Event q \lor \neg\Event p$\\[\lgap]
  \> $=$  \>  \Hint{(3.28) Excluded middle, $p\lor\neg p$ with $p := \Event p$}\\[\lgap]
  \> \>   $\Event q \lor true$\\[\lgap]
  \> $=$  \>  \Hint{(3.29) Zero of $\lor$, $p\lor true\equiv true$}\\[\lgap]
  \> \>   $true$
\end{tabbing}
\myqed\\[\lgap]
}

\opt{student}{
Exercise for the student.\\
}

This section concludes with two theorems that are common to all temporal logic systems.
\begin{equation}\label{E:eventImpAlways}
\Event (p \impl q) \equiv (\Always p \impl \Event q)
\end{equation}

\emph{Proof:}
\opt{doc}{
\begin{tabbing}
\hspace{\mymathindent} \= $= \;$ \= \kill
  \> \>   $\Event(p \impl q)$\\[\lgap]
  \> $=$  \>  \Hint{(3.59) Implication $p\impl q \equivs \neg p \lor q$}\\[\lgap]
  \> \>   $\Event(\neg p \lor q)$\\[\lgap]
  \> $=$  \>  \Hint{(\ref{E:distEventOr}) Distributivity of $\Event$ over $\lor$}\\[\lgap]
  \> \>   $\Event\neg p \lor \Event q$\\[\lgap]
  \> $=$  \>  \Hint{(\ref{E:dualAlways}) Dual of $\Always$}\\[\lgap]
  \> \>   $\neg\Always p \lor \Event q$\\[\lgap]
  \> $=$  \>  \Hint{(3.59) Implication $p\impl q \equivs \neg p \lor q$}\\[\lgap]
  \> \>   $\Always p \impl \Event q$
\end{tabbing}
\myqed\\[\lgap]
}

\opt{student}{
Exercise for the student.\\
}
\begin{equation}\label{E:alwaysAndEvent}
\Always p \land \Event q \impl \Event (p \land q)
\end{equation}

\emph{Proof:}
\opt{doc}{
\begin{tabbing}
\hspace{\mymathindent} \= $= \;$ \= \kill
  \> \>   $\Always p \land \Event q \impl \Event (p \land q)$\\[\lgap]
  \> $=$  \>  \Hint{(3.59) Implication $p\impl q \equivs \neg p \lor q$}\\[\lgap]
  \> \>   $\neg(\Always p \land \Event q) \lor \Event(p \land q)$\\[\lgap]
  \> $=$  \>  \Hint{(3.47a) De Morgan $\neg (p \land q) \equiv \neg p \lor \neg q$}\\[\lgap]
  \> \>   $\neg\Always p \lor \neg\Event q \lor \Event(p \land q)$\\[\lgap]
  \> $=$  \>  \Hint{(\ref{E:dualAlways}) Dual of $\Always$}\\[\lgap]
  \> \>   $\Event\neg p \lor \neg\Event q \lor \Event(p \land q)$\\[\lgap]
  \> $=$  \>  \Hint{(\ref{E:distEventOr}) Distributivity of $\Event$ over $\lor$}\\[\lgap]
  \> \>   $\Event(\neg p \lor (p \land q)) \lor \neg\Event q$\\[\lgap]
  \> $=$  \>  \Hint{(3.44b) Absorption, $p \lor (\neg p \land q) \equiv p \lor q$}\\[\lgap]
  \> \>   $\Event(\neg p \lor q) \lor \neg\Event q$\\[\lgap]
  \> $=$  \>  \Hint{(\ref{E:distEventOr}) Distributivity of $\Event$ over $\lor$}\\[\lgap]
  \> \>   $\Event\neg p \lor \Event q \lor \neg\Event q$\\[\lgap]
  \> $=$  \>  \Hint{(3.28) Excluded middle, $p\lor\neg p$}\\[\lgap]
  \> \>   $\Event\neg p \lor true$\\[\lgap]
  \> $=$  \>  \Hint{(3.29) Zero of $\lor$, $p\lor true\equiv true$}\\[\lgap]
  \> \>   $true$
\end{tabbing}
\myqed\\[\lgap]
}

\opt{student}{
Exercise for the student.\\
}

\subsection{Wait}

\begin{equation}\label{E:defWait}
\textbf{Definition of $\Wait$:}\quad p \Wait q \equiv (p \Until q) \lor \Always p
\end{equation}


\begin{equation}\label{E:expansionWait}
p \Wait q \equiv q \lor (p \land \Next (p \Wait q))
\end{equation}

\emph{Proof:}
\begin{tabbing}
\hspace{\mymathindent} \= $= \;$ \= \kill
\> \> $q \lor (p \land \Next (p \Wait q))$\\[\lgap]
\> $=$ \> \Hint{(\ref{E:defWait}) Definition of $\Wait$} \\[\lgap]
\> \> $q \lor (p \land \Next ((p \Until q) \lor \Always p))$\\[\lgap]
\> $=$ \> \Hint{(\ref{E:distNextOr}) Distributivity of $\Next$ over $\lor$} \\[\lgap]
\> \> $q \lor (p \land (\Next (p \Until q) \lor \Next\Always p))$\\[\lgap]
\> $=$ \> \Hint{(3.46) Distributivity of $\land$ over $\lor$, $p\land (q\lor r)\equiv (p\land q)\lor (p\land r)$}\\[\lgap]
\> \> $q \lor (p \land \Next(p \Until q)) \lor (p \land \Next\Always p)$\\[\lgap]
\> $=$ \> \Hint{(\ref{E:expansionAlways}) Expansion of $\Always$}\\[\lgap]
\> \> $q \lor (p \land \Next(p \Until q)) \lor \Always p$\\[\lgap]
\> $=$ \> \Hint{(\ref{E:expansionUntil}) Expansion of $\Until$}\\[\lgap]
\> \> $(p \Until q) \lor \Always p$\\[\lgap]
\> $=$ \> \Hint{(\ref{E:defWait}) Definition of $\Wait$} \\[\lgap]
\> \> $p \Wait q$\\[\lgap]
\end{tabbing}
\myqed\\[\lgap]


\begin{equation}\label{E:alwaysImpWait}
\Always p \Rightarrow p \Wait q
\end{equation}

\emph{Proof:}
\begin{tabbing}
\hspace{\mymathindent} \= $= \;$ \= \kill
\> \> $\Always p$\\[\lgap]
\> $=$ \> \Hint{(3.76a) Weakening, $p\impl p\lor q$} \\[\lgap]
\> \> $\Always p \lor (p \Until q)$\\[\lgap]
\> $=$ \> \Hint{(\ref{E:defWait}) Definition of $\Wait$} \\[\lgap]
\> \> $p \Wait q$\\[\lgap]
\end{tabbing}
\myqed\\[\lgap]


\begin{equation}\label{E:alwaysAsWait}
\Always p \equiv p \Wait false
\end{equation}

\emph{Proof:}
\begin{tabbing}
\hspace{\mymathindent} \= $= \;$ \= \kill
\> \> $p \Wait false$\\[\lgap]
\> $=$ \> \Hint{(\ref{E:defWait}) Definition of $\Wait$} \\[\lgap]
\> \> $(p \Until false) \lor \Always p$\\[\lgap]
\> $=$ \> \Hint{(\ref{E:untilFalse})} \\[\lgap]
\> \> $false \lor \Always p$\\[\lgap]
\> $=$ \> \Hint{(3.30) Identity of $\lor$, $p\lor false\equiv p$} \\[\lgap]
\> \> $\Always p$\\[\lgap]
\end{tabbing}
\myqed\\[\lgap]


\begin{equation}\label{E:waitInsertion}
q \impl p \Wait q
\end{equation}

\emph{Proof:}
\begin{tabbing}
\hspace{\mymathindent} \= $= \;$ \= \kill
\> \> $q$\\[\lgap]
\> $\impl$ \> \Hint{(3.76a) Weakening, $p\impl p\lor q$} \\[\lgap]
\> \> $q \lor (p \land \Next(p \Wait q))$\\[\lgap]
\> $=$ \> \Hint{(\ref{E:expansionWait})} \\[\lgap]
\> \> $p \Wait q$\\[\lgap]
\end{tabbing}
\myqed\\[\lgap]


\begin{equation}\label{E:waitToOr}
p \Wait q \impl p \lor q
\end{equation}

\emph{Proof:}
\begin{tabbing}
\hspace{\mymathindent} \= $= \;$ \= \kill
\> \> $p \Wait q$\\[\lgap]
\> $=$ \> \Hint{(\ref{E:expansionWait})} \\[\lgap]
\> \> $q \lor (p \land \Next(p \Wait q))$\\[\lgap]
\> $\impl$ \> \Hint{(3.76d) $p\lor (q\land r) \impl p\lor q$} \\[\lgap]
\> \> $p \lor q$\\[\lgap]
\end{tabbing}
\myqed\\[\lgap]


\begin{equation}\label{E:eventerWait}
\Event p \equiv \neg(\neg p \Wait false)
\end{equation}

\emph{Proof:}
\begin{tabbing}
\hspace{\mymathindent} \= $= \;$ \= \kill
\> \> $\neg(\neg p \Wait false)$\\[\lgap]
\> $=$ \> \Hint{(\ref{E:alwaysAsWait})} \\[\lgap]
\> \> $\neg\Always\neg p$\\[\lgap]
\> $=$ \> \Hint{(\ref{E:eventAsAlways})} \\[\lgap]
\> \> $\Event p$\\[\lgap]
\end{tabbing}
\myqed\\[\lgap]


\begin{equation}\label{E:waitEntailment}
p \Wait q \impl \Always p \lor \Event q
\end{equation}

\emph{Proof:}
\begin{tabbing}
\hspace{\mymathindent} \= $= \;$ \= \kill
\> \> $p \Wait q$\\[\lgap]
\> $=$ \> \Hint{(\ref{E:defWait})} \\[\lgap]
\> \> $(p \Until q) \lor \Always p$\\[\lgap]
\> $\impl$ \> \Hint{(\ref{E:eventuality}) Eventuality and (4.2) Monotonicity of $\lor$, $(p\impl q) \impl (p\lor r \impl q\lor r)$} \\[\lgap]
\> \> $\Event q \lor \Always p$\\[\lgap]
\end{tabbing}
\myqed\\[\lgap]


\begin{equation}\label{E:waitEntailAlways}
(p \Wait q) \land \Always\neg q \impl \Always p
\end{equation}

\emph{Proof:}
\begin{tabbing}
\hspace{\mymathindent} \= $= \;$ \= \kill
\> \> $(p \Wait q) \land \Always\neg q$\\[\lgap]
\> $\impl$ \> \Hint{(\ref{E:waitEntailment})} \\[\lgap]
\> \> $(\Always p \lor \Event q) \land \Always\neg q$\\[\lgap]
\> $=$ \> \Hint{(3.46) Distributivity of $\land$ over $\lor$, $p\land (q\lor r)\equiv (p\land q)\lor (p\land r)$} \\[\lgap]
\> \> $(\Always p \land \Always\neg q) \lor (\Event q \land \Always\neg q)$\\[\lgap]
\> $=$ \> \Hint{(\ref{E:dualEvent})} \\[\lgap]
\> \> $(\Always p \land \Always\neg q) \lor (\Event q \land \neg\Event q)$\\[\lgap]
\> $=$ \> \Hint{(3.42) Contradiction, $p \land \neg p \equiv false$} \\[\lgap]
\> \> $(\Always p \land \Always\neg q) \lor false$\\[\lgap]
\> $=$ \> \Hint{(3.30) Identity of $\lor$, $p\lor false\equiv p$} \\[\lgap]
\> \> $\Always p \land \Always\neg q$\\[\lgap]
\> $\impl$ \> \Hint{(3.76b) Strengthening, $p\land q \impl p$} \\[\lgap]
\> \> $\Always p$\\[\lgap]
\end{tabbing}
\myqed\\[\lgap]


\begin{equation}\label{E:waitEntailEvent}
(p \Wait q) \land \neg\Always p \impl \Event q
\end{equation}

\emph{Proof:}
\begin{tabbing}
\hspace{\mymathindent} \= $= \;$ \= \kill
\> \> $(p \Wait q) \land \neg\Always p$\\[\lgap]
\> $\impl$ \> \Hint{(\ref{E:waitEntailment})} \\[\lgap]
\> \> $(\Always p \lor \Event q) \land \neg\Always p$\\[\lgap]
\> $=$ \> \Hint{(3.46) Distributivity of $\land$ over $\lor$, $p\land (q\lor r)\equiv (p\land q)\lor (p\land r)$} \\[\lgap]
\> \> $(\Always p \land \neg\Always p) \lor (\Event q \land \neg\Always p)$\\[\lgap]
\> $=$ \> \Hint{(3.42) Contradiction, $p \land \neg p \equiv false$} \\[\lgap]
\> \> $false \lor (\Event q \land \neg\Always p)$\\[\lgap]
\> $=$ \> \Hint{(3.30) Identity of $\lor$, $p\lor false\equiv p$} \\[\lgap]
\> \> $\Event q \land \neg\Always p$\\[\lgap]
\> $\impl$ \> \Hint{(3.76b) Strengthening, $p\land q \impl p$} \\[\lgap]
\> \> $\Event q$\\[\lgap]
\end{tabbing}
\myqed\\[\lgap]


[\textit{More theorems to be written.}]

\section{Conclusion}

This paper presents an axiomatic deductive system of temporal logic whose theorems are proved with the equational
logic $\mathcal{E}$ of Gries and Schneider \cite{LADM}.
It takes unary operator next $\Next$ and binary operator until $\Until$ as primitives and defines
eventually $\Event$, always $\Always$, and wait $\Wait$ in terms of them.\\

Dijkstra and Scholten \cite{DandS}, and Feijen \cite{Feij} originally developed $\mathcal{E}$ as a logic system to prove
program correctness based on an equational style.
Gries and Schneider extend that system to a theory of sets, a theory of sequences,
relations and functions, a theory of integers, recurrence relations, modern algebra, and a theory of graphs.
Similarly, this system extends $\mathcal{E}$ to a theory of temporal logic.\\

The equational system $\mathcal{E}$ has several advantages over other logic systems.
The primary advantage is that the equational system has only one proof rule, Leibniz.
Consequently, proofs of theorems are easier to understand and more intuitive to those schooled in that system. 
We believe the advantages of $\mathcal{E}$ over other logic systems is so substantial that it should be the
tool of choice for computer science theory.
We hope that this extension of $\mathcal{E}$ to linear temporal logic will not only be of use in the
temporal logic community, but will serve as an example to promote $\mathcal{E}$ in the broader computer science community.

\bibliographystyle{plain}
\bibliography{Vega-Paper}
\end{document}
