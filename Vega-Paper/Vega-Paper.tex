% David Vega and Stan Warford
% Pepperdine University
% File: Vega-Paper

\documentclass[fleqn, leqno]{article}

\usepackage{times}
\usepackage{amsmath, amsthm, amssymb,latexsym}
\usepackage{paralist}
\usepackage{ellipsis}
%\usepackage{array}

\newcommand{\lgap}{2pt}                             % Line gap
\newcommand{\llgap}{6pt}                            % Larger line gap
\newcommand{\lllgap}{32pt}                          % Largest line gap for students to write in
\newcommand{\mymathindent}{24pt}                      % Indentation for math tabbing
\newcommand{\equivs}{\ensuremath{\;\equiv\;}}       % Equivales with space
\newcommand{\equivss}{\ensuremath{\;\;\equiv\;\;}}  % Equivales with double space
\newcommand{\nequiv}{\ensuremath{\not\equiv}}       % Inequivalent
\newcommand{\impl}{\ensuremath{\Rightarrow}}        % Implies
\newcommand{\nimpl}{\ensuremath{\not\Rightarrow}}   % Does not imply
\newcommand{\foll}{\ensuremath{\Leftarrow}}         % Follows from
\newcommand{\nfoll}{\ensuremath{\not\Leftarrow}}    % Does not follow from


% Macros for Temporal Operators
\newcommand{\until}{\;\mathcal{U}\;}
\newcommand{\wait}{\;\mathcal{W}\;}
\newcommand{\next}{\bigcirc}
\newcommand{\event}{\Diamond}
\newcommand{\always}{\Box}

\newcommand{\myqed}{\hfill\rule[-.23ex]{1.2ex}{2.0ex}}

% Thanks to David Gries for sharing the following macros
% Macros for quantifications.
\newcommand{\thedr}{\rule[-.25ex]{.32mm}{1.75ex}}   % Symbol that separates dummy from range in quantification
\newcommand{\dr}{\;\,\thedr\,\;}                    % Symbol that separates dummy from range, with spacing
\newcommand{\rb}{:}                                 % Symbol that separates range from body in quantification
\newcommand{\drrb}{\;\thedr\,{:}\;}                 % Symbol that separates dummy from body when range is missing
\newcommand{\all}{\forall}                          % Universal quantification
\newcommand{\ext}{\exists}                          % Existential quantification

% Macros for proof hints
\newcommand{\Gll} {\langle}                         % Open hint
\newcommand{\Ggg} {\rangle}                         % Close hint
\newlength{\Glllength}                              % Length of open hint symbol
\settowidth{\Glllength}{$.\Gll$}
\newcommand{\Hint}[1]     {\ \ \ $\Gll              \mbox{#1} \Ggg$ }   % Single line hint
\newcommand{\Hintfirst}[1]{\ \ \ $\Gll              \mbox{#1}$ }        % First line of multiline hint
\newcommand{\Hintmid}[1]  {\ \ $\hspace{\Glllength} \mbox{#1}$ }        % Middle line of multiline hint
\newcommand{\Hintlast}[1] {\ \ $\hspace{\Glllength} \mbox{#1} \Ggg$ }   % Last line of multiline hint

% Single and double quotes
\newcommand{\Lq}{\mbox{`}}
\newcommand{\Rq}{\mbox{'}}
\newcommand{\Lqq}{\mbox{``}}
\newcommand{\Rqq}{\mbox{''}}

\oddsidemargin  0.0in
\evensidemargin 0.0in
\textwidth      6.0in
\headheight     0.0in
\topmargin      0.0in
\textheight=8.5in
\parindent=0in
\pagestyle{plain}

\title{An Equational Deductive System\\for Linear Temporal Logic}
\author{David Vega\thanks{Research supported by Tooma Undergraduate Research Fellowship Program, Summer 2009}\\
   Computer Science Department\\
   Pepperdine University\\
   Malibu, CA 90265
   \and
   J. Stanley Warford\\
   Computer Science Department\\
   Pepperdine University\\
   Malibu, CA 90265}
\date{} % Required for no date to appear in heading

\begin{document}
\maketitle
\begin{abstract}
This is the abstract. This is the abstract. This is the abstract. This is the abstract. This is the abstract.
This is the abstract. This is the abstract. This is the abstract. This is the abstract. This is the abstract. 
This is the abstract. This is the abstract. This is the abstract. This is the abstract. This is the abstract. 
This is the abstract. This is the abstract. This is the abstract. This is the abstract. This is the abstract. 
This is the abstract. This is the abstract. This is the abstract. This is the abstract. This is the abstract. 
\end{abstract}

\thispagestyle{plain}

\section{Introduction}

Propositional calculus is a formal system of logic based on the unary operator negation $\lnot$,
the binary operators conjunction $\land$, disjunction $\lor$, implies $\impl$ (also written $\rightarrow$),
and equivalence $\equiv$ (also written $\leftrightarrow$),
variables (lowercase letters $p$, $q$, \dots), and the constants $true$ and $false$.
Hilbert-style logic systems, $\mathcal{H}$, are the deductive logic systems traditionally used in mathematics.
In the late 1980's, Dijkstra and Scholten \cite{DandS}, and Feijen \cite{Feij} developed a method of proving
program correctness with a new logic based on an equational style.
This equational deductive system, $\mathcal{E}$, has been the basis of textbooks by Kaldewaij \cite{Kald},
Cohen \cite{Cohen}, and Gries \cite{LADM}.\\

The equational deductive system of proofs is slow to be adopted by the computer science community.
The problem is two-fold.
First, a fair amount of technical detail must be mastered,
and many computer science educators and practitioners do not have the requisite
knowledge of formal logic systems, much less the equational deductive system.
Scores of textbooks for discrete mathematics for computer science could be cited that give only a cursory introduction to
formal logic. Most of these texts, such as the classic one by Rosen \cite{Rosen} are beginning to move to a more
formal treatment of logic appropriate for computer science.\\

Second, even when formal logic is taught at a depth necessary to apply it to program proofs, the older Hilbert style
still dominates.
The previously-cited texts \cite{Cohen, LADM, Kald} are among the few that rely on the equational deductive system. 
More typical is Ben-Ari's book \cite{Ben}, which is based enitrely on natural deduction systems.\\

Linear temporal logic describes how the truth values of propositions change over time.
It extends the propositional operators with the unary operators next $\next$, eventually $\event$, and always $\always$,
and the binary operators until $\until$ and wait $\wait$.
Propositional calculus applies to program correctness with the formulation of the Hoare triple to establish invariants
in programs that terminate.
Temporal logic applies to program correctness with concurrent processing to establish safety and liveness properties
in programs that possibly do not terminate.\\

As is the case for propositional and predicate calculus, most treatments of linear temporal logic use $\mathcal{H}$
instead of $\mathcal{E}$. Typical are Ben-Ari \cite{Ben2}, Emerson \cite{Emer}, and Manna and Pnueli \cite{Manna}.
Schneider \cite{Schn}, who develops an equational deductive system for temporal logic appears to be the primary exception.
The development of linear temporal logic in all these works is motivated by its use to prove correctness of concurrent programs.
The presentation typically consists of lists of valid temporal formulas, with little indication of which formulas are required
as axioms and which are theorems that can be proved using a deductive system.\\

This paper presents an equational deductive system for linear temporal logic.
It differs from previous developments of such systems in four respects.
First, it presents a numbered list of axioms and theorems to indicate which formulas are assumed, which formulas are
derived, and for those that are derived, which previous formulas they depend on.
Second, it gives a proof of every theorem.
Third, the proofs are given in the equational style $\mathcal{E}$ instead of $\mathcal{H}$.
Fourth, it presents several new and interesting linear temporal theorems.\\

Section 2 describes the deductive axioms and the proof rules for $\mathcal{E}$.
It also defines the syntax and semantics of linear temporal logic.
Section 3 presents the equational deductive system for linear temporal logic.\\

\section{Background}

\subsection{Equational Deductive Systems}

This section closely follows Gries and Schneider \cite{LADM}.\\

The definition of an expression has four parts:
\begin{enumerate}[$\bullet$]
\item A constant or variable is an expression.
\item If $E$ is an expression, then $(E)$ is an expression.
\item If $\circ$ is a unary prefix operator and $E$ is an expression, then $\circ E$ is an expression with operand $E$.
\item If $\star$ is a binary infix operator and $D$ and $E$ are expressions, then $D \star E$ is an expression with operands $D$
and $E$.
\end{enumerate}

By convention, upper-case letters ({\itshape e.g.\/} $X$, $Y$, \dots) represent expressions,
and lower-case letters ({\itshape e.g.\/} $x$, $y$, \dots) represent variables.
In the propositional calculus, the constants are {\itshape true\/} and {\itshape false\/}.\\

Here is the table of precedences.

%\begin{tabular}{lr}
%First Left & First Right\\
%Second left & Second right\\
%end{tabular}

The equational deductive system relies on the three deductive axioms
\[
\textbf{Reflexivity:}\quad x=x
\]
\[
\textbf{Symmetry:}\quad (x=y) = (y=x)
\]
\[
\textbf{Transitivity:}\quad \frac{X=Y, \quad Y=Z}{X=Z}
\]
and the proof rule
\[
\textbf{Leibniz:}\quad \frac{X=Y}{E[z:=X]=E[z:=Y]}
\]

where the square bracket indicates textual substitution of expression $X$ for variable $z$ and substitution
of expression $Y$ for variable $z$.
Roughly speaking, Leibniz allows for the substitution of equals for equals in a proof step.
The general form of a proof step is:

\begin{tabbing}
\hspace{\mymathindent} \= $= \;$ \=  \kill
  \> \>   $E[z:=X]$\\[\lgap]
  \> $=$  \>  \Hint{$X=Y$} \\[\lgap]
  \> \>   $E[z:=Y]$
\end{tabbing}

\subsection{Temporal Logic}

Temporal logic is the logic of parallel computation, that is, computations that happen concurrently
with the multicore processing chips that are becoming more popular in personal computers.
Programs are inherently more difficult to test and to prove correct when they employ concurrency.\\


\section{The Equational Temporal System}

Ben-Ari \cite{Ben} gives a deductive system $\mathcal{L}$ for temporal logic, but it is a
natural deductive system in the Hilbert style of $\mathcal{G}$ and $\mathcal{H}$. This project is to develop a complete system of
temporal logic but using the equational deductive system based on the four proof rules described
in the first section. The system will consist of a small number of axioms and a larger number of
theorems. Because the proof rules in the equational deductive system are different from those
in $\mathcal{L}$, the proofs will be different as well.\\

Note: Explain/use: The domain of discourse\\

Note: primitives!  Lets use $\next$ and $\mathcal{U}$\\

Note: lets look at until in mathcal: $\mathcal{U}$\\



\subsection{Next}

Note: below is our auto-label system for axioms and theorems.\\

\begin{equation}\label{E:selfDual}
\textbf{Axiom, Self-dual:}\quad \next\lnot p \equiv \lnot\next p
\end{equation}

\begin{equation}\label{E:distNextImp}
\textbf{Axiom, Distributivity of $\next$ over $\Rightarrow$:}\quad \next (p \Rightarrow q) \equiv \next p \Rightarrow \next q
\end{equation}

\begin{equation}\label{E:nextTruth}
\textbf{Axiom, Truth:}\quad \next true \equiv true
\end{equation}

\begin{equation}\label{E:linearity}
\textbf{Linearity:}\quad \next p \equiv \lnot\next\lnot p
\end{equation}

\emph{Proof:}
\begin{tabbing}
\hspace{\mymathindent} \= $= \;$ \= \kill
  \> \>   $\next p \equiv \lnot\next\lnot p$\\[\lgap]
  \> $=$  \>  \Hint{(3.11) with $p,q := \next\lnot p, \next p$} \\[\lgap]
  \> \>   $\lnot\next p \equiv \next\lnot p$
\end{tabbing}
which is (\ref{E:selfDual}). \myqed\\[\lgap]


\begin{equation}\label{E:nextFalse}
\textbf{Falsehood:}\quad \next false \equiv false
\end{equation}

\emph{Proof:}
\begin{tabbing}
\hspace{\mymathindent} \= $= \;$ \= \kill
  \> \>   $\next false \equiv false$\\[\lgap]
  \> $=$  \>  \Hint{(3.8) Definition of $false$} \\[\lgap]
  \> \>   $\next\lnot true \equiv \lnot true$\\[\lgap]
  \> $=$  \>  \Hint{(3.11) with $p,q := true, \next\lnot true$}\\[\lgap]
  \> \>   $\lnot\next\lnot true \equiv true$\\[\lgap]
\end{tabbing}
which is (\ref{E:linearity}) with $p := true$. \myqed\\[\lgap]


\begin{equation}\label{E:distNextOr}
\textbf{Distributivity of $\next$ over $\lor$:}\quad \next (p \lor q) \equiv \next p \lor \next q
\end{equation}


\emph{Proof:}
\begin{tabbing}
\hspace{\mymathindent} \= $= \;$ \= \kill
	\> \>   $\next$(p $\lor$ q)\\[\lgap]
	\> $=$  \>  \Hint{(3.59) Implication}\\[\lgap]
	\> \>   $\next$($\lnot$p $\Rightarrow$ q)\\[\lgap]
	\> $=$  \>  \Hint{(\ref{E:distNextImp}) Distributivity of $\next$ over $\Rightarrow$}\\[\lgap]
	\> \>   $\next\lnot$p $\Rightarrow$ $\next$q\\[\lgap]
	\> $=$  \>  \Hint{(3.59) Implication}\\[\lgap]
	\> \>   $\lnot\next\lnot$p $\lor$ $\next$q\\[\lgap]
	\> $=$  \>  \Hint{(\ref{E:linearity}) Linearity}\\[\lgap]
	\> \>   $\next$p $\lor$ $\next$q
\end{tabbing}
\myqed\\[\lgap]

\begin{equation}\label{E:distNextAnd}
\textbf{Distributivity of $\next$ over $\land$:}\quad \next (p \land q) \equiv \next p \land \next q
\end{equation}


\emph{Proof:}
\begin{tabbing}
\hspace{\mymathindent} \= $= \;$ \= \kill
  \> \>   $\next (p \land q)$\\[\lgap]
  \> $=$  \>  \Hint{(3.12) Double Negation, twice}\\[\lgap]
  \> \>   $\next (\lnot\lnot p \land \lnot\lnot q)$\\[\lgap]
  \> $=$  \>  \Hint{(3.47) DeMorgan's Law}\\[\lgap]
  \> \>   $\next\lnot(\lnot p \lor \lnot q)$\\[\lgap]
  \> $=$  \>  \Hint{(\ref{E:selfDual}) with $p:= (\lnot p \lor \lnot q$)}\\[\lgap]
  \> \>   $\lnot\next (\lnot p \lor \lnot q)$\\[\lgap]
  \> $=$  \>  \Hint{(\ref{E:distNextOr}) with $p,q := \lnot p, \lnot q$}\\[\lgap]
  \> \>   $\lnot (\next\lnot p \lor \next \lnot q)$\\[\lgap]
  \> $=$  \>  \Hint{(\ref{E:selfDual}) twice}\\[\lgap]
  \> \>   $\lnot(\lnot\next p \lor \lnot\next q)$\\[\lgap]
  \> $=$  \>  \Hint{(3.47) DeMorgan's Law}\\[\lgap]
  \> \>   $\lnot\lnot(\next p \land \next q)$\\[\lgap]
  \> $=$  \>  \Hint{(3.12) Double Negation}\\[\lgap]
  \> \>   $\next p \land \next q$\\[\lgap]
\end{tabbing}
\myqed\\[\lgap]


\begin{equation}\label{E:distNextEquiv}
\textbf{Distributivity of $\next$ over $\equiv$:}\quad \next (p \equiv q) \equiv \next p \equiv \next q
\end{equation}

\emph{Proof:}
\begin{tabbing}
\hspace{\mymathindent} \= $= \;$ \= \kill
  \> \>   $\next (p \equiv q)$\\[\lgap]
  \> $=$  \>  \Hint{(3.80) Mutual Implication}\\[\lgap]
  \> \>   $\next ((p \Rightarrow q) \land (q \Rightarrow p))$\\[\lgap]
  \> $=$  \>  \Hint{(\ref{E:distNextAnd}) Distributivity of $\next$ over $\land$}\\[\lgap]
  \> \>   $\next (p \Rightarrow q) \land \next (p \Rightarrow q)$\\[\lgap]
  \> $=$  \>  \Hint{(\ref{E:distNextImp}) Distributivity of $\next$ over $\Rightarrow$}\\[\lgap]
  \> \>   $(\next p \Rightarrow \next q) \land (\next q \Rightarrow \next p)$\\[\lgap]
  \> $=$  \>  \Hint{(3.80) Mutual Implication}\\[\lgap]
  \> \>   $\next p \equiv \next q$
\end{tabbing}
\myqed\\[\lgap]

\subsection{Until}

\begin{equation}\label{E:distNextUntil}
\textbf{Axiom, Distributivity of $\next$ over $\until$:}\quad \next (p \until q) \equiv \next p \until \next q
\end{equation}

\begin{equation}\label{E:expansionUntil}
\textbf{Axiom, Expansion of $\until$:}\quad p \until q \equiv q \lor (p \land \next (p \until q))
\end{equation}


\begin{equation}\label{E:idemUntil}
\textbf{Idempotency of $\until$:}\quad p \until p \equiv p
\end{equation}

\emph{Proof:}
\begin{tabbing}
\hspace{\mymathindent} \= $= \;$ \= \kill
  \> \>   $p \until p$\\[\lgap]
  \> $=$  \>  \Hint{(\ref{E:expansionUntil}) Expansion of $\until$}\\[\lgap]
  \> \>   $p \lor (p \land \next(p \until p))$\\[\lgap]
  \> $=$  \>  \Hint{(3.43b) Absorption}\\[\lgap]
  \> \>   $p$\\[\lgap]
\end{tabbing}
\myqed\\[\lgap]

\begin{equation}\label{E:zeroUntil}
\textbf{Right zero of $\until$:}\quad p \until true \equiv true
\end{equation}

\emph{Proof:}
\begin{tabbing}
\hspace{\mymathindent} \= $= \;$ \= \kill
  \> \>   $p \until true$\\[\lgap]
  \> $=$  \>  \Hint{(\ref{E:expansionUntil}) Expansion of $\until$}\\[\lgap]
  \> \>   $true \lor (p \land \next(p \until true))$\\[\lgap]
  \> $=$  \>  \Hint{(3.29) Zero of $\lor$}\\[\lgap]
  \> \>   $true$\\[\lgap]
\end{tabbing}
\myqed\\[\lgap]


\begin{equation}\label{E:untilImpOr}
p \until q \Rightarrow p \lor q
\end{equation}

\emph{Proof:}
\begin{tabbing}
\hspace{\mymathindent} \= $= \;$ \= \kill
  \> \>   $p \until q$\\[\lgap]
  \> $=$  \>  \Hint{(\ref{E:expansionUntil}) Expansion of $\until$}\\[\lgap]
  \> \>   $q \lor (p \land \next(p \until q))$\\[\lgap]
  \> $\Rightarrow$  \>  \Hint{(3.76d) with $p,q,r := q,p,\next(p \until q)$}\\[\lgap]
  \> \>   $p \lor q$\\[\lgap]
\end{tabbing}
\myqed\\[\lgap]


Note: we will take the following 6 expressions of until (which we know to be correct) as axioms for now:\\

\begin{equation}\label{E:untilOrImp}
\textbf{Axiom:}\quad (p \until r) \lor (q \until r) \Rightarrow (p \lor q) \until r
\end{equation}

\begin{equation}\label{E:untilAndImp}
\textbf{Axiom:}\quad p \until (q \land r) \Rightarrow (p \until q) \land (p \until r)
\end{equation}

\begin{equation}\label{E:untilAndEquiv}
\textbf{Axiom:}\quad (p \land q) \until r \equiv (p \until r) \land (q \until r)
\end{equation}

\begin{equation}\label{E:untilOrEquiv}
\textbf{Axiom:}\quad p \until (q \lor r) \equiv (p \until q) \lor (p \until r)
\end{equation}

\begin{equation}\label{E:untilIdem}
\textbf{Axiom:}\quad p \until (p \until q) \equiv p \until q
\end{equation}

\begin{equation}\label{E:untilIdemR}
\textbf{Axiom:}\quad (p \until q) \until q \equiv p \until q
\end{equation}

End Note\\

\subsection{Eventually}

\begin{equation}\label{E:defEvent}
\textbf{Definition of $\event$:}\quad \event p \equiv true \until p
\end{equation}

\begin{equation}\label{E:eventuality}
\textbf{Eventuality:}\quad p \until q \Rightarrow \event q
\end{equation}

\emph{Proof:}
\begin{tabbing}
\hspace{\mymathindent} \= $= \;$ \= \kill
  \> \>   $p \until q$\\[\lgap]
  \> $\Rightarrow$  \>  \Hint{(3.76a) Weakening}\\[\lgap]
  \> \>   $(p \until q) \lor (true \until q)$\\[\lgap]
  \> $\Rightarrow$  \>  \Hint{(\ref{E:untilOrImp})}\\[\lgap]
  \> \>   $(p \lor true) \until q$\\[\lgap]
  \> $=$  \>  \Hint{(3.29) Zero of $\lor$}\\[\lgap]
  \> \>   $true \until q$\\[\lgap]
  \> $=$  \>  \Hint{(\ref{E:defEvent}) Definition of $\event$}\\[\lgap]
  \> \>   $\event q$\\[\lgap]
\end{tabbing}
\myqed\\[\lgap]


\begin{equation}\label{E:expansionEvent}
\textbf{Expansion of $\event$:}\quad \event p \equiv p \lor \next\event p
\end{equation}

\emph{Proof:}
\begin{tabbing}
\hspace{\mymathindent} \= $= \;$ \= \kill
  \> \>   $\event p$\\[\lgap]
  \> $=$  \>  \Hint{(\ref{E:defEvent}) Definition of $\event$}\\[\lgap]
  \> \>   $true \until p$\\[\lgap]
  \> $=$  \>  \Hint{(\ref{E:expansionUntil}) Expansion of $\until$}\\[\lgap]
  \> \>   $p \lor (true \land \next(true \until p))$\\[\lgap]
  \> $=$  \>  \Hint{(\ref{E:defEvent}) Definition of $\event$}\\[\lgap]
  \> \>   $p \lor (true \land \next\event p)$\\[\lgap]
  \> $=$  \>  \Hint{(3.39) Identity of $\land$}\\[\lgap]
  \> \>   $p \lor \next\event p$\\[\lgap]
\end{tabbing}
\myqed\\[\lgap]

\begin{equation}\label{E:impEvent}
p \Rightarrow \event p
\end{equation}

\emph{Proof:}
\begin{tabbing}
\hspace{\mymathindent} \= $= \;$ \= \kill
  \> \>   $\event p$\\[\lgap]
  \> $=$  \>  \Hint{(\ref{E:expansionEvent}) Expansion of $\event$}\\[\lgap]
  \> \>   $p \lor \next\event p$\\[\lgap]
  \> $\Leftarrow$  \>  \Hint{(3.76a) Weakening}\\[\lgap]
  \> \>   $p$\\[\lgap]
\end{tabbing}
\myqed\\[\lgap]


\begin{equation}\label{E:nextEvent}
\next p \Rightarrow \event p
\end{equation}

\emph{Proof:}
\begin{tabbing}
\hspace{\mymathindent} \= $= \;$ \= \kill
  \> \>   $\event p$\\[\lgap]
  \> $=$  \>  \Hint{(\ref{E:defEvent}) Definition of $\event$}\\[\lgap]
  \> \>   $true \until p$\\[\lgap]
  \> $=$  \>  \Hint{(\ref{E:expansionUntil}) Expansion of $\until$}\\[\lgap]
  \> \>   $p \lor (true \land \next(true \until p))$\\[\lgap]
  \> $=$  \>  \Hint{(3.39) Identity of $\land$}\\[\lgap]
  \> \>   $p \lor \next(true \until p)$\\[\lgap]
  \> $=$  \>  \Hint{(\ref{E:distNextUntil}) Distributivity of $\next$ over $\until$}\\[\lgap]
  \> \>   $p \lor (\next true \until \next p)$\\[\lgap]
  \> $=$  \>  \Hint{(\ref{E:expansionUntil}) Expansion of $\until$}\\[\lgap]
  \> \>   $p \lor \next p \lor (\next true \land \next(\next true \until \next p))$\\[\lgap]
  \> $\Leftarrow$ \> \Hint{(3.76a) Weakening}\\[\lgap]
  \> \>   $\next p$\\[\lgap]
\end{tabbing}
\myqed\\[\lgap]


\begin{equation}\label{E:IdemEvent}
\textbf{Absorption of $\event$:}\quad \event\event p \equiv \event p
\end{equation}

\emph{Proof:}
\begin{tabbing}
\hspace{\mymathindent} \= $= \;$ \= \kill
  \> \>   $\event\event p$\\[\lgap]
  \> $=$  \>  \Hint{(\ref{E:defEvent}) Definition of $\event$, with $p := \event p$}\\[\lgap]
  \> \>   $true \until \event p$\\[\lgap]
  \> $=$  \>  \Hint{(\ref{E:defEvent}) Definition of $\event$}\\[\lgap]
  \> \>   $true \until (true \until p)$\\[\lgap]
  \> $=$  \>  \Hint{(\ref{E:untilIdem}) with $p,q := true,p$}\\[\lgap]
  \> \>   $true \until p$\\[\lgap]
  \> $=$  \>  \Hint{(\ref{E:defEvent}) Definition of $\event$}\\[\lgap]
  \> \>   $\event p$\\[\lgap]
\end{tabbing}
\myqed\\[\lgap]


\begin{equation}\label{E:dNextEvent}
\next\event p \equiv \event\next p
\end{equation}

\emph{Proof:}
\begin{tabbing}
\hspace{\mymathindent} \= $= \;$ \= \kill
  \> \>   $\next\event p$\\[\lgap]
  \> $=$  \>  \Hint{(\ref{E:defEvent}) Definition of $\event$}\\[\lgap]
  \> \>   $\next(true \until p)$\\[\lgap]
  \> $=$  \>  \Hint{(\ref{E:distNextUntil}) Distributivity of $\next$ over $\until$}\\[\lgap]
  \> \>   $\next true \until \next p$\\[\lgap]
  \> $=$  \>  \Hint{(\ref{E:nextTruth})}\\[\lgap]
  \> \>   $true \until \next p$\\[\lgap]
  \> $=$  \>  \Hint{(\ref{E:defEvent}) Definition of $\event$}\\[\lgap]
  \> \>   $\event\next p$\\[\lgap]
\end{tabbing}
\myqed\\[\lgap]


\begin{equation}\label{E:distEventOr}
\textbf{Distributivity of $\event$ over $\lor$:}\quad \event(p \lor q) \equiv \event p \lor \event q
\end{equation}

\emph{Proof:}
\begin{tabbing}
\hspace{\mymathindent} \= $= \;$ \= \kill
  \> \>   $\event(p \lor q)$\\[\lgap]
  \> $=$  \>  \Hint{(\ref{E:defEvent}) Definition of $\event$}\\[\lgap]
  \> \>   $true \until (p \lor q)$\\[\lgap]
  \> $=$  \>  \Hint{(\ref{E:untilOrEquiv})}\\[\lgap]
  \> \>   $(true \until p) \lor (true \until q)$\\[\lgap]
  \> $=$  \>  \Hint{(\ref{E:defEvent}) Definition of $\event$ twice}\\[\lgap]
  \> \>   $\event p \lor \event q$\\[\lgap]
\end{tabbing}
\myqed\\[\lgap]


\begin{equation}\label{E:distEventAnd}
\textbf{Distributivity of $\event$ over $\land$:}\quad \event(p \land q) \Rightarrow \event p \land \event q
\end{equation}

\emph{Proof:}
\begin{tabbing}
\hspace{\mymathindent} \= $= \;$ \= \kill
  \> \>   $\event(p \land q)$\\[\lgap]
  \> $=$  \>  \Hint{(\ref{E:defEvent}) Definition of $\event$}\\[\lgap]
  \> \>   $true \until (p \land q)$\\[\lgap]
  \> $\Rightarrow$  \>  \Hint{(\ref{E:untilAndImp})}\\[\lgap]
  \> \>   $(true \until p) \land (true \until q)$\\[\lgap]
  \> $=$  \>  \Hint{(\ref{E:defEvent}) Definition of $\event$ twice}\\[\lgap]
  \> \>   $\event p \land \event q$\\[\lgap]
\end{tabbing}
\myqed\\[\lgap]

\subsection{Always}

\begin{equation}\label{E:defAlways}
\textbf{Definition of $\always$:}\quad \always p \equiv \lnot\event\lnot p
\end{equation}

\begin{equation}\label{E:dualAlways}
\textbf{Dual of $\always$:}\quad \lnot\always p \equiv \event\lnot p
\end{equation}

\emph{Proof:}
\begin{tabbing}
\hspace{\mymathindent} \= $= \;$ \= \kill
  \> \>   $\lnot\always p \equiv \event\lnot p$\\[\lgap]
  \> $=$  \>  \Hint{(3.11) with $p,q := \always p, \event\lnot p$}\\[\lgap]
  \> \>   $\always p \equiv \lnot\event\lnot p$
\end{tabbing}
which is (\ref{E:defAlways}). \myqed\\[\lgap]


\begin{equation}\label{E:dualEvent}
\textbf{Dual of $\event$:}\quad \lnot\event p \equiv \always\lnot p
\end{equation}

\emph{Proof:}
\begin{tabbing}
\hspace{\mymathindent} \= $= \;$ \= \kill
  \> \>   $\always\lnot p$\\[\lgap]
  \> $=$  \>  \Hint{(\ref{E:defAlways}) Definition of $\always$}\\[\lgap]
  \> \>   $\lnot\event\lnot\lnot p$\\[\lgap]
  \> $=$  \>  \Hint{(3.12) Double Negation}\\[\lgap]
  \> \>   $\lnot\event p$\\[\lgap]
\end{tabbing}
\myqed\\[\lgap]


\begin{equation}\label{E:eventAsAlways}
\event p \equiv \lnot\always\lnot p
\end{equation}

\emph{Proof:}
\begin{tabbing}
\hspace{\mymathindent} \= $= \;$ \= \kill
  \> \>   $\lnot\always\lnot p$\\[\lgap]
  \> $=$  \>  \Hint{(\ref{E:defAlways}) Definition of $\always$}\\[\lgap]
  \> \>   $\lnot\lnot\event\lnot\lnot p$\\[\lgap]
  \> $=$  \>  \Hint{(3.12) Double Negation, twice}\\[\lgap]
  \> \>   $\event p$\\[\lgap]
\end{tabbing}
\myqed\\[\lgap]


\begin{equation}\label{E:expansionAlways}
\textbf{Expansion of $\always$:}\quad \always p \equiv p \land \next\always p
\end{equation}

\emph{Proof:}
\begin{tabbing}
\hspace{\mymathindent} \= $= \;$ \= \kill
  \> \>   $\always p$\\[\lgap]
  \> $=$  \>  \Hint{(\ref{E:defAlways}) Definition of $\always$}\\[\lgap]
  \> \>   $\lnot\event\lnot p$\\[\lgap]
  \> $=$  \>  \Hint{(\ref{E:defEvent}) Definition of $\event$ with $p := \lnot p$}\\[\lgap]
  \> \>   $\lnot(true \until \lnot p)$\\[\lgap]
  \> $=$  \>  \Hint{(\ref{E:expansionUntil}) Expansion of $\until$}\\[\lgap]
  \> \>   $\lnot(\lnot p \lor (true \land \next(true \until \lnot p)))$\\[\lgap]
  \> $=$  \>  \Hint{(3.39) Identity of $\land$}\\[\lgap]
  \> \>   $\lnot(\lnot p \lor \next(true \until \lnot p))$\\[\lgap]
  \> $=$  \>  \Hint{(3.47b) De Morgan's Law}\\[\lgap]
  \> \>   $\lnot\lnot p \land \lnot\next(true \until \lnot p)$\\[\lgap]
  \> $=$  \>  \Hint{(3.12) Double Negation}\\[\lgap]
  \> \>   $p \land \lnot\next(true \until \lnot p)$\\[\lgap]
  \> $=$  \>  \Hint{(\ref{E:defEvent}) Defintion of $\event$}\\[\lgap]
  \> \>   $p \land \lnot\next\event\lnot p$\\[\lgap]
  \> $=$  \>  \Hint{(\ref{E:dualAlways}) Dual of $\always$}\\[\lgap]
  \> \>   $p \land \lnot\next\lnot\always p$\\[\lgap]
  \> $=$  \>  \Hint{(\ref{E:linearity}) Linearity}\\[\lgap]
  \> \>   $p \land \next\always p$\\[\lgap]
\end{tabbing}
\myqed\\[\lgap]


\begin{equation}\label{E:IdemAlways}
\textbf{Absorption of $\always$:}\quad \always\always p \equiv \always p
\end{equation}

\emph{Proof:}
\begin{tabbing}
\hspace{\mymathindent} \= $= \;$ \= \kill
  \> \>   $\always\always p \equiv \always p$\\[\lgap]
  \> $=$  \>  \Hint{(\ref{E:defAlways}) Definition of $\always$ with $p := \always p$}\\[\lgap]
  \> \>   $\lnot\event\lnot\always p \equiv \always p$\\[\lgap]
  \> $=$  \>  \Hint{(3.11) with $p,q := \event\lnot\always p, \always p$}\\[\lgap]
  \> \>   $\event\lnot\always p \equiv \lnot\always p$\\[\lgap]
  \> $=$  \>  \Hint{(\ref{E:dualAlways}) Dual of $\always$, twice}\\[\lgap]
  \> \>   $\event\event\lnot p \equiv \event\lnot p$\\[\lgap]
  \> $=$  \>  \Hint{(\ref{E:IdemEvent}) Absorption of $\event$}\\[\lgap]
  \> \>   $\event\lnot p \equiv \event\lnot p$\\[\lgap]
\end{tabbing}
which is (3.5) with $p := \event\lnot p$. \myqed\\[\lgap]


\begin{equation}\label{E:dNextAlways}
\next\always p \equiv \always\next p
\end{equation}

\emph{Proof:}
\begin{tabbing}
\hspace{\mymathindent} \= $= \;$ \= \kill
  \> \>   $\next\always p$\\[\lgap]
  \> $=$  \>  \Hint{(\ref{E:defAlways}) Definition of $\always$}\\[\lgap]
  \> \>   $\next\lnot\event\lnot p$\\[\lgap]
  \> $=$  \>  \Hint{(\ref{E:selfDual}) Self-dual}\\[\lgap]
  \> \>   $\lnot\next\event\lnot p$\\[\lgap]
  \> $=$  \>  \Hint{(\ref{E:dNextEvent}) with $p := \lnot p$}\\[\lgap]
  \> \>   $\lnot\event\next\lnot p$\\[\lgap]
  \> $=$  \>  \Hint{(\ref{E:selfDual}) Self-dual}\\[\lgap]
  \> \>   $\lnot\event\lnot\next p$\\[\lgap]
  \> $=$  \>  \Hint{(\ref{E:defAlways}) Definition of $\always$}\\[\lgap]
  \> \>   $\always\next p$\\[\lgap]
\end{tabbing}
\myqed\\[\lgap]


\begin{equation}\label{E:impAlways}
\always p \Rightarrow p
\end{equation}

\emph{Proof:}
\begin{tabbing}
\hspace{\mymathindent} \= $= \;$ \= \kill
  \> \>   $\always p$\\[\lgap]
  \> $=$  \>  \Hint{(\ref{E:defAlways}) Definition of $\always$}\\[\lgap]
  \> \>   $\lnot\event\lnot p$\\[\lgap]
  \> $=$  \>  \Hint{(\ref{E:expansionEvent}) Expansion of $\event$}\\[\lgap]
  \> \>   $\lnot(\lnot p \lor \next\event\lnot p)$\\[\lgap]
  \> $=$  \>  \Hint{(3.47b) De Morgan's Law}\\[\lgap]
  \> \>   $\lnot\lnot p \land \lnot\next\event\lnot p$\\[\lgap]
  \> $=$  \>  \Hint{(3.12) Double Negation}\\[\lgap]
  \> \>   $p \land \lnot\next\event\lnot p$\\[\lgap]
  \> $\Rightarrow$  \>  \Hint{(3.76b) Strengthening}\\[\lgap]
  \> \>   $p$\\[\lgap]
\end{tabbing}
\myqed\\[\lgap]


\begin{equation}\label{E:impAlwaysE}
\always p \Rightarrow \event p
\end{equation}

\emph{Proof:}
\begin{tabbing}
\hspace{\mymathindent} \= $= \;$ \= \kill
  \> \>   $\always p$\\[\lgap]
  \> $\Rightarrow$  \>  \Hint{(\ref{E:impAlways})}\\[\lgap]
  \> \>   $p$\\[\lgap]
  \> $\Rightarrow$  \>  \Hint{(\ref{E:impEvent})}\\[\lgap]
  \> \>   $\event p$\\[\lgap]
\end{tabbing}
\myqed\\[\lgap]


\begin{equation}\label{E:impAlwaysN}
\always p \Rightarrow \next p
\end{equation}

\emph{Proof:}
\begin{tabbing}
\hspace{\mymathindent} \= $= \;$ \= \kill
  \> \>   $\always p$\\[\lgap]
  \> $=$  \>  \Hint{(\ref{E:expansionAlways}) Expansion of $\always$}\\[\lgap]
  \> \>   $p \land \next\always p$\\[\lgap]
  \> $=$  \>  \Hint{(\ref{E:dNextAlways})}\\[\lgap]
  \> \>   $p \land \always\next p$\\[\lgap]
  \> $=$  \>  \Hint{(\ref{E:expansionAlways}) Expansion of $\always$ with $p := \next p$}\\[\lgap]
  \> \>   $p \land \next p \land \next\always\next p$\\[\lgap]
  \> $\Rightarrow$  \>  \Hint{(3.76b) Strengthening}\\[\lgap]
  \> \>   $\next p$\\[\lgap]
\end{tabbing}
\myqed\\[\lgap]


\begin{equation}\label{E:impAlwaysNA}
\always p \Rightarrow \next\always p
\end{equation}

\emph{Proof:}
\begin{tabbing}
\hspace{\mymathindent} \= $= \;$ \= \kill
  \> \>   $\always p$\\[\lgap]
  \> $=$  \>  \Hint{(\ref{E:expansionAlways}) Expansion of $\always$}\\[\lgap]
  \> \>   $p \land \next\always p$\\[\lgap]
  \> $\Rightarrow$  \>  \Hint{(3.76b) Strengthening}\\[\lgap]
  \> \>   $\next\always p$\\[\lgap]
\end{tabbing}
\myqed\\[\lgap]


\begin{equation}\label{E:exAlwaysNot}
\always\lnot p \Rightarrow \lnot\always p
\end{equation}

\emph{Proof:}
\begin{tabbing}
\hspace{\mymathindent} \= $= \;$ \= \kill
  \> \>   $\always\lnot p$\\[\lgap]
  \> $\Rightarrow$  \>  \Hint{(\ref{E:impAlwaysE})}\\[\lgap]
  \> \>   $\event\lnot p$\\[\lgap]
  \> $=$  \>  \Hint{(\ref{E:dualAlways}) Dual of $\always$}\\[\lgap]
  \> \>   $\lnot\always p$\\[\lgap]
\end{tabbing}
\myqed\\[\lgap]


\begin{equation}\label{E:excludedMid}
\textbf{Excluded Middle:}\quad \event p \lor \always\lnot p
\end{equation}

\emph{Proof:}
\begin{tabbing}
\hspace{\mymathindent} \= $= \;$ \= \kill
  \> \>   $\event p \lor \always\lnot p$\\[\lgap]
  \> $=$  \>  \Hint{(\ref{E:dualEvent}) Dual of $\event$}\\[\lgap]
  \> \>   $\event p \lor \lnot\event p$\\[\lgap]
\end{tabbing}
which is (3.28) Excluded middle, with $p := \event p$. \myqed\\[\lgap]


\begin{equation}\label{E:distAlwaysImp}
\textbf{Distributivity of $\always$ over $\Rightarrow$:}\quad \always (p \Rightarrow q) \Rightarrow (\always p \Rightarrow \always q)
\end{equation}

\emph{Proof:}
\begin{tabbing}
\hspace{\mymathindent} \= $= \;$ \= \kill
  \> \>   $\always(p \Rightarrow q)$\\[\lgap]
  \> $=$  \>  \Hint{(\ref{E:defAlways})}\\[\lgap]
  \> \>   $\lnot\event\lnot(p \Rightarrow q)$\\[\lgap]
  \> $=$  \>  \Hint{(3.59) Implication}\\[\lgap]
  \> \>   $\lnot\event\lnot(\lnot p \lor q)$\\[\lgap]
  \> $=$  \>  \Hint{(3.47b) De Morgan}\\[\lgap]
  \> \>   $\lnot\event(\lnot\lnot p \land \lnot q)$\\[\lgap]
  \> $=$  \>  \Hint{(3.12) Double Negation}\\[\lgap]
  \> \>   $\lnot\event(p \land \lnot q)$\\[\lgap]
  \> $\Rightarrow$  \>  \Hint{(\ref{E:distEventAnd}) Distributivity of $\event$ over $\land$}\\[\lgap]
\end{tabbing}
\myqed\\[\lgap]


\subsection{Wait}

\section{Conclusion}

The results are cooool!\\

\bibliographystyle{plain}
\bibliography{Vega-Paper}
\end{document}
