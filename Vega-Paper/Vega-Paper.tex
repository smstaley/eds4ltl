% David Vega and Stan Warford
% Pepperdine University
% File: Vega-Paper

\documentclass[fleqn, leqno]{article}

\usepackage{times}
\usepackage{amsmath, amsthm, amssymb,latexsym}
\usepackage{paralist}
\usepackage{ellipsis}
\usepackage{array}

\newcommand{\lgap}{2pt}                             % Line gap
\newcommand{\llgap}{6pt}                            % Larger line gap
\newcommand{\lllgap}{32pt}                          % Largest line gap for students to write in
\newcommand{\mymathindent}{24pt}                    % Indentation for math tabbing
\newcommand{\equivs}{\ensuremath{\;\equiv\;}}       % Equivales with space
\newcommand{\equivss}{\ensuremath{\;\;\equiv\;\;}}  % Equivales with double space
\newcommand{\nequiv}{\ensuremath{\not\equiv}}       % Inequivalent
\newcommand{\impl}{\ensuremath{\Rightarrow}}        % Implies
\newcommand{\nimpl}{\ensuremath{\not\Rightarrow}}   % Does not imply
\newcommand{\foll}{\ensuremath{\Leftarrow}}         % Follows from
\newcommand{\nfoll}{\ensuremath{\not\Leftarrow}}    % Does not follow from


% Macros for Temporal Operators
\newcommand{\Until}{\;\mathcal{U}\;}
\newcommand{\Wait}{\;\mathcal{W}\;}
\newcommand{\Next}{\bigcirc}
\newcommand{\Event}{\Diamond}
\newcommand{\Always}{\Box}

\newcommand{\myqed}{\hfill\rule[-.23ex]{1.2ex}{2.0ex}}

% Thanks to David Gries for sharing the following macros
% Macros for quantifications.
\newcommand{\thedr}{\rule[-.25ex]{.32mm}{1.75ex}}   % Symbol that separates dummy from range in quantification
\newcommand{\dr}{\;\,\thedr\,\;}                    % Symbol that separates dummy from range, with spacing
\newcommand{\rb}{:}                                 % Symbol that separates range from body in quantification
\newcommand{\drrb}{\;\thedr\,{:}\;}                 % Symbol that separates dummy from body when range is missing
\newcommand{\all}{\forall}                          % Universal quantification
\newcommand{\ext}{\exists}                          % Existential quantification

% Macros for proof hints
\newcommand{\Gll} {\langle}                         % Open hint
\newcommand{\Ggg} {\rangle}                         % Close hint
\newlength{\Glllength}                              % Length of open hint symbol
\settowidth{\Glllength}{$.\Gll$}
\newcommand{\Hint}[1]     {\ \ \ $\Gll              \mbox{#1} \Ggg$ }   % Single line hint
\newcommand{\Hintfirst}[1]{\ \ \ $\Gll              \mbox{#1}$ }        % First line of multiline hint
\newcommand{\Hintmid}[1]  {\ \ $\hspace{\Glllength} \mbox{#1}$ }        % Middle line of multiline hint
\newcommand{\Hintlast}[1] {\ \ $\hspace{\Glllength} \mbox{#1} \Ggg$ }   % Last line of multiline hint

% Single and double quotes
\newcommand{\Lq}{\mbox{`}}
\newcommand{\Rq}{\mbox{'}}
\newcommand{\Lqq}{\mbox{``}}
\newcommand{\Rqq}{\mbox{''}}

\oddsidemargin  0.0in
\evensidemargin 0.0in
\textwidth      6.0in
\headheight     0.0in
\topmargin      0.0in
\textheight=8.5in
\parindent=0in
\pagestyle{plain}

\title{An Equational Deductive System\\for Linear Temporal Logic}
\author{David Vega\thanks{Research supported by Tooma Undergraduate Research Fellowship Program, Summer 2009
        and academic year 2009-10.}\\
   Computer Science Department\\
   Pepperdine University\\
   Malibu, CA 90265
   \and
   J. Stanley Warford\\
   Computer Science Department\\
   Pepperdine University\\
   Malibu, CA 90265}
\date{} % Required for no date to appear in heading

\begin{document}
\maketitle
\begin{abstract}
This paper presents an equational deductive system for linear temporal logic.
It differs from previous developments of such systems in several respects.
First, it presents a numbered list of axioms and theorems to indicate which formulas are assumed, which formulas are
derived, and for those that are derived, which previous formulas they depend on.
Second, it gives a proof of every theorem.
Third, the proofs are governed by an equational deductive system as opposed to the older Hilbert-style deductive systems.
Fourth, it presents several new and interesting linear temporal theorems.\end{abstract}

\thispagestyle{plain}

\section{Introduction}

Propositional calculus is a formal system of logic based on the unary operator negation $\lnot$,
the binary operators conjunction $\land$, disjunction $\lor$, implies $\impl$ (also written $\rightarrow$),
and equivalence $\equiv$ (also written $\leftrightarrow$),
variables (lowercase letters $p$, $q$, \dots), and the constants $true$ and $false$.
Hilbert-style logic systems, $\mathcal{H}$, are the deductive logic systems traditionally used in mathematics.
In the late 1980's, Dijkstra and Scholten \cite{DandS}, and Feijen \cite{Feij} developed a method of proving
program correctness with a new logic based on an equational style.
This equational deductive system, $\mathcal{E}$, has been the basis of textbooks by Kaldewaij \cite{Kald},
Cohen \cite{Cohen}, and Gries and Schneider \cite{LADM}.\\

The equational deductive system of proofs is slow to be adopted by the computer science community.
The problem is two-fold.
First, a fair amount of technical detail must be mastered,
and many computer science educators and practitioners do not have the requisite
knowledge of formal logic systems, much less the equational deductive system.
Scores of textbooks for discrete mathematics for computer science could be cited that give only a cursory introduction to
formal logic. Most of these texts, such as the classic one by Rosen \cite{Rosen} are beginning to move to a more
formal treatment of logic appropriate for computer science.\\

Second, even when formal logic is taught at a depth necessary to apply it to program proofs, the older Hilbert style
still dominates.
The previously-cited texts \cite{Cohen, LADM, Kald} are among the few that rely on the equational deductive system. 
More typical is Ben-Ari's book \cite{Ben}, which is based enitrely on natural deduction systems.\\

Linear temporal logic describes how the truth values of propositions change over time.
It extends the propositional operators with the unary operators next $\Next$, eventually $\Event$, and always $\Always$,
and the binary operators until $\Until$ and wait $\Wait$.
Propositional calculus applies to program correctness with the formulation of the Hoare triple to establish invariants
in programs that terminate.
Temporal logic applies to program correctness with concurrent processing to establish safety and liveness properties
in programs that possibly do not terminate.\\

As is the case for propositional and predicate calculus, most treatments of linear temporal logic use $\mathcal{H}$
instead of $\mathcal{E}$. Typical are Ben-Ari \cite{Ben2}, Emerson \cite{Emer}, Kr\"{o}ger \cite{Kroger},
Manna and Pnueli \cite{Manna}, and Schneider \cite{Schn}.
The only appearance of an equational proof of a temporal logic theorem appears to be a single example in \cite{Schn},
which otherwise uses a Hilbert-style system for temporal logic.
The development of linear temporal logic in all these works is motivated by its use to prove correctness of concurrent programs.
The presentation typically consists of lists of valid temporal formulas, with little emphasis of which formulas are required
as axioms and which are theorems that can be proved using a deductive system.\\

This paper presents an equational deductive system for linear temporal logic.
It differs from previous developments of such systems in several respects.
First, it presents a numbered list of axioms and theorems to indicate which formulas are assumed, which formulas are
derived, and for those that are derived, which previous formulas they depend on.
Second, it gives a proof of every theorem.
Third, the proofs are given in the equational style $\mathcal{E}$ instead of $\mathcal{H}$.
Fourth, it presents several new and interesting linear temporal theorems.\\

Section 2 describes the deductive axioms and the proof rules for $\mathcal{E}$.
It also defines the syntax and semantics of linear temporal logic.
Section 3 presents the equational deductive system for linear temporal logic.\\

\section{Background}

\subsection{Equational Deductive Systems}

This section follows Gries and Schneider \cite{LADM}.\\

The definition of an expression has four parts:
\begin{itemize}[$\bullet$]
\item A constant or variable is an expression.
\item If $E$ is an expression, then $(E)$ is an expression.
\item If $\circ$ is a unary prefix operator and $E$ is an expression, then $\circ E$ is an expression with operand $E$.
\item If $\star$ is a binary infix operator and $D$ and $E$ are expressions, then $D \star E$ is an expression with operands $D$
and $E$.
\end{itemize}

By convention, upper-case letters ({\itshape e.g.\/} $X$, $Y$, \dots) represent expressions,
and lower-case letters ({\itshape e.g.\/} $x$, $y$, \dots) represent variables.
In the propositional calculus, the constants are {\itshape true\/} and {\itshape false\/}.\\

Here is the table of precedences.\\

\setlength\extrarowheight{2pt}
\begin{tabular}{lr}
\hline
$[x := e]$ (textual substitution) & Highest precedence\\
$\neg$\quad $\Next$\quad $\Event$\quad $\Always$ &\\
$\Until$\quad $\Wait$ &\\
$=$\quad (conjunctional) &\\
$\lor$\quad $\land$ &\\
$\impl$\quad $\foll$ &\\
$\equiv$ \quad (associative) & Lowest precedence\\
\hline
\end{tabular}\\[\llgap]

Textual substitution has the highest precedence.
All the unary operators have the next highest precedence.
They are necessarily right associative.
For example, $\lnot \Next \lnot p$ means $\lnot (\Next (\lnot p))$.
In this system, two binary operators that have the same precedence require parentheses to disambiguate.
As in \cite{LADM}, conjunction $\land$ and disjunction $\lor$ have the same precedence so that $p\land q\lor r$
must be disambiguated as either $(p\land q)\lor r$ or $p\land (q\lor r)$.
This contrasts with many systems in which conjunction has higher precedence than disjunction.\\

Also consistent with the equational system of \cite{LADM} but different from most other deductive logic systems
is the difference between operators equals $=$ and equivales $\equiv$.
Equals applies to any mathematical type including, {\itshape e.g.\/}, boolean, natural number, and set.
Equivales applies only to boolean, and is commonly denoted $\leftrightarrow$ in other systems.
Another difference is that equals is conjunctive, while equivales is associative.
For example, the expression $p = q = r$ means $(p = q) \land (q = r)$, while the expression $p \equiv q \equiv r$
can be taken as either $(p \equiv q) \equiv r$ or $p \equiv (q \equiv r)$.
This property of equivales is the first axiom in the equational deductive system of \cite{LADM}.
\\

The equational deductive system relies on the three deductive axioms for equality
\[
\textbf{Reflexivity:}\quad x=x
\]
\[
\textbf{Symmetry:}\quad (x=y) = (y=x)
\]
\[
\textbf{Transitivity:}\quad \frac{X=Y, \quad Y=Z}{X=Z}
\]
and the proof rule
\[
\textbf{Leibniz:}\quad \frac{X=Y}{E[z:=X]=E[z:=Y]}
\]

where the square bracket indicates textual substitution of expression $X$ for variable $z$ and substitution
of expression $Y$ for variable $z$.
Roughly speaking, Leibniz allows for the substitution of equals for equals in a proof step.
The general form of a proof step is

\begin{tabbing}
\hspace{\mymathindent} \= $= \;$ \=  \kill
  \> \>   $E[z:=X]$\\[\lgap]
  \> $=$  \>  \Hint{$X=Y$} \\[\lgap]
  \> \>   $E[z:=Y]$
\end{tabbing}

where the expression enclosed in angle brackets $\Gll\;\Ggg$ called the ``hint'' is the justification for the step.
An example of a proof step from the proof of theorem (\ref{E:distNextAnd}) below is

\begin{tabbing}
\hspace{\mymathindent} \= $= \;$ \= \kill
  \> \>   $\lnot\Next (\lnot p \lor \lnot q)$\\[\lgap]
  \> $=$  \>  \Hint{(\ref{E:distNextOr}) with $p,q := \lnot p, \lnot q$}\\[\lgap]
  \> \>   $\lnot (\Next\lnot p \lor \Next \lnot q)$
\end{tabbing}

This proof step uses the previously proved theorem (\ref{E:distNextOr}), distributivity of $\Next$ over $\lor$,
which is $\Next (p \lor q) \equiv \Next p \lor \Next q$.
The expressions in Leibniz for the step are

\begin{tabbing}
\hspace{\mymathindent} \= $= \;$ \= \kill
  \> $E:\quad \lnot z$\\[\lgap]
  \> $X:\quad \Next (\lnot p \lor \lnot q)$\\[\lgap]
  \> $Y:\quad \Next \lnot p \lor \Next \lnot q$
\end{tabbing}

The textual substitutions are

\begin{tabbing}
\hspace{\mymathindent} \= $= \;$ \= \kill
  \> $E[z:=X]:\quad \lnot\Next (\lnot p \lor \lnot q)$\\[\lgap]
  \> $E[z:=Y]:\quad \lnot\Next (\lnot p \lor \lnot q)$
\end{tabbing}

And the justification in the hint $X=Y$ comes from the textual substitution of $\lnot p$ for $p$
and $\lnot q$ for $q$ in (\ref{E:distNextOr}) as follows

\begin{tabbing}
\hspace{\mymathindent} \= $= \;$ \= \kill
  \> $(\Next (p \lor q) \equiv \Next p \lor \Next q)[p,q:=\lnot p, \lnot q]:\quad
      \Next (\lnot p \lor \lnot q) \equiv \Next \lnot p \lor \Next \lnot q$
\end{tabbing}

Gries and Schneider \cite{LADM} extend the proof format to incorporate implication using its transitive properties
with itself and with equivales.
An example is a proof of (\ref{E:impEvent}), $p \Rightarrow \Event p$.

\begin{tabbing}
\hspace{\mymathindent} \= $= \;$ \= \kill
  \> \>   $\Event p$\\[\lgap]
  \> $=$  \>  \Hint{(\ref{E:expansionEvent}) Expansion of $\Event$}\\[\lgap]
  \> \>   $p \lor \Next\Event p$\\[\lgap]
  \> $\Leftarrow$  \>  \Hint{Weakening $p\impl p\lor q$ with $q:=\Next\Event p$}\\[\lgap]
  \> \>   $p$
\end{tabbing}

Because $\Event p$ equivales $p \lor \Next\Event p$, and $p \lor \Next\Event p$ follows from $p$, it follows by
transitivity that $\Event p$ follows from $p$. 

\subsection{Temporal Logic}

This section follows Manna and Pnueli \cite{Manna} and Schneider \cite{Schn}.\\

The operators of propositional calculus, $\neg$, $=$, $\land$, $\lor$, $\impl$, $\foll$, and $\equiv$ are static.
That is, they apply at a single point in time.
Each operator has a truth table that dictates how to evaluate the truth value of an expression.
A state is an assignment of a truth value to each variable in the expression.
A given boolean expression may be false in all states, true in some states and false in others, or true in all states, in which case the expression is known as a theorem or validity or tautology.\\

The operators of temporal logic, $\Next$, $\Event$, $\Always$, $\Until$, and $\Wait$ are dynamic.
That is, they do not apply at a single point in time, but apply over an infinite sequence of states.
Each state corresponds to a discrete point in time that represents one point in the execution of a program,
possibly having several threads running concurrently but whose instruction executions have been serialized.
As one instruction in the program executes, the state changes, and hence the truth value of an expression may change as well.\\

A model $\sigma$ is an infinite sequence of the form

\begin{tabbing}
\hspace{\mymathindent} \= $= \;$ \= \kill
  \> $\sigma: s_0, s_1, s_2, \dots$
\end{tabbing}

where $s_0$ is the initial state and each state $s_i, 0 \le i$ is the state at time $i$.
For example, suppose $x$ is an integer variable whose value varies at each step of the computation.
Then $x$ and the expression $x < 10$, known as a state expression, might evolve as follows.\\

\begin{tabular}{c|ccccccc}
  $\sigma$   & $s_0$ & $s_1$ & $s_2$ & $s_3$ & $s_4$ & \dots \\
  \hline
  $x$        & 8     & 9     & 10    & 11    & 12    & \dots\\
  $x<10$     & T     & T     & F     & F     & F     & \dots
\end{tabular}\\

The bottom row shows the evaluation of the state expression at each state.
Temporal logic extends propositional logic by considering the evolution of expression evaluations in time.
For example, if you assume that $x$ in the above sequence keeps increasing by one you can assert
informally in English, ``For the sequence $\sigma$, eventually $x<10$ will always be false.''\\

The notation

\begin{tabbing}
\hspace{\mymathindent} \= $= \;$ \= \kill
  \> $(\sigma, j) \models p$
\end{tabbing}

means that the expression $p$ holds at position $j$ in a sequence $\sigma$.
In the above example,

\begin{tabbing}
\hspace{\mymathindent} \= $= \;$ \= \kill
  \> $(\sigma, 1) \models x<10$
\end{tabbing}

You can read $\models$ as ``satisfies'', so the above expression is read as
``State 1 in sequence $\sigma$ satisfies $x<10$''.
Or, using ``holds'', you can say, ``$x<10$ holds in state 1 of sequence $\sigma$''.
The following sections use $\models$ to formalize the interpretation of each temporal operator.

\subsubsection*{The \textit{next} operator $\Next$}

The semantics of the unary prefix operator $\Next$ is

\begin{tabbing}
\hspace{\mymathindent} \= $= \;$ \= \kill
  \> $(\sigma, j) \models \Next p$ \quad iff \quad $(\sigma, j+1) \models p$
\end{tabbing}

That is, $\Next p$ holds at position $j$ iff $p$ holds at position $j+1$.\\

For example, in the above sequence $\Next x \ge 10$ holds at state $s_1$ because $x \ge 10$
holds at state $s_2$.
In other words,

\begin{tabbing}
\hspace{\mymathindent} \= $= \;$ \= \kill
  \> $(\sigma, 1) \models \Next x \ge 10$ \quad because \quad $(\sigma, 2) \models  x \ge 10$
\end{tabbing}

\subsubsection*{The \textit{until} operator $\Until$}

The semantics of the binary infix operator $\Until$ is

\begin{tabbing}
\hspace{\mymathindent} \= $= \;$ \= \kill
  \> $(\sigma, j) \models p \Until q$ \quad iff \quad $(\ext k \dr k \le j \rb (\sigma,k) \models q \land
      (\all j \dr j\le i < k \rb (\sigma,i) \models p))$
\end{tabbing}

\subsubsection*{The \textit{eventually} operator $\Event$}

The semantics of the unary prefix operator $\Event$ is

\begin{tabbing}
\hspace{\mymathindent} \= $= \;$ \= \kill
  \> $(\sigma, j) \models \Event p$ \quad iff \quad $(\ext k \dr k \ge j \rb (\sigma,k) \models p)$
\end{tabbing}

\subsubsection*{The \textit{always} operator $\Always$}

The semantics of the unary prefix operator $\Always$ is

\begin{tabbing}
\hspace{\mymathindent} \= $= \;$ \= \kill
  \> $(\sigma, j) \models \Always p$ \quad iff \quad $(\all k \dr k \ge j \rb (\sigma,k) \models p)$
\end{tabbing}


\subsubsection*{The \textit{wait} operator $\Wait$}

The semantics of the binary infix operator $\Wait$ in terms of $\Until$ and $\Always$ is

\begin{tabbing}
\hspace{\mymathindent} \= $= \;$ \= \kill
  \> $(\sigma, j) \models p \Wait q$ \quad iff \quad $(\sigma, j) \models p \Until q \; \lor \; (\sigma, j) \models \Always p$
\end{tabbing}


\section{The Equational Temporal System}

Ben-Ari \cite{Ben} gives a deductive system $\mathcal{L}$ for temporal logic, but it is a
natural deductive system in the Hilbert style of $\mathcal{G}$ and $\mathcal{H}$. This project is to develop a complete system of
temporal logic but using the equational deductive system based on the four proof rules described
in the first section. The system will consist of a small number of axioms and a larger number of
theorems. Because the proof rules in the equational deductive system are different from those
in $\mathcal{L}$, the proofs will be different as well.\\



\subsection{Next}

Note: below is our auto-label system for axioms and theorems.\\

\begin{equation}\label{E:selfDual}
\textbf{Axiom, Self-dual:}\quad \Next\lnot p \equiv \lnot\Next p
\end{equation}

\begin{equation}\label{E:distNextImp}
\textbf{Axiom, Distributivity of $\Next$ over $\Rightarrow$:}\quad \Next (p \Rightarrow q) \equiv \Next p \Rightarrow \Next q
\end{equation}

\begin{equation}\label{E:linearity}
\textbf{Linearity:}\quad \Next p \equiv \lnot\Next\lnot p
\end{equation}

\emph{Proof:}
\begin{tabbing}
\hspace{\mymathindent} \= $= \;$ \= \kill
  \> \>   $\Next p \equiv \lnot\Next\lnot p$\\[\lgap]
  \> $=$  \>  \Hint{(3.11) with $p,q := \Next\lnot p, \Next p$} \\[\lgap]
  \> \>   $\lnot\Next p \equiv \Next\lnot p$
\end{tabbing}
which is (\ref{E:selfDual}), Self-dual. \myqed\\[\lgap]


\begin{equation}\label{E:distNextOr}
\textbf{Distributivity of $\Next$ over $\lor$:}\quad \Next (p \lor q) \equiv \Next p \lor \Next q
\end{equation}


\emph{Proof:}
\begin{tabbing}
\hspace{\mymathindent} \= $= \;$ \= \kill
	\> \>   $\Next$(p $\lor$ q)\\[\lgap]
	\> $=$  \>  \Hint{(3.59) Implication}\\[\lgap]
	\> \>   $\Next$($\lnot$p $\Rightarrow$ q)\\[\lgap]
	\> $=$  \>  \Hint{(\ref{E:distNextImp}) Distributivity of $\Next$ over $\Rightarrow$}\\[\lgap]
	\> \>   $\Next\lnot$p $\Rightarrow$ $\Next$q\\[\lgap]
	\> $=$  \>  \Hint{(3.59) Implication}\\[\lgap]
	\> \>   $\lnot\Next\lnot$p $\lor$ $\Next$q\\[\lgap]
	\> $=$  \>  \Hint{(\ref{E:linearity}) Linearity}\\[\lgap]
	\> \>   $\Next$p $\lor$ $\Next$q
\end{tabbing}
\myqed\\[\lgap]

\begin{equation}\label{E:distNextAnd}
\textbf{Distributivity of $\Next$ over $\land$:}\quad \Next (p \land q) \equiv \Next p \land \Next q
\end{equation}


\emph{Proof:}
\begin{tabbing}
\hspace{\mymathindent} \= $= \;$ \= \kill
  \> \>   $\Next (p \land q)$\\[\lgap]
  \> $=$  \>  \Hint{(3.12) Double Negation, twice}\\[\lgap]
  \> \>   $\Next (\lnot\lnot p \land \lnot\lnot q)$\\[\lgap]
  \> $=$  \>  \Hint{(3.47b) De Morgan}\\[\lgap]
  \> \>   $\Next\lnot(\lnot p \lor \lnot q)$\\[\lgap]
  \> $=$  \>  \Hint{(\ref{E:selfDual}) with $p:= (\lnot p \lor \lnot q$)}\\[\lgap]
  \> \>   $\lnot\Next (\lnot p \lor \lnot q)$\\[\lgap]
  \> $=$  \>  \Hint{(\ref{E:distNextOr}) with $p,q := \lnot p, \lnot q$}\\[\lgap]
  \> \>   $\lnot (\Next\lnot p \lor \Next \lnot q)$\\[\lgap]
  \> $=$  \>  \Hint{(\ref{E:selfDual}) twice}\\[\lgap]
  \> \>   $\lnot(\lnot\Next p \lor \lnot\Next q)$\\[\lgap]
  \> $=$  \>  \Hint{(3.47a) De Morgan}\\[\lgap]
  \> \>   $\lnot\lnot(\Next p \land \Next q)$\\[\lgap]
  \> $=$  \>  \Hint{(3.12) Double Negation}\\[\lgap]
  \> \>   $\Next p \land \Next q$\\[\lgap]
\end{tabbing}
\myqed\\[\lgap]


\begin{equation}\label{E:distNextEquiv}
\textbf{Distributivity of $\Next$ over $\equiv$:}\quad \Next (p \equiv q) \equiv \Next p \equiv \Next q
\end{equation}

\emph{Proof:}
\begin{tabbing}
\hspace{\mymathindent} \= $= \;$ \= \kill
  \> \>   $\Next (p \equiv q)$\\[\lgap]
  \> $=$  \>  \Hint{(3.80) Mutual Implication}\\[\lgap]
  \> \>   $\Next ((p \Rightarrow q) \land (q \Rightarrow p))$\\[\lgap]
  \> $=$  \>  \Hint{(\ref{E:distNextAnd}) Distributivity of $\Next$ over $\land$}\\[\lgap]
  \> \>   $\Next (p \Rightarrow q) \land \Next (p \Rightarrow q)$\\[\lgap]
  \> $=$  \>  \Hint{(\ref{E:distNextImp}) Distributivity of $\Next$ over $\Rightarrow$}\\[\lgap]
  \> \>   $(\Next p \Rightarrow \Next q) \land (\Next q \Rightarrow \Next p)$\\[\lgap]
  \> $=$  \>  \Hint{(3.80) Mutual Implication}\\[\lgap]
  \> \>   $\Next p \equiv \Next q$
\end{tabbing}
\myqed\\[\lgap]


\begin{equation}\label{E:nextTruth}
\textbf{Truth:}\quad \Next true \equiv true
\end{equation}

\emph{Proof:}
\begin{tabbing}
\hspace{\mymathindent} \= $= \;$ \= \kill
	\> \>   $\Next true$\\[\lgap]
	\> $=$  \>  \Hint{(3.28) Excluded middle}\\[\lgap]
	\> \>   $\Next(p \lor \lnot p)$\\[\lgap]
	\> $=$  \>  \Hint{(\ref{E:distNextOr}) Distributivty of $\Next$ over $\lor$}\\[\lgap]
	\> \>   $\Next p \lor \Next\lnot p$\\[\lgap]
	\> $=$  \>  \Hint{(\ref{E:selfDual}) Self-dual}\\[\lgap]
	\> \>   $\Next p \lor \lnot\Next p$\\[\lgap]
	\> $=$  \>  \Hint{(3.28) Excluded middle}\\[\lgap]
	\> \>   $true$\\[\lgap]
\end{tabbing}
\myqed\\[\lgap]


\begin{equation}\label{E:nextFalse}
\textbf{Falsehood:}\quad \Next false \equiv false
\end{equation}

\emph{Proof:}
\begin{tabbing}
\hspace{\mymathindent} \= $= \;$ \= \kill
  \> \>   $\Next false \equiv false$\\[\lgap]
  \> $=$  \>  \Hint{(3.8) Definition of $false$} \\[\lgap]
  \> \>   $\Next\lnot true \equiv \lnot true$\\[\lgap]
  \> $=$  \>  \Hint{(3.11) with $p,q := true, \Next\lnot true$}\\[\lgap]
  \> \>   $\lnot\Next\lnot true \equiv true$\\[\lgap]
  \> $=$  \>  \Hint{(\ref{E:linearity}) Linearity}\\[\lgap]
  \> \>   $\Next true \equiv true$\\[\lgap]
\end{tabbing}
which is (\ref{E:nextTruth}). \myqed\\[\lgap]

\subsection{Until}

\begin{equation}\label{E:distNextUntil}
\textbf{Axiom, Distributivity of $\Next$ over $\Until$:}\quad \Next (p \Until q) \equiv \Next p \Until \Next q
\end{equation}

\begin{equation}\label{E:expansionUntil}
\textbf{Axiom, Expansion of $\Until$:}\quad p \Until q \equiv q \lor (p \land \Next (p \Until q))
\end{equation}


\begin{equation}\label{E:untilFalse}
\textbf{Axiom:}\quad p \Until false \equiv false
\end{equation}


\begin{equation}\label{E:idemUntil}
\textbf{Idempotency of $\Until$:}\quad p \Until p \equiv p
\end{equation}

\emph{Proof:}
\begin{tabbing}
\hspace{\mymathindent} \= $= \;$ \= \kill
  \> \>   $p \Until p$\\[\lgap]
  \> $=$  \>  \Hint{(\ref{E:expansionUntil}) Expansion of $\Until$}\\[\lgap]
  \> \>   $p \lor (p \land \Next(p \Until p))$\\[\lgap]
  \> $=$  \>  \Hint{(3.43b) Absorption}\\[\lgap]
  \> \>   $p$\\[\lgap]
\end{tabbing}
\myqed\\[\lgap]

\begin{equation}\label{E:zeroUntil}
\textbf{Right zero of $\Until$:}\quad p \Until true \equiv true
\end{equation}

\emph{Proof:}
\begin{tabbing}
\hspace{\mymathindent} \= $= \;$ \= \kill
  \> \>   $p \Until true$\\[\lgap]
  \> $=$  \>  \Hint{(\ref{E:expansionUntil}) Expansion of $\Until$}\\[\lgap]
  \> \>   $true \lor (p \land \Next(p \Until true))$\\[\lgap]
  \> $=$  \>  \Hint{(3.29) Zero of $\lor$}\\[\lgap]
  \> \>   $true$\\[\lgap]
\end{tabbing}
\myqed\\[\lgap]


\begin{equation}\label{E:untilImpOr}
p \Until q \Rightarrow p \lor q
\end{equation}

\emph{Proof:}
\begin{tabbing}
\hspace{\mymathindent} \= $= \;$ \= \kill
  \> \>   $p \Until q$\\[\lgap]
  \> $=$  \>  \Hint{(\ref{E:expansionUntil}) Expansion of $\Until$}\\[\lgap]
  \> \>   $q \lor (p \land \Next(p \Until q))$\\[\lgap]
  \> $\Rightarrow$  \>  \Hint{(3.76d) with $p,q,r := q,p,\Next(p \Until q)$}\\[\lgap]
  \> \>   $p \lor q$\\[\lgap]
\end{tabbing}
\myqed\\[\lgap]


Note: we will take the following 6 expressions of until (which we know to be correct) as axioms for now:\\

\begin{equation}\label{E:untilOrImp}
\textbf{Axiom:}\quad (p \Until r) \lor (q \Until r) \Rightarrow (p \lor q) \Until r
\end{equation}

\begin{equation}\label{E:untilAndImp}
\textbf{Axiom:}\quad p \Until (q \land r) \Rightarrow (p \Until q) \land (p \Until r)
\end{equation}

\begin{equation}\label{E:untilAndEquiv}
\textbf{Axiom:}\quad (p \land q) \Until r \equiv (p \Until r) \land (q \Until r)
\end{equation}

\begin{equation}\label{E:untilOrEquiv}
\textbf{Axiom:}\quad p \Until (q \lor r) \equiv (p \Until q) \lor (p \Until r)
\end{equation}

\begin{equation}\label{E:untilIdem}
\textbf{Axiom:}\quad p \Until (p \Until q) \equiv p \Until q
\end{equation}

\begin{equation}\label{E:untilIdemR}
\textbf{Axiom:}\quad (p \Until q) \Until q \equiv p \Until q
\end{equation}

End Note\\

\subsection{Eventually}

\begin{equation}\label{E:defEvent}
\textbf{Definition of $\Event$:}\quad \Event p \equiv true \Until p
\end{equation}

\begin{equation}\label{E:eventuality}
\textbf{Eventuality:}\quad p \Until q \Rightarrow \Event q
\end{equation}

\emph{Proof:}
\begin{tabbing}
\hspace{\mymathindent} \= $= \;$ \= \kill
  \> \>   $p \Until q$\\[\lgap]
  \> $\Rightarrow$  \>  \Hint{(3.76a) Weakening}\\[\lgap]
  \> \>   $(p \Until q) \lor (true \Until q)$\\[\lgap]
  \> $\Rightarrow$  \>  \Hint{(\ref{E:untilOrImp})}\\[\lgap]
  \> \>   $(p \lor true) \Until q$\\[\lgap]
  \> $=$  \>  \Hint{(3.29) Zero of $\lor$}\\[\lgap]
  \> \>   $true \Until q$\\[\lgap]
  \> $=$  \>  \Hint{(\ref{E:defEvent}) Definition of $\Event$}\\[\lgap]
  \> \>   $\Event q$\\[\lgap]
\end{tabbing}
\myqed\\[\lgap]


\begin{equation}\label{E:eventTrue}
\textbf{Truth:}\quad \Event true \equiv true
\end{equation}

\emph{Proof:}
\begin{tabbing}
\hspace{\mymathindent} \= $= \;$ \= \kill
  \> \>   $\Event true$\\[\lgap]
  \> $=$  \>  \Hint{(\ref{E:defEvent}) Definition of $\Event$}\\[\lgap]
  \> \>   $true \Until true$\\[\lgap]
  \> $=$  \>  \Hint{(\ref{E:idemUntil}) Idempotency of $\Until$}\\[\lgap]
  \> \>   $true$\\[\lgap]
\end{tabbing}
\myqed\\[\lgap]


\begin{equation}\label{E:eventFalse}
\textbf{Falsehood:}\quad \Event false \equiv false
\end{equation}

\emph{Proof:}
\begin{tabbing}
\hspace{\mymathindent} \= $= \;$ \= \kill
  \> \>   $\Event false$\\[\lgap]
  \> $=$  \>  \Hint{(\ref{E:defEvent}) Definition of $\Event$}\\[\lgap]
  \> \>   $true \Until false$\\[\lgap]
  \> $=$  \>  \Hint{(\ref{E:untilFalse})}\\[\lgap]
  \> \>   $false$\\[\lgap]
\end{tabbing}
\myqed\\[\lgap]


\begin{equation}\label{E:expansionEvent}
\textbf{Expansion of $\Event$:}\quad \Event p \equiv p \lor \Next\Event p
\end{equation}

\emph{Proof:}
\begin{tabbing}
\hspace{\mymathindent} \= $= \;$ \= \kill
  \> \>   $\Event p$\\[\lgap]
  \> $=$  \>  \Hint{(\ref{E:defEvent}) Definition of $\Event$}\\[\lgap]
  \> \>   $true \Until p$\\[\lgap]
  \> $=$  \>  \Hint{(\ref{E:expansionUntil}) Expansion of $\Until$}\\[\lgap]
  \> \>   $p \lor (true \land \Next(true \Until p))$\\[\lgap]
  \> $=$  \>  \Hint{(\ref{E:defEvent}) Definition of $\Event$}\\[\lgap]
  \> \>   $p \lor (true \land \Next\Event p)$\\[\lgap]
  \> $=$  \>  \Hint{(3.39) Identity of $\land$}\\[\lgap]
  \> \>   $p \lor \Next\Event p$\\[\lgap]
\end{tabbing}
\myqed\\[\lgap]

\begin{equation}\label{E:impEvent}
\textbf{Weakening of $\Event$:}\quad p \Rightarrow \Event p
\end{equation}

\emph{Proof:}
\begin{tabbing}
\hspace{\mymathindent} \= $= \;$ \= \kill
  \> \>   $\Event p$\\[\lgap]
  \> $=$  \>  \Hint{(\ref{E:expansionEvent}) Expansion of $\Event$}\\[\lgap]
  \> \>   $p \lor \Next\Event p$\\[\lgap]
  \> $\Leftarrow$  \>  \Hint{(3.76a) Weakening}\\[\lgap]
  \> \>   $p$\\[\lgap]
\end{tabbing}
\myqed\\[\lgap]


\begin{equation}\label{E:nextEvent}
\textbf{Weakening of $\Event$:}\quad \Next p \Rightarrow \Event p
\end{equation}

\emph{Proof:}
\begin{tabbing}
\hspace{\mymathindent} \= $= \;$ \= \kill
  \> \>   $\Event p$\\[\lgap]
  \> $=$  \>  \Hint{(\ref{E:defEvent}) Definition of $\Event$}\\[\lgap]
  \> \>   $true \Until p$\\[\lgap]
  \> $=$  \>  \Hint{(\ref{E:expansionUntil}) Expansion of $\Until$}\\[\lgap]
  \> \>   $p \lor (true \land \Next(true \Until p))$\\[\lgap]
  \> $=$  \>  \Hint{(3.39) Identity of $\land$}\\[\lgap]
  \> \>   $p \lor \Next(true \Until p)$\\[\lgap]
  \> $=$  \>  \Hint{(\ref{E:distNextUntil}) Distributivity of $\Next$ over $\Until$}\\[\lgap]
  \> \>   $p \lor (\Next true \Until \Next p)$\\[\lgap]
  \> $=$  \>  \Hint{(\ref{E:expansionUntil}) Expansion of $\Until$}\\[\lgap]
  \> \>   $p \lor \Next p \lor (\Next true \land \Next(\Next true \Until \Next p))$\\[\lgap]
  \> $\Leftarrow$ \> \Hint{(3.76a) Weakening}\\[\lgap]
  \> \>   $\Next p$\\[\lgap]
\end{tabbing}
\myqed\\[\lgap]


\begin{equation}\label{E:IdemEvent}
\textbf{Absorption of $\Event$:}\quad \Event\Event p \equiv \Event p
\end{equation}

\emph{Proof:}
\begin{tabbing}
\hspace{\mymathindent} \= $= \;$ \= \kill
  \> \>   $\Event\Event p$\\[\lgap]
  \> $=$  \>  \Hint{(\ref{E:defEvent}) Definition of $\Event$, with $p := \Event p$}\\[\lgap]
  \> \>   $true \Until \Event p$\\[\lgap]
  \> $=$  \>  \Hint{(\ref{E:defEvent}) Definition of $\Event$}\\[\lgap]
  \> \>   $true \Until (true \Until p)$\\[\lgap]
  \> $=$  \>  \Hint{(\ref{E:untilIdem}) with $p,q := true,p$}\\[\lgap]
  \> \>   $true \Until p$\\[\lgap]
  \> $=$  \>  \Hint{(\ref{E:defEvent}) Definition of $\Event$}\\[\lgap]
  \> \>   $\Event p$\\[\lgap]
\end{tabbing}
\myqed\\[\lgap]


\begin{equation}\label{E:dNextEvent}
\Next\Event p \equiv \Event\Next p
\end{equation}

\emph{Proof:}
\begin{tabbing}
\hspace{\mymathindent} \= $= \;$ \= \kill
  \> \>   $\Next\Event p$\\[\lgap]
  \> $=$  \>  \Hint{(\ref{E:defEvent}) Definition of $\Event$}\\[\lgap]
  \> \>   $\Next(true \Until p)$\\[\lgap]
  \> $=$  \>  \Hint{(\ref{E:distNextUntil}) Distributivity of $\Next$ over $\Until$}\\[\lgap]
  \> \>   $\Next true \Until \Next p$\\[\lgap]
  \> $=$  \>  \Hint{(\ref{E:nextTruth})}\\[\lgap]
  \> \>   $true \Until \Next p$\\[\lgap]
  \> $=$  \>  \Hint{(\ref{E:defEvent}) Definition of $\Event$}\\[\lgap]
  \> \>   $\Event\Next p$\\[\lgap]
\end{tabbing}
\myqed\\[\lgap]


\begin{equation}\label{E:distEventOr}
\textbf{Distributivity of $\Event$ over $\lor$:}\quad \Event(p \lor q) \equiv \Event p \lor \Event q
\end{equation}

\emph{Proof:}
\begin{tabbing}
\hspace{\mymathindent} \= $= \;$ \= \kill
  \> \>   $\Event(p \lor q)$\\[\lgap]
  \> $=$  \>  \Hint{(\ref{E:defEvent}) Definition of $\Event$}\\[\lgap]
  \> \>   $true \Until (p \lor q)$\\[\lgap]
  \> $=$  \>  \Hint{(\ref{E:untilOrEquiv})}\\[\lgap]
  \> \>   $(true \Until p) \lor (true \Until q)$\\[\lgap]
  \> $=$  \>  \Hint{(\ref{E:defEvent}) Definition of $\Event$ twice}\\[\lgap]
  \> \>   $\Event p \lor \Event q$\\[\lgap]
\end{tabbing}
\myqed\\[\lgap]


\begin{equation}\label{E:distEventAnd}
\textbf{Distributivity of $\Event$ over $\land$:}\quad \Event(p \land q) \Rightarrow \Event p \land \Event q
\end{equation}

\emph{Proof:}
\begin{tabbing}
\hspace{\mymathindent} \= $= \;$ \= \kill
  \> \>   $\Event(p \land q)$\\[\lgap]
  \> $=$  \>  \Hint{(\ref{E:defEvent}) Definition of $\Event$}\\[\lgap]
  \> \>   $true \Until (p \land q)$\\[\lgap]
  \> $\Rightarrow$  \>  \Hint{(\ref{E:untilAndImp})}\\[\lgap]
  \> \>   $(true \Until p) \land (true \Until q)$\\[\lgap]
  \> $=$  \>  \Hint{(\ref{E:defEvent}) Definition of $\Event$ twice}\\[\lgap]
  \> \>   $\Event p \land \Event q$\\[\lgap]
\end{tabbing}
\myqed\\[\lgap]

\subsection{Always}

\begin{equation}\label{E:defAlways}
\textbf{Definition of $\Always$:}\quad \Always p \equiv \lnot\Event\lnot p
\end{equation}


\begin{equation}\label{E:dualAlways}
\textbf{Dual of $\Always$:}\quad \lnot\Always p \equiv \Event\lnot p
\end{equation}

\emph{Proof:}
\begin{tabbing}
\hspace{\mymathindent} \= $= \;$ \= \kill
  \> \>   $\lnot\Always p \equiv \Event\lnot p$\\[\lgap]
  \> $=$  \>  \Hint{(3.11) with $p,q := \Always p, \Event\lnot p$}\\[\lgap]
  \> \>   $\Always p \equiv \lnot\Event\lnot p$
\end{tabbing}
which is (\ref{E:defAlways}). \myqed\\[\lgap]


\begin{equation}\label{E:dualEvent}
\textbf{Dual of $\Event$:}\quad \lnot\Event p \equiv \Always\lnot p
\end{equation}

\emph{Proof:}
\begin{tabbing}
\hspace{\mymathindent} \= $= \;$ \= \kill
  \> \>   $\Always\lnot p$\\[\lgap]
  \> $=$  \>  \Hint{(\ref{E:defAlways}) Definition of $\Always$}\\[\lgap]
  \> \>   $\lnot\Event\lnot\lnot p$\\[\lgap]
  \> $=$  \>  \Hint{(3.12) Double Negation}\\[\lgap]
  \> \>   $\lnot\Event p$\\[\lgap]
\end{tabbing}
\myqed\\[\lgap]


\begin{equation}\label{E:eventAsAlways}
\Event p \equiv \lnot\Always\lnot p
\end{equation}

\emph{Proof:}
\begin{tabbing}
\hspace{\mymathindent} \= $= \;$ \= \kill
  \> \>   $\lnot\Always\lnot p$\\[\lgap]
  \> $=$  \>  \Hint{(\ref{E:defAlways}) Definition of $\Always$}\\[\lgap]
  \> \>   $\lnot\lnot\Event\lnot\lnot p$\\[\lgap]
  \> $=$  \>  \Hint{(3.12) Double Negation, twice}\\[\lgap]
  \> \>   $\Event p$\\[\lgap]
\end{tabbing}
\myqed\\[\lgap]


\begin{equation}\label{E:alwaysTrue}
\textbf{Truth:}\quad \Always true \equiv true
\end{equation}

\emph{Proof:}
\begin{tabbing}
\hspace{\mymathindent} \= $= \;$ \= \kill
  \> \>   $\Always true$\\[\lgap]
  \> $=$  \>  \Hint{(\ref{E:defAlways}) Definition of $\Always$}\\[\lgap]
  \> \>   $\lnot\Event\lnot true$\\[\lgap]
  \> $=$  \>  \Hint{(3.8) Definition of $false$}\\[\lgap]
  \> \>   $\lnot\Event false$\\[\lgap]
  \> $=$  \>  \Hint{(\ref{E:eventFalse}) Falsehood}\\[\lgap]
  \> \>   $\lnot false$\\[\lgap]
  \> $=$  \>  \Hint{(3.13) Negation of $false$}\\[\lgap]
  \> \>   $true$\\[\lgap]
\end{tabbing}
\myqed\\[\lgap]


\begin{equation}\label{E:alwaysFalse}
\textbf{Falsehood:}\quad \Always false \equiv false
\end{equation}

\emph{Proof:}
\begin{tabbing}
\hspace{\mymathindent} \= $= \;$ \= \kill
  \> \>   $\Always false \equiv false$\\[\lgap]
  \> $=$  \>  \Hint{(3.8) Definition of $false$, twice}\\[\lgap]
  \> \>   $\Always\lnot true \equiv \lnot true$\\[\lgap]
  \> $=$  \>  \Hint{(3.11)}\\[\lgap]
  \> \>   $\lnot\Always\lnot\ true \equiv true$\\[\lgap]
  \> $=$  \>  \Hint{(\ref{E:eventAsAlways})}\\[\lgap]
  \> \>   $\Event true \equiv true$\\[\lgap]
\end{tabbing}
which is (\ref{E:eventTrue}) Truth. \myqed\\[\lgap]


\begin{equation}\label{E:expansionAlways}
\textbf{Expansion of $\Always$:}\quad \Always p \equiv p \land \Next\Always p
\end{equation}

\emph{Proof:}
\begin{tabbing}
\hspace{\mymathindent} \= $= \;$ \= \kill
  \> \>   $\Always p$\\[\lgap]
  \> $=$  \>  \Hint{(\ref{E:defAlways}) Definition of $\Always$}\\[\lgap]
  \> \>   $\lnot\Event\lnot p$\\[\lgap]
  \> $=$  \>  \Hint{(\ref{E:defEvent}) Definition of $\Event$ with $p := \lnot p$}\\[\lgap]
  \> \>   $\lnot(true \Until \lnot p)$\\[\lgap]
  \> $=$  \>  \Hint{(\ref{E:expansionUntil}) Expansion of $\Until$}\\[\lgap]
  \> \>   $\lnot(\lnot p \lor (true \land \Next(true \Until \lnot p)))$\\[\lgap]
  \> $=$  \>  \Hint{(3.39) Identity of $\land$}\\[\lgap]
  \> \>   $\lnot(\lnot p \lor \Next(true \Until \lnot p))$\\[\lgap]
  \> $=$  \>  \Hint{(3.47b) De Morgan's Law}\\[\lgap]
  \> \>   $\lnot\lnot p \land \lnot\Next(true \Until \lnot p)$\\[\lgap]
  \> $=$  \>  \Hint{(3.12) Double Negation}\\[\lgap]
  \> \>   $p \land \lnot\Next(true \Until \lnot p)$\\[\lgap]
  \> $=$  \>  \Hint{(\ref{E:defEvent}) Defintion of $\Event$}\\[\lgap]
  \> \>   $p \land \lnot\Next\Event\lnot p$\\[\lgap]
  \> $=$  \>  \Hint{(\ref{E:dualAlways}) Dual of $\Always$}\\[\lgap]
  \> \>   $p \land \lnot\Next\lnot\Always p$\\[\lgap]
  \> $=$  \>  \Hint{(\ref{E:linearity}) Linearity}\\[\lgap]
  \> \>   $p \land \Next\Always p$\\[\lgap]
\end{tabbing}
\myqed\\[\lgap]


\begin{equation}\label{E:IdemAlways}
\textbf{Absorption of $\Always$:}\quad \Always\Always p \equiv \Always p
\end{equation}

\emph{Proof:}
\begin{tabbing}
\hspace{\mymathindent} \= $= \;$ \= \kill
  \> \>   $\Always\Always p \equiv \Always p$\\[\lgap]
  \> $=$  \>  \Hint{(\ref{E:defAlways}) Definition of $\Always$ with $p := \Always p$}\\[\lgap]
  \> \>   $\lnot\Event\lnot\Always p \equiv \Always p$\\[\lgap]
  \> $=$  \>  \Hint{(3.11) with $p,q := \Event\lnot\Always p, \Always p$}\\[\lgap]
  \> \>   $\Event\lnot\Always p \equiv \lnot\Always p$\\[\lgap]
  \> $=$  \>  \Hint{(\ref{E:dualAlways}) Dual of $\Always$, twice}\\[\lgap]
  \> \>   $\Event\Event\lnot p \equiv \Event\lnot p$\\[\lgap]
  \> $=$  \>  \Hint{(\ref{E:IdemEvent}) Absorption of $\Event$}\\[\lgap]
  \> \>   $\Event\lnot p \equiv \Event\lnot p$\\[\lgap]
\end{tabbing}
which is (3.5) with $p := \Event\lnot p$. \myqed\\[\lgap]


\begin{equation}\label{E:dNextAlways}
\Next\Always p \equiv \Always\Next p
\end{equation}

\emph{Proof:}
\begin{tabbing}
\hspace{\mymathindent} \= $= \;$ \= \kill
  \> \>   $\Next\Always p$\\[\lgap]
  \> $=$  \>  \Hint{(\ref{E:defAlways}) Definition of $\Always$}\\[\lgap]
  \> \>   $\Next\lnot\Event\lnot p$\\[\lgap]
  \> $=$  \>  \Hint{(\ref{E:selfDual}) Self-dual}\\[\lgap]
  \> \>   $\lnot\Next\Event\lnot p$\\[\lgap]
  \> $=$  \>  \Hint{(\ref{E:dNextEvent}) with $p := \lnot p$}\\[\lgap]
  \> \>   $\lnot\Event\Next\lnot p$\\[\lgap]
  \> $=$  \>  \Hint{(\ref{E:selfDual}) Self-dual}\\[\lgap]
  \> \>   $\lnot\Event\lnot\Next p$\\[\lgap]
  \> $=$  \>  \Hint{(\ref{E:defAlways}) Definition of $\Always$}\\[\lgap]
  \> \>   $\Always\Next p$\\[\lgap]
\end{tabbing}
\myqed\\[\lgap]


\begin{equation}\label{E:impAlways}
\textbf{Strengthening of $\Always$:}\quad \Always p \Rightarrow p
\end{equation}

\emph{Proof:}
\begin{tabbing}
\hspace{\mymathindent} \= $= \;$ \= \kill
  \> \>   $\Always p$\\[\lgap]
  \> $=$  \>  \Hint{(\ref{E:defAlways}) Definition of $\Always$}\\[\lgap]
  \> \>   $\lnot\Event\lnot p$\\[\lgap]
  \> $=$  \>  \Hint{(\ref{E:expansionEvent}) Expansion of $\Event$}\\[\lgap]
  \> \>   $\lnot(\lnot p \lor \Next\Event\lnot p)$\\[\lgap]
  \> $=$  \>  \Hint{(3.47b) De Morgan's Law}\\[\lgap]
  \> \>   $\lnot\lnot p \land \lnot\Next\Event\lnot p$\\[\lgap]
  \> $=$  \>  \Hint{(3.12) Double Negation}\\[\lgap]
  \> \>   $p \land \lnot\Next\Event\lnot p$\\[\lgap]
  \> $\Rightarrow$  \>  \Hint{(3.76b) Strengthening}\\[\lgap]
  \> \>   $p$\\[\lgap]
\end{tabbing}
\myqed\\[\lgap]


\begin{equation}\label{E:impAlwaysE}
\textbf{Strengthening of $\Always$:}\quad \Always p \Rightarrow \Event p
\end{equation}

\emph{Proof:}
\begin{tabbing}
\hspace{\mymathindent} \= $= \;$ \= \kill
  \> \>   $\Always p$\\[\lgap]
  \> $\Rightarrow$  \>  \Hint{(\ref{E:impAlways}) Strengthening of $\Always$}\\[\lgap]
  \> \>   $p$\\[\lgap]
  \> $\Rightarrow$  \>  \Hint{(\ref{E:impEvent}) Weakening of $\Event$}\\[\lgap]
  \> \>   $\Event p$\\[\lgap]
\end{tabbing}
\myqed\\[\lgap]


\begin{equation}\label{E:impAlwaysN}
\textbf{Strengthening of $\Always$:}\quad \Always p \Rightarrow \Next p
\end{equation}

\emph{Proof:}
\begin{tabbing}
\hspace{\mymathindent} \= $= \;$ \= \kill
  \> \>   $\Always p$\\[\lgap]
  \> $=$  \>  \Hint{(\ref{E:expansionAlways}) Expansion of $\Always$}\\[\lgap]
  \> \>   $p \land \Next\Always p$\\[\lgap]
  \> $=$  \>  \Hint{(\ref{E:dNextAlways})}\\[\lgap]
  \> \>   $p \land \Always\Next p$\\[\lgap]
  \> $=$  \>  \Hint{(\ref{E:expansionAlways}) Expansion of $\Always$ with $p := \Next p$}\\[\lgap]
  \> \>   $p \land \Next p \land \Next\Always\Next p$\\[\lgap]
  \> $\Rightarrow$  \>  \Hint{(3.76b) Strengthening}\\[\lgap]
  \> \>   $\Next p$\\[\lgap]
\end{tabbing}
\myqed\\[\lgap]


\begin{equation}\label{E:impAlwaysNA}
\textbf{Strengthening of $\Always$:}\quad \Always p \Rightarrow \Next\Always p
\end{equation}

\emph{Proof:}
\begin{tabbing}
\hspace{\mymathindent} \= $= \;$ \= \kill
  \> \>   $\Always p$\\[\lgap]
  \> $=$  \>  \Hint{(\ref{E:expansionAlways}) Expansion of $\Always$}\\[\lgap]
  \> \>   $p \land \Next\Always p$\\[\lgap]
  \> $\Rightarrow$  \>  \Hint{(3.76b) Strengthening}\\[\lgap]
  \> \>   $\Next\Always p$\\[\lgap]
\end{tabbing}
\myqed\\[\lgap]


\begin{equation}\label{E:exAlwaysNot}
\Always\lnot p \Rightarrow \lnot\Always p
\end{equation}

\emph{Proof:}
\begin{tabbing}
\hspace{\mymathindent} \= $= \;$ \= \kill
  \> \>   $\Always\lnot p$\\[\lgap]
  \> $\Rightarrow$  \>  \Hint{(\ref{E:impAlwaysE}) Strengthening of $\Always$}\\[\lgap]
  \> \>   $\Event\lnot p$\\[\lgap]
  \> $=$  \>  \Hint{(\ref{E:dualAlways}) Dual of $\Always$}\\[\lgap]
  \> \>   $\lnot\Always p$\\[\lgap]
\end{tabbing}
\myqed\\[\lgap]


\begin{equation}\label{E:excludedMid}
\textbf{Excluded Middle:}\quad \Event p \lor \Always\lnot p
\end{equation}

\emph{Proof:}
\begin{tabbing}
\hspace{\mymathindent} \= $= \;$ \= \kill
  \> \>   $\Event p \lor \Always\lnot p$\\[\lgap]
  \> $=$  \>  \Hint{(\ref{E:dualEvent}) Dual of $\Event$}\\[\lgap]
  \> \>   $\Event p \lor \lnot\Event p$\\[\lgap]
\end{tabbing}
which is (3.28) Excluded middle, with $p := \Event p$. \myqed\\[\lgap]


\begin{equation}\label{E:distAlwaysAnd}
\textbf{Distributivity of $\Always$ over $\land$:}\quad \Always (p \land q) \equiv \Always p \land \Always q
\end{equation}

\emph{Proof:}
\begin{tabbing}
\hspace{\mymathindent} \= $= \;$ \= \kill
  \> \>   $\Always (p \land q)$\\[\lgap]
  \> $=$  \>  \Hint{(\ref{E:defAlways}) Definition of $\Always$}\\[\lgap]
  \> \>   $\lnot\Event\lnot (p \land q)$\\[\lgap]
  \> $=$  \>  \Hint{(3.47a) De Morgan}\\[\lgap]
  \> \>   $\lnot\Event (\lnot p \lor \lnot q)$\\[\lgap]
  \> $=$  \>  \Hint{(\ref{E:distEventOr}) Distributivity of $\Event$ over $\lor$}\\[\lgap]
  \> \>   $\lnot (\Event\lnot p \lor \Event\lnot q)$\\[\lgap]
  \> $=$  \>  \Hint{(3.47b) De Morgan}\\[\lgap]
  \> \>   $\lnot\Event\lnot p \land \lnot\Event\lnot q$\\[\lgap]
  \> $=$  \>  \Hint{(\ref{E:defAlways}) Definition of $\Always$, twice}\\[\lgap]
  \> \>   $\Always p \land \Always q$
\end{tabbing}
\myqed\\[\lgap]


\begin{equation}\label{E:distAlwaysOr}
\textbf{Distributivity of $\Always$ over $\lor$:}\quad (\Always p \lor \Always q) \Rightarrow \Always (p \lor q)
\end{equation}

\emph{Proof:}
\begin{tabbing}
\hspace{\mymathindent} \= $= \;$ \= \kill
  \> \>   $\Always p \lor \Always q \Rightarrow \Always(p \lor q)$\\[\lgap]
  \> $=$  \>  \Hint{(3.60) Implication}\\[\lgap]
  \> \>   $(\Always p \lor \Always q) \land \Always(p \lor q) \equiv \Always p \lor \Always q$\\[\lgap]
  \> $=$  \>  \Hint{(3.46) Distributivity of $\land$ over $\lor$}\\[\lgap]
  \> \>   $(\Always (p \lor q) \land \Always p) \lor (\Always (p \lor q) \land \Always q) \equiv \Always p \lor \Always q$\\[\lgap]
  \> $=$  \>  \Hint{(\ref{E:distAlwaysAnd}) Distributivity of $\Always$ over $\land$}\\[\lgap]
  \> \>   $\Always(p \land (p \lor q)) \lor \Always(q \land (p \lor q)) \equiv \Always p \lor \Always q$\\[\lgap]
  \> $=$  \>  \Hint{(3.43) Absorption, twice}\\[\lgap]
  \> \>   $\Always p \lor \Always q \equiv \Always p \lor \Always q$\\[\lgap]
\end{tabbing}
which is (3.5) Reflexivity of $\equiv$. \myqed\\[\lgap]


\begin{equation}\label{E:distAlwaysEquiv}
\textbf{Distributivity of $\Always$ over $\equiv$:}\quad \Always (p \equiv q) \Rightarrow (\Always p \equiv \Always q)
\end{equation}

\emph{Proof:}
\begin{tabbing}
\hspace{\mymathindent} \= $= \;$ \= \kill
  \> \>   $\Always (p \equiv q) \Rightarrow (\Always p \equiv \Always q)$\\[\lgap]
  \> $=$  \>  \Hint{(3.62)}\\[\lgap]
  \> \>   $\Always (p \equiv q) \land \Always p \equiv \Always (p \equiv q) \land \Always q$\\[\lgap]
  \> $=$  \>  \Hint{(\ref{E:distAlwaysAnd}) Distributivity of $\Always$ over $\land$, twice}\\[\lgap]
  \> \>   $\Always((p \equiv q) \land p) \equiv \Always((p \equiv q) \land q)$\\[\lgap]
  \> $=$  \>  \Hint{(3.50), twice}\\[\lgap]
  \> \>   $\Always(p \land q) \equiv \Always (p \land q)$\\[\lgap]
\end{tabbing}
which is (3.5) Reflexivity of $\equiv$. \myqed\\[\lgap]


\begin{equation}\label{E:distAlwaysImp}
\textbf{Distributivity of $\Always$ over $\Rightarrow$:}\quad \Always (p \Rightarrow q) \Rightarrow (\Always p \Rightarrow \Always q)
\end{equation}

\emph{Proof:}
\begin{tabbing}
\hspace{\mymathindent} \= $= \;$ \= \kill
  \> \>   $\Always (p \Rightarrow q)$\\[\lgap]
  \> $=$  \>  \Hint{(3.60) Implication}\\[\lgap]
  \> \>   $\Always (p \land q \equiv p)$\\[\lgap]
  \> $\Rightarrow$  \>  \Hint{(\ref{E:distAlwaysEquiv}) Distributivity of $\Always$ over $\equiv$}\\[\lgap]
  \> \>   $\Always(p \land q) \equiv \Always p$\\[\lgap]
  \> $=$  \>  \Hint{(\ref{E:distAlwaysAnd}) Distributivity of $\Always$ over $\land$}\\[\lgap]
  \> \>   $\Always p \land \Always q \equiv \Always p$\\[\lgap]
  \> $=$  \>  \Hint{(3.60) Implication}\\[\lgap]
  \> \>   $\Always p \Rightarrow \Always q$\\[\lgap]
\end{tabbing}
\myqed\\[\lgap]


\begin{equation}\label{E:distAlwaysEventAnd}
\textbf{Distributivity of $\Always\Event$ over $\land$:}\quad \Always\Event(p \land q) \Rightarrow \Always\Event p \land \Always\Event q
\end{equation}

\emph{Proof:}
\begin{tabbing}
\hspace{\mymathindent} \= $= \;$ \= \kill
  \> \>   $\Always\Event(p \land q) \Rightarrow \Always\Event p \land \Always\Event q$\\[\lgap]
  \> $=$  \>  \Hint{(3.60) Implication}\\[\lgap]
  \> \>   $\Always\Event(p \land q) \land \Always\Event p \land \Always\Event q \equiv \Always\Event(p \land q)$\\[\lgap]
  \> $=$  \>  \Hint{(\ref{E:distAlwaysAnd}) Distributivity of $\Always$ over $\land$}\\[\lgap]
  \> \>   $\Always(\Event(p \land q) \land \Event p \land \Event q) \equiv \Always\Event(p \land q)$\\[\lgap]
  \> $=$  \>  \Hint{Lemma: $\Event(p \land q) \land \Event p \land \Event q \equiv \Event(p \land q)$}\\[\lgap]
  \> \>   $\Always\Event(p \land q) \equiv \Always\Event(p \land q)$\\[\lgap]
\end{tabbing}
which is (3.5) Reflexivity of $\equiv$. \myqed\\[\lgap]

\emph{Proof of Lemma:}
\begin{tabbing}
\hspace{\mymathindent} \= $= \;$ \= \kill
  \> \>   $\Event(p \land q) \land \Event p \land \Event q \equiv \Event(p \land q)$\\[\lgap]
  \> $=$  \>  \Hint{(3.60) Implication}\\[\lgap]
  \> \>   $\Event(p \land q) \Rightarrow \Event p \land \Event q$\\[\lgap]
\end{tabbing}
which is (\ref{E:distEventAnd}) Distributivity of $\Event$ over $\land$. \myqed\\[\lgap]


\begin{equation}\label{E:distEventAlwaysOr}
\textbf{Distributivity of $\Event\Always$ over $\lor$:}\quad \Event\Always p \lor \Event\Always q \Rightarrow \Event\Always (p \lor q)
\end{equation}

\emph{Proof:}
\begin{tabbing}
\hspace{\mymathindent} \= $= \;$ \= \kill
  \> \>   $\Event\Always p \lor \Event\Always q \Rightarrow \Event\Always(p \lor q)$\\[\lgap]
  \> $=$  \>  \Hint{(3.57) Definition of Implication}\\[\lgap]
  \> \>   $\Event\Always p \lor \Event\Always q \lor \Event\Always(p \lor q) \equiv \Event\Always(p \lor q)$\\[\lgap]
  \> $=$  \>  \Hint{(\ref{E:distEventOr}) Distributivity of $\Event$ over $\lor$}\\[\lgap]
  \> \>   $\Event(\Always p \lor \Always q \lor \Always(p \lor q)) \equiv \Event\Always(p \lor q)$\\[\lgap]
  \> $=$  \>  \Hint{Lemma: $\Always p \lor \Always q \lor \Always(p \lor q) \equiv \Always(p \lor q)$}\\[\lgap]
  \> \>   $\Event\Always(p \lor q) \equiv \Event\Always(p \lor q)$\\[\lgap]
\end{tabbing}
which is (3.5) Reflexivity of $\equiv$. \myqed\\[\lgap]

\emph{Proof of Lemma:}
\begin{tabbing}
\hspace{\mymathindent} \= $= \;$ \= \kill
  \> \>   $\Always p \lor \Always q \lor \Always(p \lor q) \equiv \Always(p \lor q)$\\[\lgap]
  \> $=$  \>  \Hint{(3.57) Definition of Implication}\\[\lgap]
  \> \>   $\Always p \lor \Always q \Rightarrow \Always(p \lor q)$\\[\lgap]
\end{tabbing}
which is (\ref{E:distAlwaysOr}) Distributivity of $\Always$ over $\lor$. \myqed\\[\lgap]


\begin{equation}\label{E:distAlwaysEventOr}
\textbf{Distributivity of $\Always\Event$ over $\lor$:}\quad \Always\Event(p \lor q) \equiv \Always\Event p \lor \Always\Event q
\end{equation}


\begin{equation}\label{E:distEventAlwaysAnd}
\textbf{Distributivity of $\Event\Always$ over $\land$:}\quad \Event\Always(p \land q) \equiv (\Event\Always p \land \Event\Always q)
\end{equation}


\begin{equation}\label{E:eventAlwaysImp}
\Event\Always p \Rightarrow \Always\Event p
\end{equation}



\begin{equation}\label{E:absEvent}
\textbf{Absorption of $\Event$ into $\Always$:}\quad \Event\Always\Event p \equiv \Always\Event p
\end{equation}



\begin{equation}\label{E:absAlways}
\textbf{Absorption of $\Always$ into $\Event$:}\quad \Always\Event\Always p \equiv \Event\Always p
\end{equation}



\begin{equation}\label{E:induction}
\textbf{Induction:}\quad \Always (p \Rightarrow \Next p) \Rightarrow (p \Rightarrow \Always p)
\end{equation}



\begin{equation}\label{E:alwaysImpNexts}
\textbf{Monotonicity of $\Next$:}\quad \Always (p \Rightarrow q) \Rightarrow (\Next p \Rightarrow \Next q)
\end{equation}

\emph{Proof:}
\begin{tabbing}
\hspace{\mymathindent} \= $= \;$ \= \kill
  \> \>   $\Always (p \Rightarrow q)$\\[\lgap]
  \> $\Rightarrow$  \>  \Hint{(\ref{E:impAlwaysN}) Strengthening}\\[\lgap]
  \> \>   $\Next (p \Rightarrow q)$\\[\lgap]
  \> $=$  \>  \Hint{(\ref{E:distNextImp}) Distributivity of $\Next$ over $\Rightarrow$}\\[\lgap]
  \> \>   $\Next p \Rightarrow \Next q$\\[\lgap]  
\end{tabbing}
\myqed\\[\lgap]



\begin{equation}\label{E:alwaysImpEvents}
\textbf{Monotonicity of $\Event$:}\quad \Always (p \Rightarrow q) \Rightarrow (\Event p \Rightarrow \Event q)
\end{equation}

\emph{Proof:}
\begin{tabbing}
\hspace{\mymathindent} \= $= \;$ \= \kill
  \> \>   $\Always (p \Rightarrow q) \Rightarrow (\Event p \Rightarrow \Event q)$\\[\lgap]
  \> $=$  \>  \Hint{(3.59) Implication, three times}\\[\lgap]
  \> \>   $\lnot\Always (\lnot p \lor q) \lor \lnot\Event p \lor \Event q$\\[\lgap]
  \> $=$  \>  \Hint{(\ref{E:dualAlways}) Dual of $\Always$}\\[\lgap]
  \> \>   $\Event\lnot (\lnot p \lor q) \lor \lnot\Event p \lor \Event q$\\[\lgap]  
  \> $=$  \>  \Hint{(3.47b) De Morgan, and Double negation}\\[\lgap]
  \> \>   $\Event(p \land \lnot q) \lor \lnot\Event p \lor \Event q$\\[\lgap]
  \> $=$  \>  \Hint{(\ref{E:distEventOr}) Distributivity of $\Event$ over $\lor$}\\[\lgap]
  \> \>   $\Event((p \land \lnot q) \lor q) \lor \lnot\Event p$\\[\lgap]
  \> $=$  \>  \Hint{(3.44b) Absorption}\\[\lgap]
  \> \>   $\Event(p \lor q) \lor \lnot\Event p$\\[\lgap]
  \> $=$  \>  \Hint{(\ref{E:distEventOr}) Distributivity of $\Event$ over $\lor$}\\[\lgap]
  \> \>   $\Event p \lor \Event q \lor \lnot\Event p$\\[\lgap]
  \> $=$  \>  \Hint{(3.28) Excluded Middle, with $p := \Event p$}\\[\lgap]
  \> \>   $\Event q \lor true$\\[\lgap]
  \> $=$  \>  \Hint{(3.29) Zero of $\lor$}\\[\lgap]
  \> \>   $true$\\[\lgap]
\end{tabbing}
\myqed\\[\lgap]


\begin{equation}\label{E:eventImpAlways}
\Event (p \Rightarrow q) \equiv (\Always p \Rightarrow \Event q)
\end{equation}

\emph{Proof:}
\begin{tabbing}
\hspace{\mymathindent} \= $= \;$ \= \kill
  \> \>   $\Event(p \Rightarrow q)$\\[\lgap]
  \> $=$  \>  \Hint{(3.59) Implication}\\[\lgap]
  \> \>   $\Event(\lnot p \lor q)$\\[\lgap]
  \> $=$  \>  \Hint{(\ref{E:distEventOr}) Distributivity of $\Event$ over $\lor$}\\[\lgap]
  \> \>   $\Event\lnot p \lor \Event q$\\[\lgap]
  \> $=$  \>  \Hint{(\ref{E:dualAlways}) Dual of $\Always$}\\[\lgap]
  \> \>   $\lnot\Always p \lor \Event q$\\[\lgap]
  \> $=$  \>  \Hint{(3.59) Implication}\\[\lgap]
  \> \>   $\Always p \Rightarrow \Event q$\\[\lgap]  
\end{tabbing}
\myqed\\[\lgap]


\begin{equation}\label{E:alwaysAndEvent}
\Always p \land \Event q \Rightarrow \Event (p \land q)
\end{equation}

\emph{Proof:}
\begin{tabbing}
\hspace{\mymathindent} \= $= \;$ \= \kill
  \> \>   $\Always p \land \Event q \Rightarrow \Event (p \land q)$\\[\lgap]
  \> $=$  \>  \Hint{(3.59) Implication}\\[\lgap]
  \> \>   $\lnot(\Always p \land \Event q) \lor \Event(p \land q)$\\[\lgap]
  \> $=$  \>  \Hint{(3.47a) De Morgan}\\[\lgap]
  \> \>   $\lnot\Always p \lor \lnot\Event q \lor \Event(p \land q)$\\[\lgap]
  \> $=$  \>  \Hint{(\ref{E:dualAlways}) Dual of $\Always$}\\[\lgap]
  \> \>   $\Event\lnot p \lor \lnot\Event q \lor \Event(p \land q)$\\[\lgap]
  \> $=$  \>  \Hint{(\ref{E:distEventOr}) Distributivity of $\Event$ over $\lor$}\\[\lgap]
  \> \>   $\Event(\lnot p \lor (p \land q)) \lor \lnot\Event q$\\[\lgap]
  \> $=$  \>  \Hint{(3.44b) Absorption}\\[\lgap]
  \> \>   $\Event(\lnot p \lor q) \lor \lnot\Event q$\\[\lgap]
  \> $=$  \>  \Hint{(\ref{E:distEventOr}) Distributivity of $\Event$ over $\lor$}\\[\lgap]
  \> \>   $\Event\lnot p \lor \Event q \lor \lnot\Event q$\\[\lgap]
  \> $=$  \>  \Hint{(3.28) Excluded middle}\\[\lgap]
  \> \>   $\Event\lnot p \lor true$\\[\lgap]
  \> $=$  \>  \Hint{(3.29) Zero of $\lor$}\\[\lgap]
  \> \>   $true$\\[\lgap]
\end{tabbing}
\myqed\\[\lgap]

\subsection{Wait}

\section{Conclusion}

The results are cooool!\\

\bibliographystyle{plain}
\bibliography{Vega-Paper}
\end{document}
