% David Vega and Stan Warford
% Pepperdine University
% File: Vega-Paper

\documentclass[fleqn, leqno]{article}

\usepackage{times}
\usepackage{amsmath, amsthm, amssymb,latexsym}

\newcommand{\lgap}{2pt}                             % Line gap
\newcommand{\llgap}{6pt}                            % Larger line gap
\newcommand{\lllgap}{32pt}                          % Largest line gap for students to write in
\newcommand{\mymathindent}{24pt}                      % Indentation for math tabbing
\newcommand{\equivs}{\ensuremath{\;\equiv\;}}       % Equivales with space
\newcommand{\equivss}{\ensuremath{\;\;\equiv\;\;}}  % Equivales with double space
\newcommand{\nequiv}{\ensuremath{\not\equiv}}       % Inequivalent
\newcommand{\impl}{\ensuremath{\Rightarrow}}        % Implies
\newcommand{\nimpl}{\ensuremath{\not\Rightarrow}}   % Does not imply
\newcommand{\foll}{\ensuremath{\Leftarrow}}         % Follows from
\newcommand{\nfoll}{\ensuremath{\not\Leftarrow}}    % Does not follow from

% Thanks to David Gries for sharing the following macros
% Macros for quantifications.
\newcommand{\thedr}{\rule[-.25ex]{.32mm}{1.75ex}}   % Symbol that separates dummy from range in quantification
\newcommand{\dr}{\;\,\thedr\,\;}                    % Symbol that separates dummy from range, with spacing
\newcommand{\rb}{:}                                 % Symbol that separates range from body in quantification
\newcommand{\drrb}{\;\thedr\,{:}\;}                 % Symbol that separates dummy from body when range is missing
\newcommand{\all}{\forall}                          % Universal quantification
\newcommand{\ext}{\exists}                          % Existential quantification

\newcommand{\myqed}{\hfill\rule[-.23ex]{1.2ex}{2.0ex}}
\newcommand{\until}{\;\mathcal{U}\;}

% Macros for proof hints
\newcommand{\Gll} {\langle}                         % Open hint
\newcommand{\Ggg} {\rangle}                         % Close hint
\newlength{\Glllength}                              % Length of open hint symbol
\settowidth{\Glllength}{$.\Gll$}
\newcommand{\Hint}[1]     {\ \ \ $\Gll              \mbox{#1} \Ggg$ }   % Single line hint
\newcommand{\Hintfirst}[1]{\ \ \ $\Gll              \mbox{#1}$ }        % First line of multiline hint
\newcommand{\Hintmid}[1]  {\ \ $\hspace{\Glllength} \mbox{#1}$ }        % Middle line of multiline hint
\newcommand{\Hintlast}[1] {\ \ $\hspace{\Glllength} \mbox{#1} \Ggg$ }   % Last line of multiline hint


% Single and double quotes
\newcommand{\Lq}{\mbox{`}}
\newcommand{\Rq}{\mbox{'}}
\newcommand{\Lqq}{\mbox{``}}
\newcommand{\Rqq}{\mbox{''}}

\oddsidemargin  0.0in
\evensidemargin 0.0in
\textwidth      6.0in
\headheight     0.0in
\topmargin      0.0in
\textheight=8.5in
\parindent=0in
\pagestyle{plain}

\title{An Equational Deductive System\\for Linear Temporal Logic}
\author{David Vega\thanks{Research supported by Tooma Undergraduate Research Fellowship Program, Summer 2009}\\
   Computer Science Department\\
   Pepperdine University\\
   Malibu, CA 90265
   \and
   J. Stanley Warford\\
   Computer Science Department\\
   Pepperdine University\\
   Malibu, CA 90265}
\date{} % Required for no date to appear in heading

\begin{document}
\maketitle


\begin{equation}\label{E:selfDual}
\textbf{Axiom, Self-dual:}\quad \bigcirc\lnot p \equiv \lnot\bigcirc p
\end{equation}

\begin{equation}\label{E:distNextImp}
\textbf{Axiom, Distributivity of $\bigcirc$ over $\Rightarrow$:}\quad \bigcirc (p \Rightarrow q) \equiv \bigcirc p \Rightarrow \bigcirc q
\end{equation}

\begin{equation}\label{E:nextTruth}
\textbf{Axiom, Truth:}\quad \bigcirc true \equiv true
\end{equation}

\begin{equation}\label{E:linearity}
\textbf{Linearity:}\quad \bigcirc p \equiv \lnot\bigcirc\lnot p
\end{equation}

\begin{equation}\label{E:nextFalse}
\textbf{Falsehood:}\quad \bigcirc false \equiv false
\end{equation}

\begin{equation}\label{E:distNextOr}
\textbf{Distributivity of $\bigcirc$ over $\lor$:}\quad \bigcirc (p \lor q) \equiv \bigcirc p \lor \bigcirc q
\end{equation}

\begin{equation}\label{E:distNextAnd}
\textbf{Distributivity of $\bigcirc$ over $\land$:}\quad \bigcirc (p \land q) \equiv \bigcirc p \land \bigcirc q
\end{equation}

\begin{equation}\label{E:distNextEquiv}
\textbf{Distributivity of $\bigcirc$ over $\equiv$:}\quad \bigcirc (p \equiv q) \equiv \bigcirc p \equiv \bigcirc q
\end{equation}

\begin{equation}\label{E:distNextUntil}
\textbf{Axiom, Distributivity of $\bigcirc$ over $\until$:}\quad \bigcirc (p \until q) \equiv \bigcirc p \until \bigcirc q
\end{equation}

\begin{equation}\label{E:expansionUntil}
\textbf{Axiom, Expansion of $\until$:}\quad p \until q \equiv q \lor (p \land \bigcirc (p \until q))
\end{equation}

Note: we will take the following 6 expressions of until (which we know to be correct) as axioms for now:\\

\begin{equation}\label{E:untilOrImp}
\textbf{Axiom:}\quad (p \until r) \lor (q \until r) \Rightarrow (p \lor q) \until r
\end{equation}

\begin{equation}\label{E:untilAndImp}
\textbf{Axiom:}\quad p \until (q \land r) \Rightarrow (p \until q) \land (p \until r)
\end{equation}

\begin{equation}\label{E:untilAndEquiv}
\textbf{Axiom:}\quad (p \land q) \until r \equiv (p \until r) \land (q \until r)
\end{equation}

\begin{equation}\label{E:untilOrEquiv}
\textbf{Axiom:}\quad p \until (q \lor r) \equiv (p \until q) \lor (p \until r)
\end{equation}

\begin{equation}\label{E:untilIdem}
\textbf{Axiom:}\quad p \until (p \until q) \equiv p \until q
\end{equation}

\begin{equation}\label{E:untilIdemR}
\textbf{Axiom:}\quad (p \until q) \until q \equiv p \until q
\end{equation}

End Note\\

\begin{equation}\label{E:defEvent}
\textbf{Axiom, Definition of $\Diamond$:}\quad \Diamond p \equiv true \until p
\end{equation}

\begin{equation}\label{E:untilImpEvent}
\textbf{Axiom:}\quad p \until q \Rightarrow \Diamond q
\end{equation}

\begin{equation}\label{E:nextEvent}
\textbf{Axiom:}\quad \bigcirc p \Rightarrow \Diamond p
\end{equation}


\begin{equation}\label{E:expansionEvent}
\textbf{Expansion of $\Diamond$:}\quad \Diamond p \equiv p \lor \bigcirc\Diamond p
\end{equation}

\begin{equation}\label{E:impEvent}
p \Rightarrow \Diamond p
\end{equation}

\begin{equation}\label{E:IdemEvent}
\textbf{Idempotency of $\Diamond$:}\quad \Diamond\Diamond p \equiv \Diamond p
\end{equation}

\begin{equation}\label{E:dNextEvent}
\bigcirc\Diamond p \equiv \Diamond\bigcirc p
\end{equation}

Note: Figure out what to do with the next 2 theorems, as well as devise something for $\Diamond$ over $\Rightarrow$.\\

\begin{equation}\label{E:distEventOr}
\Diamond(p \lor q) \equiv \Diamond p \lor \Diamond q
\end{equation}

\begin{equation}\label{E:distEventAnd}
\Diamond(p \land q) \Rightarrow \Diamond p \land \Diamond q
\end{equation}

End Note.\\

\begin{equation}\label{E:defAlways}
\textbf{Axiom, Definition of $\Box$:}\quad \Box p \equiv \lnot\Diamond\lnot p
\end{equation}

\begin{equation}\label{E:dualAlways}
\textbf{Dual of $\Box$:}\quad \lnot\Box p \equiv \Diamond\lnot p
\end{equation}

\begin{equation}\label{E:dualEvent}
\textbf{Dual of $\Diamond$:}\quad \lnot\Diamond p \equiv \Box\lnot p
\end{equation}

\begin{equation}\label{E:expansionEvent}
\textbf{Expansion of $\Box$:}\quad \Box p \equiv p \land \bigcirc\Box p
\end{equation}

\begin{equation}\label{E:IdemAlways}
\textbf{Idempotency of $\Box$:}\quad \Box\Box p \equiv \Box p
\end{equation}

\begin{equation}\label{E:dNextAlways}
\bigcirc\Box p \equiv \Box\bigcirc p
\end{equation}

\begin{equation}\label{E:ImpAlways}
\Box p \Rightarrow p
\end{equation}

\begin{equation}\label{E:ImpAlwaysE}
\Box p \Rightarrow \Diamond p
\end{equation}

\begin{equation}\label{E:ImpAlwaysN}
\Box p \Rightarrow \bigcirc p
\end{equation}

\begin{equation}\label{E:ImpAlwaysNA}
\Box p \Rightarrow \bigcirc\Box p
\end{equation}

\begin{equation}\label{E:excludedMid}
\textbf{Excluded Middle:}\quad \Diamond p \lor \Box\lnot p
\end{equation}

\begin{equation}\label{E:distAlwaysImp}
\textbf{Distributivity of $\Box$ over $\Rightarrow$:}\quad \Box (p \Rightarrow q) \Rightarrow (\Box p \Rightarrow \Box q)
\end{equation}

\begin{equation}\label{E:distAlwaysAnd}
\textbf{Distributivity of $\Box$ over $\land$:}\quad \Box (p \land q) \equiv (\Box p \land \Box q)
\end{equation}

\begin{equation}\label{E:distAlwaysOr}
\textbf{Distributivity of $\Box$ over $\lor$:}\quad (\Box p \lor \Box q) \Rightarrow \Box (p \lor q)
\end{equation}



\end{document}